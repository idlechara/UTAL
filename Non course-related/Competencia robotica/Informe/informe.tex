\documentclass[11pt]{utalcaDoc}
\usepackage{alltt}
\usepackage{underscore}
\usepackage[latin1]{inputenc}
\usepackage[activeacute,spanish]{babel}
\usepackage{verbatim}
\usepackage{graphicx}
\usepackage{ae}
\usepackage{amsmath}
\usepackage{algorithm}
\usepackage{algorithmic}
\floatname{algorithm}{Algoritmo}
\renewcommand{\listalgorithmname}{Lista de algoritmos}
\renewcommand{\algorithmicrequire}{\textbf{Precondici\'{o}n:}}
\renewcommand{\algorithmicensure}{\textbf{Postcondici\'{o}n:}}
\renewcommand{\algorithmicend}{\textbf{fin}}
\renewcommand{\algorithmicif}{\textbf{si}}
\renewcommand{\algorithmicthen}{\textbf{entonces}}
\renewcommand{\algorithmicelse}{\textbf{si no}}
\renewcommand{\algorithmicelsif}{\algorithmicelse,\ \algorithmicif}
\renewcommand{\algorithmicendif}{\algorithmicend\ \algorithmicif}
\renewcommand{\algorithmicfor}{\textbf{para}}
\renewcommand{\algorithmicforall}{\textbf{para todo}}
\renewcommand{\algorithmicdo}{\textbf{hacer}}
\renewcommand{\algorithmicendfor}{\algorithmicend\ \algorithmicfor}
\renewcommand{\algorithmicwhile}{\textbf{mientras}}
\renewcommand{\algorithmicendwhile}{\algorithmicend\ \algorithmicwhile}
\renewcommand{\algorithmicloop}{\textbf{repetir}}
\renewcommand{\algorithmicendloop}{\algorithmicend\ \algorithmicloop}
\renewcommand{\algorithmicrepeat}{\textbf{repetir}}
\renewcommand{\algorithmicuntil}{\textbf{hasta que}}
\renewcommand{\algorithmicprint}{\textbf{imprimir}} 
\renewcommand{\algorithmicreturn}{\textbf{devolver}} 
\renewcommand{\algorithmictrue}{\textbf{verdadero }} 
\renewcommand{\algorithmicfalse}{\textbf{falso }} 


\title{{\bf Algoritmo de resoluci�n de laberintos y circuitos sobre terreno llano}\\Implementaci�n sobre robot \textit{Pololu 3$\pi$}}
\author{Erik Regla\\ eregla09@alumnos.utalca.cl}
\date{1 de Mayo del 2014}

\begin{document}
\renewcommand{\figurename}{Figura~}
\renewcommand{\tablename}{Tabla~}

\maketitle

\section{Introducci�n}
\textit{Pololu 3$\pi$} es un robot multipro�sito de peque�a escala, dise�ado principalmente para detecci�n de l�neas sobre superfices homog�neas, utilizando sensores infrarrojos para medir la reflectancia sobre esta. En el siguiente informe, se detallar� la implementaci�n de un controlador \textit{PID} para la resoluci�n del problema de seguir l�neas y un algoritmo basado en \textit{Djikstra} para solucionar el problema del laberinto.



\section{Terreno}
El terreno a utilizar para pruebas es una superficie de melamina opaca de tono blanco, com�nmente utilizada en los muebles de cocina. Presenta una alta reflectancia ante luz natural y leve difusi�n de la misma. Bajo situaciones de luz natural o luz artifical hal�gena, se perciben diferencias al calibrar los sensores a $500~700us$, a luz artificial com�n es necesario subir el tiempo a $1000~1200us$.

Para las l�neas se utilizar� una cinta negra de \textit{PVC} de color negro. La reflectancia es muy baja, la que por motivos de calibraci�n se considerar� nula.


\section{Implementaci�n}

\subsection{Algoritmo de calibraci�n}
Para asegurarse de obtener una lectura apropiada de la l�nea a lo largo del circuito se consideran los siguientes factores:
\begin{itemize}
\item En el ambiente de la competencia, la luz puede variar durante el transcurso de esta.
\item Hay m�ltiples fuentes de luz en el escenario, lo cual puede afectar las lecturas si estas se realizan de manera fija o si se precalcula.
\item El robot no cuenta con alg�n sistema de barrera contra luz par�sita.
\item Solo se puede configurar el emisor IR al momento de cargar el programa, una vez activo no se puede volver a apagar.
\end{itemize}

Por ende, una medida tomada en un momento $t_1$ del d�a puede no ser v�lida para un momento $t_2$, las medidas podr�an ser muy altas o muy bajas. El Algoritmo \ref{alg:calibraci�n} implementa una calibraci�n realizada en linea.

\begin{algorithm}
\begin{algorithmic}[1]
\REQUIRE $S_{max}$ un arreglo de tama�o $n$, inicializado en $0$. $S_{min}$  un arreglo de tama�o $n$, inicializado en $\infty$. $t_{total}$ es el tiempo destinado a la calibraci�n.
\ENSURE $S_{max}$ y $S_{min}$ contienen los valores m�ximos y respectivos de la calibraci�n.
%%%%%%%%%%%%%%%%%%%%%%%%%%%%%%%BEGIN ALGORITHM%%%%%%%%%%%%%%%%%%%%%%%%%%%%%%%%
$t_{actual} \leftarrow 0$
\WHILE{$t_{actual} < \frac{t_{total}}{2}$}
	\STATE $D \leftarrow$ valor de los sensores en el instante $t_{actual}$
	\FOR{$0..n \rightarrow i$}
		\IF{$D_{i} > S_{{max}_{i}}$}
			\STATE $S_{{max}_{i}} \leftarrow D_{i}$
		\ENDIF
		\IF{$D_{i} < S_{{min}_{i}}$}
			\STATE $S_{{min}_{i}} \leftarrow D_{i}$
		\ENDIF
	\ENDFOR
	\STATE $$
\ENDWHILE
%%%%%%%%%%%%%%%%%%%%%%%%%%%%%%END ALGORITHM%%%%%%%%%%%%%%%%%%%%%%%%%%%%%%%%%
\end{algorithmic}
\caption{Calibraci�n \textit{en l�nea} para lectura de valores}\label{alg:calibraci�n}
\end{algorithm}

\end{document}
