\documentclass[11pt]{utalcaDoc}
\usepackage{alltt}
\usepackage{underscore}
\usepackage[latin1]{inputenc}
\usepackage[activeacute,spanish]{babel}
\usepackage{verbatim}
\usepackage[pdftex]{graphicx}
\usepackage{epstopdf}
\usepackage{ae}
\usepackage{enumerate}
\usepackage{amsmath}
\usepackage{amsfonts}
\usepackage{algorithm}
\usepackage{algorithmic}
\usepackage[stable]{footmisc}
\floatname{algorithm}{Algoritmo}
\renewcommand{\listalgorithmname}{Lista de algoritmos}
\renewcommand{\algorithmicrequire}{\textbf{Precondici\'{o}n:}}
\renewcommand{\algorithmicensure}{\textbf{Postcondici\'{o}n:}}
\renewcommand{\algorithmicend}{\textbf{fin}}
\renewcommand{\algorithmicif}{\textbf{si}}
\renewcommand{\algorithmicthen}{\textbf{entonces}}
\renewcommand{\algorithmicelse}{\textbf{si no}}
\renewcommand{\algorithmicelsif}{\algorithmicelse,\ \algorithmicif}
\renewcommand{\algorithmicendif}{\algorithmicend\ \algorithmicif}
\renewcommand{\algorithmicfor}{\textbf{para}}
\renewcommand{\algorithmicforall}{\textbf{para todo}}
\renewcommand{\algorithmicdo}{\textbf{hacer}}
\renewcommand{\algorithmicendfor}{\algorithmicend\ \algorithmicfor}
\renewcommand{\algorithmicwhile}{\textbf{mientras}}
\renewcommand{\algorithmicendwhile}{\algorithmicend\ \algorithmicwhile}
\renewcommand{\algorithmicloop}{\textbf{repetir}}
\renewcommand{\algorithmicendloop}{\algorithmicend\ \algorithmicloop}
\renewcommand{\algorithmicrepeat}{\textbf{repetir}}
\renewcommand{\algorithmicuntil}{\textbf{hasta que}}
\renewcommand{\algorithmicprint}{\textbf{imprimir}} 
\renewcommand{\algorithmicreturn}{\textbf{devolver}} 
\renewcommand{\algorithmictrue}{\textbf{verdadero }} 
\renewcommand{\algorithmicfalse}{\textbf{falso }} 


\title{{\bf Sistemas Operativos}\\Proyecto 1\footnote{El c�digo entregado utiliza la implementaci�n de \textit{pthreads} conforme a la norma ISO/IEC 9945-1:1996 (POSIX.1) y la implementaci�n de sem�foros especificada en POSIX Realtime Extension (1003.1b-1993/1003.1i-1995).
}}
\author{Erik Regla\\ \texttt{eregla09@alumnos.utalca.cl}\\}
\date{31 de Agosto del 2015}

\begin{document}
\renewcommand{\figurename}{Figura~}
\renewcommand{\tablename}{Tabla~}

\maketitle

\section{Dise�o del algoritmo}
El programa implementa literalmente lo que el enunciado pide (no fue necesario adaptar nada, dado que el mismo enunciado presenta un dise�o para la soluci�n del problema).

Para poder sincronizar el trabajo de las abejas se utiliza un sem�foro com�n para las abejas (en vista que no se solicita respetar el orden en que estas llegan) y otro para el oso. 

El oso espera a ser notificado de la presencia de miel en el panal y devora todo el contenido disponible. Luego notifica que ha terminado a la abeja que le indic� la situaci�n.

Por parte de las abejas, estas esperan a que su bloqueo mutuo sea liberado para poder depositar miel en el panal sin competir por este, sin embargo al momento de liberarse este. pueden ocurrir dos casos:
\begin{itemize}
\item \textbf{La abeja ha depositado miel en el panal pero no es suficiente como para que el oso coma}, en este caso, el bloqueo de la abeja es liberado para permitir a otra depositar miel.
\item \textbf{La abeja ha depositado miel en el panal y esta es suficiente como para que el oso coma}, en la cual el bloqueo del oso es liberado (para que pueda comer) y la abeja espera a que su bloqueo sea liberado por parte del oso. Una vez que el oso termina de comer, le indica a la abeja que esta puede seguir su camino tranquila y con esto, el hilo de las abejas vuelve a funcionar normalmente.
\end{itemize}

Para efectos de pruebas, primero son creados todos los sem�foros e hilos y solo una vez que estos est�n correctamente inicializados el bloqueo para las abejas es liberado. En caso de existir un error en alguna operaci�n de sincronizaci�n, el programa termina inmediatamente.

\end{document}