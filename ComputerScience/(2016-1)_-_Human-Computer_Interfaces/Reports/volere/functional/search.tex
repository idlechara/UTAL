\section{B�squeda}
\begin{requirement}{functional}{PUC\_03}
	\reqdescription{Las busquedas entregan resultados dependiendo del grado de acceso 
	que tenga el usuario.}
	\reqrationale{Las credenciales de acceso son utilizadas para diferenciar entre los
	diferentes tipos de usuarios existentes. Como no se desea mostrar toda la informaci�n a
	los usuario, algunos proyectos solo son visibles por profesores, otros por alumnos y 
	profesores y finalmente otros que pueden ser p�blicos, disponibles para todo el mundo.}
	\reqsource{Rodrigo Bustamante}
	\reqfitcriterion{Al ejecutarse una b�squeda, se agregan los resultados que contegan 
	las palabras clave ya sea en alguna de sus categor�as, etiquetas, descripci�n o t�tulo.}
	\reqsatisfaction{5}{3}
	\reqdepencies{\reffr{RF:U:PRIVILEGIO}}{}
	\reqdocuments{Entrevista 1}
	\reqhistory{}
	\label{RF:S:PRIVILEGES}
\end{requirement}

\begin{requirement}{functional}{PUC\_03}
	\reqdescription{Un proyecto puede ser identificado en base a sus palabras clave.}
	\reqrationale{Debe de ser posible realizar una b�squeda sobre el texto de el 
	t�tulo y la descripci�n, texto de etiquetas, etc.}
	\reqsource{Rodrigo Bustamante}
	\reqfitcriterion{Al ejecutarse una b�squeda, se agregan los resultados que contegan 
	las palabras clave ya sea en alguna de sus categor�as, etiquetas, descripci�n o t�tulo.}
	\reqsatisfaction{3}{3}
	\reqdepencies{\reffr{RF:P:PRIVILEGES}, \reffr{RF:P:TITULO}, \reffr{RF:P:CATEGORIAS},\reffr{RF:P:ETIQUETAS}}{}
	\reqdocuments{Entrevista 1}
	\reqhistory{}
	\label{RF:S:PALABRA_CLAVE}
\end{requirement}

\begin{requirement}{functional}{PUC\_03}
	\reqdescription{Un proyecto puede ser identificado en base a sus categor�as.}
	\reqrationale{Las b�squedas se deben de poder realizar utilizando como base las
	categor�as del proyecto.}
	\reqsource{Rodrigo Bustamante}
	\reqfitcriterion{Al ejecutarse una b�squeda se agregan a los resultados los
	proyectos cuyas categor�as coincidan con los de la b�squeda}
	\reqsatisfaction{3}{3}
	\reqdepencies{\reffr{RF:P:PRIVILEGES},\reffr{RF:P:CATEGORIAS}}{}
	\reqdocuments{Entrevista 1}
	\reqhistory{}
	\label{RF:S:ETIQUETAS}
\end{requirement}


\begin{requirement}{functional}{PUC\_03}
	\reqdescription{Un proyecto puede ser identificado en base a sus etiquetas.}
	\reqrationale{Las b�squedas se deben de poder realizar utilizando como base las
	etiquetas del proyecto.}
	\reqsource{Rodrigo Bustamante}
	\reqfitcriterion{Al ejecutarse una b�squeda se agregan a los resultados los
	proyectos cuyas etiquetas coincidan con los de la b�squeda}
	\reqsatisfaction{3}{3}
	\reqdepencies{\reffr{RF:P:PRIVILEGES},\reffr{RF:P:ETIQUETAS}}{}
	\reqdocuments{Entrevista 1}
	\reqhistory{}
	\label{RF:S:ETIQUETAS}
\end{requirement}


\begin{requirement}{functional}{PUC\_03}
	\reqdescription{Los administradores pueden usar como criterio de b�squeda campos eliminados (deshabilitados)}
	\reqrationale{No todos los proyectos deben ser visibles a todos los usuarios, sin emabrgo, los administradores
	necesitan estar en control del estado del producto.}
	\reqsource{Rodrigo Bustamante}
	\reqfitcriterion{El administrador puede ver una lista de proyectos, categor�as, etiquetas eliminadas 
	junto con una opci�n en el panel de b�squeda para ellas..}
	\reqsatisfaction{2}{3}
	\reqdepencies{\reffr{RF:P:PRIVILEGES},\reffr{RF:U:PRIVILEGIO}}{}
	\reqdocuments{Entrevista 2}
	\reqhistory{}
	\label{RF:S:ELIMINADOS}
\end{requirement}
