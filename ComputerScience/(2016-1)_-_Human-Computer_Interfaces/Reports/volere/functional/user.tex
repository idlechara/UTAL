\section{Usuarios}
\subsection{Creaci�n de cuentas de usuario}

\begin{requirement}{1}{}
	\reqdescription{Un usuario puede solicitar el registro de una cuenta.}
	\reqrationale{Algunos usuarios pueden querer acceso a proyectos privados o bien,
		subir proyectos a la plataforma, para lo cual requieren de una cuenta registrada.}
	\reqsource{Rodrigo Bustamante}
	\reqfitcriterion{El administrador recibe solicitudes de registro elevadas por usuarios.}
	\reqsatisfaction{3}{3}
	\reqdepencies{\reffr{RF:U:FORMULARIO_CREACION_SOLICITUD}, \reffr{RF:U:FORMULARIO_EMAIL_UNICO}, \reffr{RF:U:FORMULARIO_CREACION_CONTENIDO}}{}
	\reqdocuments{Entrevista 1, Entrevista 6}
	\reqhistory{}
	\label{RF:U:ELEVAR_SOLICITUD}
\end{requirement}


\begin{requirement}{functional}{PUC\_01}
	\reqdescription{El formulario de solicitud de cuenta muestra los campos 
		\texttt{nombre completo}, \texttt{direcci�n de correo electr�nico} y \texttt{contrase�a}.
		De manera opcional \texttt{afiliaci�n} y \texttt{n�mero de matr�cula}.}
		\reqrationale{Es necesario tener la informaci�n de contacto de cada uno de los usuarios. 
		Adem�s, la informaci�n de afiliaci�n sirve para determinar el nivel de acceso de un usuario. 
		Por ejemplo, los correos pertenecientes al dominio \texttt{alumnos.utalca.cl} solo pertenecen
		a alumnos, y las pertenecientes al dominio \texttt{utalca.cl} solo pertenecen a funcionarios. En caso
		de no tener un correo institucional se asume
		que este es un usuario externo.}
	\reqsource{Rodrigo Bustamante}
	\reqfitcriterion{El formulario de registro solicita los campos \texttt{nombre completo},
		\texttt{direcci�n de correo electr�nico}, \texttt{contrase�a}, \texttt{afiliaci�n} y
		\texttt{n�mero de matr�cula}.}
	\reqsatisfaction{3}{4}
	\reqdepencies{}{}
	\reqdocuments{Entrevista 1,2, 5}
	\reqhistory{Entrevista 6: Formulario solicita informaci�n de afiliaci�n.}
	\label{RF:U:FORMULARIO_CREACION_CONTENIDO}
\end{requirement}

\begin{requirement}{functional}{PUC\_01}
	\reqdescription{El usuario para solicitar una cuenta debe enviar a trav�s de formulario
		de creaci�n de cuentas su \texttt{nombre completo}, \texttt{direcci�n de correo electr�nico}
		y \texttt{contrase�a} y. De manera opcional \texttt{afiliaci�n} y \texttt{n�mero de matr�cula}.}
		\reqrationale{Es necesario tener la informaci�n de contacto de cada uno de los usuarios. Adem�s,
		la informaci�n de afiliaci�n sirve para determinar el nivel de acceso de un usuario.
		Por ejemplo, los correos pertenecientes al dominio \texttt{alumnos.utalca.cl} solo pertenecen
		a alumnos, y las pertenecientes al dominio \texttt{solo pertenecen a funcionarios}. En caso
		de no tener un correo institucional se asume
		que este es un usuario externo.}
	\reqsource{Rodrigo Bustamante}
	\reqfitcriterion{El formulario de registro solicita los campos \texttt{nombre completo},
		\texttt{direcci�n de correo electr�nico}, \texttt{contrase�a}, \texttt{afiliaci�n} y
		\texttt{n�mero de matr�cula}.}
	\reqsatisfaction{2}{3}
	\reqdepencies{\reffr{RF:U:FORMULARIO_CREACION_CONTENIDO}}{}
	\reqdocuments{Entrevista 1,2, 5}
	\reqhistory{Entrevista 6: Formulario solicita informaci�n de afiliaci�n.}
	\label{RF:U:FORMULARIO_CREACION_SOLICITUD}
\end{requirement}

\begin{requirement}{functional}{PUC\_01}
	\reqdescription{El campo \texttt{direcci�n de correo electr�nico} enviado por el usuario a 
		trav�s del formulario de creaci�n debe ser �nico y no estar registrado previamente.}
	\reqrationale{Es necesario tener la informaci�n de contacto de cada uno de los usuarios.
		Para esto, es necesario que la informaci�n de 
	los usuarios permita diferenciar entre estos.}
	\reqsource{Rodrigo Bustamante}
	\reqfitcriterion{El producto verifica que el correo ingresado por el usuario sea �nico. De
		serlo, ingresa 	la solicitud, caso contrario, notifica al usuario.}
	\reqsatisfaction{2}{3}
	\reqdepencies{\reffr{RF:U:FORMULARIO_CREACION_SOLICITUD}}{}
	\reqdocuments{Entrevista 5}
	\reqhistory{}
	\label{RF:U:FORMULARIO_EMAIL_UNICO}
\end{requirement}


\begin{requirement}{functional}{PUC\_13}
	\reqdescription{Un administrador puede aceptar o rechazar una solicitud de registro.}
	\reqrationale{El administrador del sistema es el encargado de aceptar, rechazar y verificar
		la informaci�n provista por el usuario para la creaci�n de una cuenta.}
	\reqsource{Rodrigo Bustamante}
	\reqfitcriterion{El administrador tiene opciones disponibles para aceptar o rechazar
		solicitudes de registro.}
	\reqsatisfaction{3}{2}
	\reqdepencies{\reffr{RF:U:ELEVAR_SOLICITUD}, \reffr{RF:M:NOTIF_NUEVA_CUENTA}}{}
	\reqdocuments{Entrevista 1}
	\reqhistory{}
	\label{RF:U:ADMIN_SOLICITUD}
\end{requirement}

\begin{requirement}{functional}{}
	\reqdescription{Los administradores pueden crear cuentas de usuario con privilegios de moderaci�n.}
	\reqrationale{Los moderadores son encargados de velar por la correcta conducta en los comentarios y
	 contenido. Por tanto, ellos deben tener experiencia en los temas que moderan. Para ellos se les crea
	 credenciales de acceso separadas de las de un usuario normal las cuales tienen limites como por 
	 ejemplo, subir proyectos.}
	\reqsource{Rodrigo Bustamante}
	\reqfitcriterion{El administrador tiene opciones disponibles para crear nuevas credenciales de moderador}
	\reqsatisfaction{3}{2}
	\reqdepencies{}{}
	\reqdocuments{Entrevista 6}
	\reqhistory{}
	\label{RF:U:MODERADOR_CREATE}
\end{requirement}
\subsection{Borrado y deshabilitaci�n de usuarios}
\begin{requirement}{functional}{}
	\reqdescription{Administrador puede deshabilitar usuarios}
	\reqrationale{Por motivos de conducta o bien de organizaci�n y desuso, los administradores
	pueden limitar el uso de una cuenta independiente de su tipo,
	deshabilitandola de ser necesario.}
	\reqsource{Rodrigo Bustamante}
	\reqfitcriterion{Administradores tienen una opci�n para deshabilitar usuarios.}
	\reqsatisfaction{1}{1}
	\reqdepencies{\reffr{RF:U:ELEVAR_SOLICITUD}}{}
	\reqdocuments{Entrevista 5}
	\reqhistory{}
	\label{RF:U:ADMIN_DEL_USER}
\end{requirement}
\subsection{Login}
\begin{requirement}{functional}{}
	\reqdescription{Usuario puede ingresar e iniciar sesi�n en el sistema 
		utilizando sus credenciales de acceso}
	\reqrationale{Se busca que los usuarios registrados puedan tener acceso a proyectos no
		disponibles de manera p�blica como tambien el registro de proyectos. En el caso de 
		administradores y moderadores, estos pueden ingresar al sistema para poder
		cumplir con sus funciones.}
	\reqsource{Rodrigo Bustamante}
	\reqfitcriterion{Usuarios con credenciales pueden realizar ``login'' en la plataforma.}
	\reqsatisfaction{2}{2}
	\reqdepencies{\reffr{RF:U:ELEVAR_SOLICITUD}}{}
	\reqdocuments{Entrevista 1,2, 5}
	\reqhistory{Entrevista 6: Formulario solicita informaci�n de afiliaci�n.}
	\label{RF:U:LOGIN}
\end{requirement}

\begin{requirement}{functional}{}
	\reqdescription{Producto notifica al usuario cuando sus credenciales son incorrectas.}
	\reqrationale{Si no permite el acceso al usuario con sus credenciales, debe mostrar un mensaje
	que diga que son incorrectas sin ense�ar mayor detalle del porqu�}
	\reqsource{Rodrigo Bustamante}
	\reqfitcriterion{Cuando un usuario ingresa con credenciales incorrectas muestra un mensaje de alerta.}
	\reqsatisfaction{2}{2}
	\reqdepencies{\reffr{RF:U:LOGIN}}{}
	\reqdocuments{Entrevista 1}
	\reqhistory{}
	\label{RF:U:LOGIN_WARNING}
\end{requirement}
\begin{requirement}{functional}{PUC\_03}
	\reqdescription{Usuarios tienen nivel de acceso}
	\reqrationale{No todos los proyectos deben ser visibles a todos los usuarios. Y tambi�n
	 no todos los usuarios tienen las mismas acciones disponibles, por tanto es necesario 
	 controlar los accesos a cada uno de los componentes del producto.}
	\reqsource{Rodrigo Bustamante}
	\reqfitcriterion{Las operaciones sobre proyectos solo pueden ser realizadas cuando el
	nivel de acceso del usuario es compatible con el del proyecto}
	\reqsatisfaction{5}{5}
	\reqdepencies{}{}
	\reqdocuments{Entrevista 1}
%	\reqhistory{w}
	\label{RF:U:PRIVILEGIO}
\end{requirement}

\subsection{Administraci�n}
\begin{requirement}{functional}{PUC\_19}
	\reqdescription{Los administradores pueden cambiar las credenciales de acceso de cualquier usuario}
	\reqrationale{En caso de realizar mantenimiento, o bien en caso de p�rdida de contrase�a, como 
	el administrador es quien toma la acci�n correctiva, es necesario que pueda editar los datos
	de los dem�s usuarios.}
	\reqsource{Rodrigo Bustamante}
	\reqfitcriterion{Un administrador puede cambiar las credenciales de acceso de cualquier usuario por 
	medio de la plataforma}
	\reqsatisfaction{2}{1}
	\reqdepencies{\reffr{RF:U:LOGIN}}{}
	\reqdocuments{Entrevista 1, Entrevista 2}
	\reqhistory{}
	\label{RF:P:EDICION_CREDENCIAL}
\end{requirement}


\begin{requirement}{functional}{}
	\reqdescription{Un usuario registrado puede cambiar su propia contrase�a}
	\reqrationale{Requerido por seguridad propia de la cuenta.}
	\reqsource{Rodrigo Bustamante}
	\reqfitcriterion{Los usuarios pueden cambiar sus propias credenciales de acceso.}
	\reqsatisfaction{2}{1}
	\reqdepencies{\reffr{RF:U:LOGIN}}{}
	\reqdocuments{Entrevista 1, Entrevista 2}
	\reqhistory{}
	\label{RF:P:EDICION_CREDENCIAL_PROPIA}
\end{requirement}





% \begin{requirement}{functional}{}
% 	\reqdescription{El campo \texttt{direcci�n de correo electr�nico} enviado por el usuario a trav�s del formulario de creaci�n debe ser �nico y no estar registrado previamente.}
% 	\reqrationale{Es necesario tener la informaci�n de contacto de cada uno de los usuarios. Para esto, es necesario que la informaci�n de 
% 	los usuarios permita diferenciar entre estos.}
% 	\reqsource{Rodrigo Bustamante}
% 	\reqfitcriterion{El producto verifica que el correo ingresado por el usuario sea �nico. De serlo, ingresa la solicitud, caso contrario, notifica al usuario.}
% 	\reqsatisfaction{2}{3}
% 	\reqdepencies{\reffr{RF:U:FORMULARIO_CREACION_SOLICITUD}}{}
% 	\reqdocuments{Entrevista 5}
% 	\reqhistory{}
% 	\label{RF:U:FORMULARIO_EMAIL_UNICO}
% \end{requirement}


% \begin{requirement}{functional}{BUC02}
% 	\reqdescription{Usuario puede buscar proyectos en la plataforma de acuerdo a los niveles de acceso del proyecto}
% 	\reqrationale{El objetivo de almacenar proyectos es su posterior b�squeda, sin embargo, es necesario poder filtrar los resultados de acuerdo a los privilegios de acceso de este para proyectos p�blicos y privados}
% 	\reqsource{Rodrigo Bustamante}
% 	\reqfitcriterion{Las b�squedas realizadas entregan proyectos correspondientes al nivel de privilegio del usuario}
% 	\reqsatisfaction{5}{5}
% 	\reqdepencies{\reffr{RF:PROYECTOS_PRIVILEGIOS}}{}
% 	\reqdocuments{Entrevista 1,2, 5}
% %	\reqhistory{w}
% 	\label{RF:BUSQUEDA_PRIVILEGIOS}
% \end{requirement}

% \begin{requirement}{functional}{BUC02}
% 	\reqdescription{Usuario puede realizar comentarios a un proyecto}
% 	\reqrationale{Los usuarios pueden utilizar la plataforma para complementar la informaci�n existente del proyecto o bien para realizar consultas. Estos comentarios son mediados por un moderador.}
% 	\reqsource{Rodrigo Bustamante}
% 	\reqfitcriterion{Los proyectos muestran un apartado para realizar comentarios}
% 	\reqsatisfaction{5}{5}
% 	\reqdepencies{Ninguna}{}
% 	\reqdocuments{Entrevista 5}
% %	\reqhistory{w}
% 	\label{RF:PROYECTO_CREAR_COMENTARIO}
% \end{requirement}

% \begin{requirement}{functional}{BUC02}
% 	\reqdescription{Moderador debe ser notificado cuando un comentario nuevo se realiza.}
% 	\reqrationale{Los moderadores est�n encargados de aceptar o no comentarios a proyectos realizados a travez de la plataforma, por tanto deben estar actualizados respecto al estado de estos.}
% 	\reqsource{Rodrigo Bustamante}
% 	\reqfitcriterion{Cuando un comentario es creado, los moderadores pertinentes son notificados}
% 	\reqsatisfaction{5}{5}
% 	\reqdepencies{\reffr{RF:PROYECTOS_CREAR_COMENTARIO}}{}
% 	\reqdocuments{Entrevista 5}
% %	\reqhistory{w}
% 	\label{RF:MODERADOR_NOTIFICIACION_NUEVO_COMENTARIO}
% \end{requirement}


% \begin{requirement}{functional}{BUC02}
% 	\reqdescription{Los comentarios deben ser almacenados junto a su proyecto correspondiente.}
% 	\reqrationale{Existe una relaci�n 1:N entre proyectos y comentarios, por tanto la informaci�n de los comentarios solo est� relacionada a estos.}
% 	\reqsource{I.O.R.S.H. Dev Team}
% 	\reqfitcriterion{Los comentarios de un proyecto solo pertenecen a ellos y no son visibles desde otros.}
% 	\reqsatisfaction{5}{5}
% 	\reqdepencies{\reffr{RF:PROYECTO_CREAR_COMENTARIO}}{}
% 	\reqdocuments{Entrevista 5}
% %	\reqhistory{w}
% 	\label{RF:COMENTARIO_RELACION_PROYECTO}
% \end{requirement}