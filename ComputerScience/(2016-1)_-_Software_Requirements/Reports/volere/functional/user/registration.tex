\subsection{Creaci�n de cuentas de usuario}

\begin{requirement}{functional}{PUC\_01}
	\reqdescription{Un usuario puede solicitar el registro de una cuenta.}
	\reqrationale{Algunos usuarios pueden querer acceso a proyectos privados o bien,
		subir proyectos a la plataforma, para lo cual requieren de una cuenta registrada.}
	\reqsource{Rodrigo Bustamante}
	\reqfitcriterion{El administrador recibe solicitudes de registro elevadas por usuarios.}
	\reqsatisfaction{3}{3}
	\reqdepencies{\reffr{RF:U:FORMULARIO_CREACION_SOLICITUD}, \reffr{RF:U:FORMULARIO_EMAIL_UNICO}, \reffr{RF:U:FORMULARIO_CREACION_CONTENIDO}}{}
	\reqdocuments{Entrevista 1, Entrevista 6}
	\reqhistory{}
	\label{RF:U:ELEVAR_SOLICITUD}
\end{requirement}


\begin{requirement}{functional}{PUC\_01}
	\reqdescription{El formulario de solicitud de cuenta muestra los campos 
		\texttt{nombre completo}, \texttt{direcci�n de correo electr�nico} y \texttt{contrase�a}.
		De manera opcional \texttt{afiliaci�n} y \texttt{n�mero de matr�cula}.}
		\reqrationale{Es necesario tener la informaci�n de contacto de cada uno de los usuarios. 
		Adem�s, la informaci�n de afiliaci�n sirve para determinar el nivel de acceso de un usuario. 
		Por ejemplo, los correos pertenecientes al dominio \texttt{alumnos.utalca.cl} solo pertenecen
		a alumnos, y las pertenecientes al dominio \texttt{utalca.cl} solo pertenecen a funcionarios. En caso
		de no tener un correo institucional se asume
		que este es un usuario externo.}
	\reqsource{Rodrigo Bustamante}
	\reqfitcriterion{El formulario de registro solicita los campos \texttt{nombre completo},
		\texttt{direcci�n de correo electr�nico}, \texttt{contrase�a}, \texttt{afiliaci�n} y
		\texttt{n�mero de matr�cula}.}
	\reqsatisfaction{3}{4}
	\reqdepencies{}{}
	\reqdocuments{Entrevista 1,2, 5}
	\reqhistory{Entrevista 6: Formulario solicita informaci�n de afiliaci�n.}
	\label{RF:U:FORMULARIO_CREACION_CONTENIDO}
\end{requirement}

\begin{requirement}{functional}{PUC\_01}
	\reqdescription{El usuario para solicitar una cuenta debe enviar a trav�s de formulario
		de creaci�n de cuentas su \texttt{nombre completo}, \texttt{direcci�n de correo electr�nico}
		y \texttt{contrase�a} y. De manera opcional \texttt{afiliaci�n} y \texttt{n�mero de matr�cula}.}
		\reqrationale{Es necesario tener la informaci�n de contacto de cada uno de los usuarios. Adem�s,
		la informaci�n de afiliaci�n sirve para determinar el nivel de acceso de un usuario.
		Por ejemplo, los correos pertenecientes al dominio \texttt{alumnos.utalca.cl} solo pertenecen
		a alumnos, y las pertenecientes al dominio \texttt{solo pertenecen a funcionarios}. En caso
		de no tener un correo institucional se asume
		que este es un usuario externo.}
	\reqsource{Rodrigo Bustamante}
	\reqfitcriterion{El formulario de registro solicita los campos \texttt{nombre completo},
		\texttt{direcci�n de correo electr�nico}, \texttt{contrase�a}, \texttt{afiliaci�n} y
		\texttt{n�mero de matr�cula}.}
	\reqsatisfaction{2}{3}
	\reqdepencies{\reffr{RF:U:FORMULARIO_CREACION_CONTENIDO}}{}
	\reqdocuments{Entrevista 1,2, 5}
	\reqhistory{Entrevista 6: Formulario solicita informaci�n de afiliaci�n.}
	\label{RF:U:FORMULARIO_CREACION_SOLICITUD}
\end{requirement}

\begin{requirement}{functional}{PUC\_01}
	\reqdescription{El campo \texttt{direcci�n de correo electr�nico} enviado por el usuario a 
		trav�s del formulario de creaci�n debe ser �nico y no estar registrado previamente.}
	\reqrationale{Es necesario tener la informaci�n de contacto de cada uno de los usuarios.
		Para esto, es necesario que la informaci�n de 
	los usuarios permita diferenciar entre estos.}
	\reqsource{Rodrigo Bustamante}
	\reqfitcriterion{El producto verifica que el correo ingresado por el usuario sea �nico. De
		serlo, ingresa 	la solicitud, caso contrario, notifica al usuario.}
	\reqsatisfaction{2}{3}
	\reqdepencies{\reffr{RF:U:FORMULARIO_CREACION_SOLICITUD}}{}
	\reqdocuments{Entrevista 5}
	\reqhistory{}
	\label{RF:U:FORMULARIO_EMAIL_UNICO}
\end{requirement}


\begin{requirement}{functional}{PUC\_13}
	\reqdescription{Un administrador puede aceptar o rechazar una solicitud de registro.}
	\reqrationale{El administrador del sistema es el encargado de aceptar, rechazar y verificar
		la informaci�n provista por el usuario para la creaci�n de una cuenta.}
	\reqsource{Rodrigo Bustamante}
	\reqfitcriterion{El administrador tiene opciones disponibles para aceptar o rechazar
		solicitudes de registro.}
	\reqsatisfaction{3}{2}
	\reqdepencies{\reffr{RF:U:ELEVAR_SOLICITUD}, \reffr{RF:M:NOTIF_NUEVA_CUENTA}}{}
	\reqdocuments{Entrevista 1}
	\reqhistory{}
	\label{RF:U:ADMIN_SOLICITUD}
\end{requirement}

\begin{requirement}{functional}{}
	\reqdescription{Los administradores pueden crear cuentas de usuario con privilegios de moderaci�n.}
	\reqrationale{Los moderadores son encargados de velar por la correcta conducta en los comentarios y
	 contenido. Por tanto, ellos deben tener experiencia en los temas que moderan. Para ellos se les crea
	 credenciales de acceso separadas de las de un usuario normal las cuales tienen limites como por 
	 ejemplo, subir proyectos.}
	\reqsource{Rodrigo Bustamante}
	\reqfitcriterion{El administrador tiene opciones disponibles para crear nuevas credenciales de moderador}
	\reqsatisfaction{3}{2}
	\reqdepencies{}{}
	\reqdocuments{Entrevista 6}
	\reqhistory{}
	\label{RF:U:MODERADOR_CREATE}
\end{requirement}