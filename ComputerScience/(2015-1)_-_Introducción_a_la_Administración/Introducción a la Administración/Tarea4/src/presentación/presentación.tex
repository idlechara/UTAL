\documentclass{beamer}
% Class options include: notes, handout, trans
%                        

% Theme for beamer presentation 
% Other themes include: beamerthemebars, beamerthemelined,beamerthemetree, beamerthemeplain
\usetheme{Copenhagen}
\usepackage[latin1]{inputenc}
\usepackage[T1]{fontenc}
\def\spanishoptions{chile}
\usepackage[spanish]{babel}
\selectlanguage{spanish}
\usepackage{amssymb}
\usepackage{soul}
\usepackage{graphicx}
\usepackage{listings}
\definecolor{listinggray}{gray}{0.9}
\definecolor{lbcolor}{rgb}{0.9,0.9,0.9}
\lstset{
backgroundcolor=\color{lbcolor},
    tabsize=4,    
%   rulecolor=,
    language=[GNU]C++,
        basicstyle=\scriptsize,
        upquote=true,
        aboveskip={1.5\baselineskip},
        columns=fixed,
        showstringspaces=false,
        extendedchars=false,
        breaklines=true,
        prebreak = \raisebox{0ex}[0ex][0ex]{\ensuremath{\hookleftarrow}},
        frame=single,
        numbers=left,
        showtabs=false,
        showspaces=false,
        showstringspaces=false,
        identifierstyle=\ttfamily,
        keywordstyle=\color[rgb]{0,0,1},
        commentstyle=\color[rgb]{0.026,0.112,0.095},
        stringstyle=\color[rgb]{0.627,0.126,0.941},
        numberstyle=\color[rgb]{0.205, 0.142, 0.73},
%        \lstdefinestyle{C++}{language=C++,style=numbers}?.
}
\lstset{
    backgroundcolor=\color{lbcolor},
    tabsize=4,
  language=C++,
  captionpos=b,
  tabsize=3,
  frame=lines,
  numbers=left,
  numberstyle=\tiny,
  numbersep=5pt,
  breaklines=true,
  showstringspaces=false,
  basicstyle=\footnotesize,
%  identifierstyle=\color{magenta},
  keywordstyle=\color[rgb]{0,0,1},
  commentstyle=\color[rgb]{0.026,0.112,0.095},
  stringstyle=\color{red}
  }

\usefonttheme{professionalfonts}


\title[Introducci�n a la administraci�n]{Introducci�n a la administraci�n}
\subtitle{Tarea 4 - An�lisis de caso de estudio}    % Enter your title between curly braces
\author[E. Regla]{Erik Regla}                 % Enter your name between curly braces
\institute[UTalca]{Universidad de Talca}      % Enter your institute name between curly braces
\date{\today}      % Enter the date or \today between curly braces

\begin{document}

% Creates title page of slide show using above information
\begin{frame}
  \titlepage
\end{frame}
\note{Talk for 30 minutes} % Add notes to yourself that will be displayed when typeset with the notes class option.

%% Creates table of contents slide incorporating all \section and \subsection commands.
%\begin{frame}
%  \tableofcontents
%\end{frame}

\section{Descripci�n}
\begin{frame}
\frametitle{Situaci�n a analizar}
Arnold Bosch est� a cargo del departamento de ``Control de entradas'' en una empresa de textiles alemana la cual actualmente est� al borde de una crisis debido a problemas en el suministro. Estudios llevados a cabo en la empresa est�n actualmente apuntando sus dedos al departamento en el que A. Bosch est� a cargo.
\end{frame}


\section{Principales falencias}
\begin{frame}
\frametitle{Situaci�n a analizar}
\begin{itemize}
	\item Existencia de un l�der informal y posterior p�rdida de este.
	\item Falta de proactividad de parte de Bosch para resolver problemas y poca presencia en los procesos.
	\item Baja identificaci�n con la tarea de parte del personal
\end{itemize}
\end{frame}



\section{Repercusi�n de fallos}
\begin{frame}
\frametitle{Repercusi�n de fallos en el departamento de ``Control de entradas'' en la producci�n}
\begin{itemize}
	\item Retraso en envios
	\item Fallas mec�nicas en maquinaria producto de la baja calidad del material
	\item Desmotivaci�n del personal de producci�n (debido a la raz�n anterior)
\end{itemize}
\end{frame}


\section{Medidas a tomar}
\begin{frame}
\frametitle{En general}
\begin{itemize}
	\item Reemplazo del l�der informal
	\item Implementar pol�ticas de control
	\item Analizar el estado de los trabajadores
\end{itemize}
\end{frame}

\begin{frame}
\frametitle{Delegaci�n}
\begin{itemize}
	\item Analizar permanencia del cargo
	\item Analizar nivel de identificaci�n con la tarea
\end{itemize}
\end{frame}

\begin{frame}
\frametitle{Implementaci�n}
\begin{itemize}
	\item Solucionar problema de falta de motivaci�n
	\item Falta de mecanismos de control
\end{itemize}
\end{frame}

\begin{frame}
\frametitle{Control}
\begin{itemize}
	\item Generar espacio para presentar inquetudes
	\item Establecer un control durante el proceso
	\item Evaluaciones de desempe�o y estado
\end{itemize}
\end{frame}
\end{document}
