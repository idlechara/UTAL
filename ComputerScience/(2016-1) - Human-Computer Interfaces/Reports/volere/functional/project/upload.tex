\subsection{Subida}
\begin{requirement}{functional}{}
	\reqdescription{Un proyecto puede ser enviado a revisi�n por parte de el administrador.}
	\reqrationale{Uno de los roles del administrador es revisar que el proyecto cumpla con las 
	normativas de est�ndar impuestas por la organizaci�n. Adem�s de verificar que el c�digo fuente
	no tenga artefactos extra�os y el paquete ejecutable pueda efectivamente ser ejecutado.}
	\reqsource{Rodrigo Bustamante}
	\reqfitcriterion{Al enviar un proyecto se verifica que est�n listos los campos requeridos, luego
	se le env�a al administrador y se le notifica a este de la acci�n realizada.}
	\reqsatisfaction{3}{4}
	\reqdepencies{\reffr{RF:P:CATEGORIAS}, \reffr{RF:P:ETIQUETAS}, \reffr{RF:P:TITULO},
	\reffr{RF:P:PROYECTO_VACIO}, \reffr{RF:P:BLOQUEO}, \reffr{RF:P:DOCUMENTACION},
	\reffr{RF:P:CODIGO_FUENTE}, \reffr{RF:P:EJECUTABLE}}{}
	\reqdocuments{Entrevista 1, Entrevista 5}
	\reqhistory{Entrevista 5: Se elimina la necesidad de incluir todos los campos en el formulario de creaci�n de proyectos antes de la revision.}
	\label{RF:P:ENVIO_REVISION}
\end{requirement}

\begin{requirement}{functional}{}
	\reqdescription{Un proyecto enviado a revision bloquea la edicion por parte de sus usuarios hasta
	que un administrador dictamine el resultado de su evaluaci�n.}
	\reqrationale{Uno de los roles del administrador es revisar que el proyecto cumpla con las 
	normativas de est�ndar impuestas por la organizaci�n. Adem�s de verificar que el c�digo fuente
	no tenga artefactos extra�os y el paquete ejecutable pueda efectivamente ser ejecutado. 
	Por tanto, editar estos campos durante el proceso puede llevar a problemas en la revisi�n del mismo.}
	\reqsource{Rodrigo Bustamante}
	\reqfitcriterion{Al enviar un proyecto se verifica que est�n listos los campos requeridos, luego
	se le env�a al administrador y se le notifica a este de la acci�n realizada.}
	\reqsatisfaction{4}{4}
	\reqdepencies{}{}
	\reqdocuments{Entrevista 1}
	\reqhistory{}
	\label{RF:P:BLOQUEO}
\end{requirement}

\begin{requirement}{functional}{}
	\reqdescription{Un usuario que tenga permisos de edicion sobre un proyecto puede editar el
	el paquete que contiene la documentaci�n del proyecto}
	\reqrationale{Un usuario que est� editando el proyecto necesita hacer cambios sobre el estado
	de este ya sea por errores al subir el paquete, o bien para incluir una versi�n nueva antes
	del proceso de revisi�n}
	\reqsource{Rodrigo Bustamante}
	\reqfitcriterion{}
	\reqsatisfaction{3}{1}
	\reqdepencies{\reffr{RF:P:PROYECTO_VACIO}}{}
	\reqdocuments{Entrevista 1}
	\reqhistory{}
	\label{RF:P:DOCUMENTACION}
\end{requirement}
\begin{requirement}{functional}{}
	\reqdescription{Un usuario que tenga permisos de edicion sobre un proyecto puede editar el
	el paquete que contiene el c�digo fuente del proyecto}
	\reqrationale{Un usuario que est� editando el proyecto necesita hacer cambios sobre el estado
	de este ya sea por errores al subir el paquete, o bien para incluir una versi�n nueva antes
	del proceso de revisi�n}
	\reqsource{Rodrigo Bustamante}
	\reqfitcriterion{}
	\reqsatisfaction{3}{1}
	\reqdepencies{\reffr{RF:P:PROYECTO_VACIO}}{}
	\reqdocuments{Entrevista 1}
	\reqhistory{}
	\label{RF:P:CODIGO_FUENTE}
\end{requirement}
\begin{requirement}{functional}{}
	\reqdescription{Un usuario que tenga permisos de edicion sobre un proyecto puede editar el
	el paquete que contiene el paquete de ejecuci�n del proyecto}
	\reqrationale{Un usuario que est� editando el proyecto necesita hacer cambios sobre el estado
	de este ya sea por errores al subir el paquete, o bien para incluir una versi�n nueva antes
	del proceso de revisi�n}
	\reqsource{Rodrigo Bustamante}
	\reqfitcriterion{}
	\reqsatisfaction{3}{1}
	\reqdepencies{\reffr{RF:P:PROYECTO_VACIO}}{}
	\reqdocuments{Entrevista 1}
	\reqhistory{}
	\label{RF:P:EJECUTABLE}
\end{requirement}