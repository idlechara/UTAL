\section{Visualizaci�n}
\begin{requirement}{functional}{}
	\reqdescription{Un proyecto puede lanzar una instancia temporal de ejecuci�n de estar disponible.}
	\reqrationale{Algunos c�digos fuente o paquetes ejecutables pueden ser desplegados desde dentro de la apliaci�n. En estos casos, si el administrador habilita la opci�n de ejecuci�n temporal, debe de haber
	una opci�n disponible en la vista del proyecto.}
	\reqsource{Rodrigo Bustamante}
	\reqfitcriterion{Administrador puede indicar que un proyecto puede ser ejecutado directamente desde
	la plataforma y usuarios pueden ejecutar proyectos directamente de estar habilitada la opci�n.}
	\reqsatisfaction{2}{3}
	\reqdepencies{\reffr{RF:P:REVIEW_PUBLISH}, \reffr{RF:U:PRIVILEGIO}}{}
	\reqdocuments{Entrevista 1}
	\reqhistory{}
	\label{RF:P:VIRTUAL_ENVIRONMENT}
\end{requirement}

\begin{requirement}{functional}{}
	\reqdescription{El administrador puede habilitar, deshabilitar y configurar instancias temporales de
	 ejecuci�n para un proyecto}
	\reqrationale{Algunos c�digos fuente o paquetes ejecutables pueden ser desplegados desde dentro de la apliaci�n. 
	En estos casos, si el administrador habilita la opci�n de ejecuci�n temporal, debe de haber
	una opci�n disponible en la vista del proyecto.}
	\reqsource{Rodrigo Bustamante}
	\reqfitcriterion{Administrador puede indicar que un proyecto puede ser ejecutado directamente desde
	la plataforma y usuarios pueden ejecutar proyectos directamente de estar habilitada la opci�n.}
	\reqsatisfaction{2}{3}
	\reqdepencies{\reffr{RF:P:REVIEW_PUBLISH}, \reffr{RF:U:PRIVILEGIO}}{}
	\reqdocuments{Entrevista 1}
	\reqhistory{}
	\label{RF:P:VIRTUAL_ENVIRONMENT_CONFIG}
\end{requirement}

\begin{requirement}{functional}{}
	\reqdescription{Las instancias temporales de ejecuci�n son �nicas y por defecto est�n globalmente disponibles 
	para los usuarios con los permisos adecuados.}
	\reqrationale{Algunos c�digos fuente o paquetes ejecutables pueden ser desplegados desde dentro de la apliaci�n. 
	En estos casos, si el administrador habilita la opci�n de ejecuci�n temporal, debe de haber
	una opci�n disponible en la vista del proyecto. Sin embargo, no tiene sentido ejecutar m�ltiples 
	instancias de una misma aplicaci�n, esencialmente por un tema de recursos de la organizaci�n. Si bien 
	puede existir el caso de un usuario que requiera un ambiente m�s controlado, como no es el caso general 
	la opci�n por defecto es que sea una instancia de la ejecuci�n global.}
	\reqsource{I.O.R.S.H.}
	\reqfitcriterion{Cuando m�s de un usuario lanza la misma aplicaci�n virtual esta por defecto lo redirige a la
	instancia global.}
	\reqsatisfaction{2}{3}
	\reqdepencies{\reffr{RF:P:REVIEW_PUBLISH}, \reffr{RF:U:PRIVILEGIO}, \reffr{RF:P:VIRTUAL_ENVIRONMENT_CONFIG}}{}
	\reqdocuments{}
	\reqhistory{}
	\label{RF:P:VIRTUAL_ENVIRONMENT_GLOBAL}
\end{requirement}


\begin{requirement}{functional}{}
	\reqdescription{Las instancias temporales de ejecuci�n son �nicas y por defecto est�n globalmente 
	disponibles para los usuarios 
	con los permisos adecuados.}
	\reqrationale{Algunos c�digos fuente o paquetes ejecutables pueden ser desplegados desde dentro de la 
	apliaci�n. En estos casos, si el administrador habilita la opci�n de ejecuci�n temporal, debe de haber
	una opci�n disponible en la vista del proyecto. Sin embargo, no tiene sentido ejecutar m�ltiples 
	instancias de una misma aplicaci�n, esencialmente por un tema de recursos de la organizaci�n. Si 
	bien puede existir el caso de un usuario que requiera un ambiente m�s controlado, como no es el 
	caso general la opci�n por defecto es que sea una instancia de la ejecuci�n global.}
	\reqsource{I.O.R.S.H.}
	\reqfitcriterion{Cuando m�s de un usuario lanza la misma aplicaci�n virtual esta por defecto lo redirige a la
	instancia global.}
	\reqsatisfaction{2}{3}
	\reqdepencies{\reffr{RF:P:REVIEW_PUBLISH}, \reffr{RF:U:PRIVILEGIO}, \reffr{RF:P:VIRTUAL_ENVIRONMENT_CONFIG}}{}
	\reqdocuments{}
	\reqhistory{}
	\label{RF:P:VIRTUAL_ENVIRONMENT_GLOBAL}
\end{requirement}
