\subsection{Revisiones}
\begin{requirement}{functional}{}
	\reqdescription{Administrador puede rechazar un proyecto enviado.}
	\reqrationale{Un proyecto enviado a revisi�n debe ser inspeccionado en busca de artefactos,
	fallas de consistencia, claridad de soluci�n, documentaci�n en orden, paquete ejecutable en
	buen estado, c�digo malicioso, etc. Dependiendo de los resultados el puede aceptar o rechazar
	el mismo.}
	\reqsource{Rodrigo Bustamante}
	\reqfitcriterion{El administrador tiene la opci�n de aceptar o rechazar un proyecto enviado. La 
	desici�n es notificada al usuario que envia el proyecto.}
	\reqsatisfaction{2}{3}
	\reqdepencies{\reffr{RF:P:ENVIO_REVISION}}{}
	\reqdocuments{Entrevista 1}
	\reqhistory{}
	\label{RF:P:REVIEW_REJECT}
\end{requirement}

\begin{requirement}{functional}{}
	\reqdescription{Administrador puede aceptar un proyecto enviado.}
	\reqrationale{Un proyecto enviado a revisi�n debe ser inspeccionado en busca de artefactos,
	fallas de consistencia, claridad de soluci�n, documentaci�n en orden, paquete ejecutable en
	buen estado, c�digo malicioso, etc. Dependiendo de los resultados el puede aceptar o rechazar
	el mismo.}
	\reqsource{Rodrigo Bustamante}
	\reqfitcriterion{El administrador tiene la opci�n de aceptar o rechazar un proyecto enviado. La 
	desici�n es notificada al usuario que envia el proyecto.}
	\reqsatisfaction{4}{3}
	\reqdepencies{\reffr{RF:P:ENVIO_REVISION}}{}
	\reqdocuments{Entrevista 1}
	\reqhistory{}
	\label{RF:P:REVIEW_ACCEPT}
\end{requirement}


\begin{requirement}{functional}{}
	\reqdescription{Administrador puede publicar un proyecto aceptado.}
	\reqrationale{Una vez que un proyecto est� aceptado, este est� listo para ser publicado y tiene por defecto visibilidad privada.}
	\reqsource{Rodrigo Bustamante}
	\reqfitcriterion{El administrador tiene la opci�n de publicar un proyecto aceptado. Este por defecto tiene visibilidad privada-}
	\reqsatisfaction{5}{5}
	\reqdepencies{\reffr{RF:P:REVIEW_ACCEPT},\reffr{RF:U:PRIVILEGIO}}{}
	\reqdocuments{Entrevista 1}
	\reqhistory{}
	\label{RF:P:REVIEW_PUBLISH}
\end{requirement}


\begin{requirement}{functional}{}
	\reqdescription{Administrador puede cambiar el nivel de visibilidad de proyectos.}
	\reqrationale{Una vez que un proyecto est� aceptado, este est� listo para ser publicado y tiene por 
	defecto visibilidad privada. sin embargo, quiz�s el proyecto puede estar pensado en que pueda ser visto
	por usuarios sin cuenta en la plataforma (el caso de los pasantes).}
	\reqsource{Rodrigo Bustamante}
	\reqfitcriterion{El administrador tiene la opci�n de publicar un proyecto aceptado. Este por defecto tiene visibilidad privada-}
	\reqsatisfaction{4}{3}
	\reqdepencies{\reffr{RF:P:REVIEW_PUBLISH},\reffr{RF:U:PRIVILEGIO}}{}
	\reqdocuments{Entrevista 1}
	\reqhistory{}
	\label{RF:P:REVIEW_PERMISSION}
\end{requirement}

\begin{requirement}{functional}{}
	\reqdescription{Un proyecto rechazado debe volver a habilitar la edicion para efectuar cambios.}
	\reqrationale{Los proyectos rechazados deben ser ``reparados'', lo cual no es posible si este est�
	bloqueado para efectuar cambios sobre el mismo.}
	\reqsource{Rodrigo Bustamante}
	\reqfitcriterion{Cuando el administrador rechaza un proyecto, los usuarios con permisos
	de escritura sobre el proyecto pueden volver a efectuar cambios nuevamente.}
	\reqsatisfaction{4}{3}
	\reqdepencies{\reffr{RF:P:REVIEW_REJECT}}{}
	\reqdocuments{Entrevista 1}
	\reqhistory{}
	\label{RF:P:REVIEW_REENABLE}
\end{requirement}
