\section{Mensajeria}
\begin{requirement}{functional}{PUC\_13}
	\reqdescription{El sistema notifica al administrador de las nuevas solicitudes de cuentas
	al momento de ser elevadas al administrador y esta es agregada a la lista de tareas pendientes.}
	\reqrationale{El administrador debe estar sensibilizado de sus tareas.}
	\reqsource{I.O.R.S.H.}
	\reqfitcriterion{Cuando una nueva solicitud es recibida por la plataforma, el usuario
	recibe una notificaci�n en tiempo real.}
	\reqsatisfaction{2}{2}
	\reqdepencies{\reffr{RF:U:ELEVAR_SOLICITUD}}{}
	\reqdocuments{Entrevista 1}
	\reqhistory{}
	\label{RF:M:NOTIF_NUEVA_CUENTA}
\end{requirement}

\begin{requirement}{functional}{PUC\_13}
	\reqdescription{El sistema notifica al usuario de la aceptaci�n o rechazo de la cuenta solicitada.}
	\reqrationale{El usuario necesita saber el estado de las solicitudes.}
	\reqsource{I.O.R.S.H.}
	\reqfitcriterion{Cuando una nueva solicitud es aceptada por el administrador, el usuario
	recibe una notificaci�n en tiempo real.}
	\reqsatisfaction{2}{2}
	\reqdepencies{\reffr{RF:U:ADMIN_SOLICITUD}}{}
	\reqdocuments{Entrevista 3}
	\reqhistory{}
	\label{RF:M:NOTIF_CUENTA_ACEPTADA}
\end{requirement}

\begin{requirement}{functional}{PUC\_12}
	\reqdescription{El sistema notifica al usuario de la aceptaci�n o rechazo de una revisi�n de proyectos.}
	\reqrationale{El usuario necesita saber el estado de las revisiones.}
	\reqsource{I.O.R.S.H.}
	\reqfitcriterion{Cuando una nueva solicitud es aceptada o rechazada por el administrador, el usuario
	recibe una notificaci�n en tiempo real.}
	\reqsatisfaction{2}{2}
	\reqdepencies{\reffr{RF:P:REVIEW_REJECT},\reffr{RF:P:REVIEW_ACCEPT}}{}
	\reqdocuments{Entrevista 6}
	\reqhistory{}
	\label{RF:M:NOTIF_REVIEW}
\end{requirement}


\begin{requirement}{functional}{PUC\_12}
	\reqdescription{El sistema notifica al usuario de la publicaci�n de un proyecto.}
	\reqrationale{Como un acto de buena f� se le informa a las partes involucradas si es que 
	uno de los proyectos ha sido aprobado para poder verse privada o p�blicamente en el sistema.}
	\reqsource{Ruth Garrido, Rodrigo Bustamante}
	\reqfitcriterion{Cuando un proyecto perteneciente a un usuario es publicado, este es notificado.}
	\reqsatisfaction{2}{2}
	\reqdepencies{\reffr{RF:P:REVIEW_PUBLISH}}{}
	\reqdocuments{Entrevista 6}
	\reqhistory{}
	\label{RF:M:NOTIF_PUBLISH}
\end{requirement}


\begin{requirement}{functional}{PUC\_12}
	\reqdescription{El sistema notifica al administrador de una nueva solicitud de revisi�n de proyectos y 
	la agrega a la lista de tareas pendientes.}
	\reqrationale{Es necesario sensibilizar al administrador respecto del estado de las tareas
	que necesita realizar, en este caso, de los proyectos pendientes de revisi�n.}
	\reqsource{Rodrigo Bustamante}
	\reqfitcriterion{Cuando un proyecto es enviado a revisi�n, la lista de tareas pendientes y
	notificaciones son actualizadas.}
	\reqsatisfaction{2}{2}
	\reqdepencies{\reffr{RF:P:ENVIO_REVISION}}{}
	\reqdocuments{Entrevista 7}
	\reqhistory{}
	\label{RF:M:NOTIF_BRINGUP}
\end{requirement}


\begin{requirement}{functional}{PUC\_09}
	\reqdescription{El sistema notifica al moderador de un nuevo comentario realizado en un proyecto de su �rea y 
	la agrega a la lista de tareas pendientes.}
	\reqrationale{Es necesario sensibilizar al moderador respecto del estado de las tareas
	que necesita realizar, en este caso, de los proyectos pendientes de revisar comentarios.}
	\reqsource{Rodrigo Bustamante}
	\reqfitcriterion{Cuando un usuario comenta un proyecto, la lista de tareas pendientes y
	notificaciones son actualizadas.}
	\reqsatisfaction{2}{2}
	\reqdepencies{\reffr{RF:P:NEW_COMMENT}}{}
	\reqdocuments{Entrevista 7}
	\reqhistory{}
	\label{RF:M:NOTIF_COMMENT}
\end{requirement}



\begin{requirement}{functional}{PUC\_02}
	\reqdescription{Los usuarios pueden realizar comentarios sobre un proyecto proveendo su nombre y correo electr�nico.}
	\reqrationale{No es necesario que un usuario est� registrado al moento de comentar, pero el proveer de alg�na cosa que
	permita identificarlo en el caso de producirse una discusi�n es algo �til de implementar..}
	\reqsource{Rodrigo Bustamante}
	\reqfitcriterion{Un usuario puede comentar un proyecto que pueda tener acceso.}
	\reqsatisfaction{2}{2}
	\reqdepencies{\reffr{RF:P:REVIEW_PUBLISH}}{}
	\reqdocuments{Entrevista 7}
	\reqhistory{}
	\label{RF:M:NEW_COMMENT}
\end{requirement}