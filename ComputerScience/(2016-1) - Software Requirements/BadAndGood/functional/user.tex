\begin{requirement}{functional}{BUC02}
	\reqdescription{Usuario puede buscar proyectos en la plataforma de acuerdo a los niveles de acceso del proyecto}
	\reqrationale{El objetivo de almacenar proyectos es su posterior b�squeda, sin embargo, es necesario poder filtrar los resultados de acuerdo a los privilegios de acceso de este para proyectos p�blicos y privados}
	\reqsource{Rodrigo Bustamante}
	\reqfitcriterion{Las b�squedas realizadas entregan proyectos correspondientes al nivel de privilegio del usuario}
	\reqsatisfaction{5}{5}
	\reqdepencies{\reffr{RF:PROYECTOS_PRIVILEGIOS}}{}
	\reqdocuments{Entrevista 1,2, 5}
%	\reqhistory{w}
	\label{RF:BUSQUEDA_PRIVILEGIOS}
\end{requirement}

\begin{requirement}{functional}{BUC02}
	\reqdescription{Usuario puede realizar comentarios a un proyecto}
	\reqrationale{Los usuarios pueden utilizar la plataforma para complementar la informaci�n existente del proyecto o bien para realizar consultas. Estos comentarios son mediados por un moderador.}
	\reqsource{Rodrigo Bustamante}
	\reqfitcriterion{Los proyectos muestran un apartado para realizar comentarios}
	\reqsatisfaction{5}{5}
	\reqdepencies{Ninguna}{}
	\reqdocuments{Entrevista 5}
%	\reqhistory{w}
	\label{RF:PROYECTO_CREAR_COMENTARIO}
\end{requirement}

\begin{requirement}{functional}{BUC02}
	\reqdescription{Moderador debe ser notificado cuando un comentario nuevo se realiza.}
	\reqrationale{Los moderadores est�n encargados de aceptar o no comentarios a proyectos realizados a travez de la plataforma, por tanto deben estar actualizados respecto al estado de estos.}
	\reqsource{Rodrigo Bustamante}
	\reqfitcriterion{Cuando un comentario es creado, los moderadores pertinentes son notificados}
	\reqsatisfaction{5}{5}
	\reqdepencies{\reffr{RF:PROYECTOS_CREAR_COMENTARIO}}{}
	\reqdocuments{Entrevista 5}
%	\reqhistory{w}
	\label{RF:MODERADOR_NOTIFICIACION_NUEVO_COMENTARIO}
\end{requirement}


\begin{requirement}{functional}{BUC02}
	\reqdescription{Los comentarios deben ser almacenados junto a su proyecto correspondiente.}
	\reqrationale{Existe una relaci�n 1:N entre proyectos y comentarios, por tanto la informaci�n de los comentarios solo est� relacionada a estos.}
	\reqsource{I.O.R.S.H. Dev Team}
	\reqfitcriterion{Los comentarios de un proyecto solo pertenecen a ellos y no son visibles desde otros.}
	\reqsatisfaction{5}{5}
	\reqdepencies{\reffr{RF:PROYECTO_CREAR_COMENTARIO}}{}
	\reqdocuments{Entrevista 5}
%	\reqhistory{w}
	\label{RF:COMENTARIO_RELACION_PROYECTO}
\end{requirement}