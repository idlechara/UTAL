\documentclass{beamer}
% Class options include: notes, handout, trans
%                        

% Theme for beamer presentation 
% Other themes include: beamerthemebars, beamerthemelined,beamerthemetree, beamerthemeplain
\usetheme{Copenhagen}
\usepackage[latin1]{inputenc}
\usepackage[T1]{fontenc}
\def\spanishoptions{chile}
\usepackage[spanish]{babel}
\selectlanguage{spanish}
\usepackage{amssymb}
\usepackage{soul}
\usepackage{graphicx}
\usepackage{listings}
\definecolor{listinggray}{gray}{0.9}
\definecolor{lbcolor}{rgb}{0.9,0.9,0.9}
\lstset{
backgroundcolor=\color{lbcolor},
    tabsize=4,    
%   rulecolor=,
    language=[GNU]C++,
        basicstyle=\scriptsize,
        upquote=true,
        aboveskip={1.5\baselineskip},
        columns=fixed,
        showstringspaces=false,
        extendedchars=false,
        breaklines=true,
        prebreak = \raisebox{0ex}[0ex][0ex]{\ensuremath{\hookleftarrow}},
        frame=single,
        numbers=left,
        showtabs=false,
        showspaces=false,
        showstringspaces=false,
        identifierstyle=\ttfamily,
        keywordstyle=\color[rgb]{0,0,1},
        commentstyle=\color[rgb]{0.026,0.112,0.095},
        stringstyle=\color[rgb]{0.627,0.126,0.941},
        numberstyle=\color[rgb]{0.205, 0.142, 0.73},
%        \lstdefinestyle{C++}{language=C++,style=numbers}?.
}
\lstset{
    backgroundcolor=\color{lbcolor},
    tabsize=4,
  language=C++,
  captionpos=b,
  tabsize=3,
  frame=lines,
  numbers=left,
  numberstyle=\tiny,
  numbersep=5pt,
  breaklines=true,
  showstringspaces=false,
  basicstyle=\footnotesize,
%  identifierstyle=\color{magenta},
  keywordstyle=\color[rgb]{0,0,1},
  commentstyle=\color[rgb]{0.026,0.112,0.095},
  stringstyle=\color{red}
  }

\usefonttheme{professionalfonts}


\title[Introducci�n a la administraci�n]{Introducci�n a la administraci�n}
\subtitle{Tarea 5 - An�lisis de caso de estudio}    % Enter your title between curly braces
\author[E. Regla]{Erik Regla}                 % Enter your name between curly braces
\institute[UTalca]{Universidad de Talca}      % Enter your institute name between curly braces
\date{\today}      % Enter the date or \today between curly braces

\begin{document}

% Creates title page of slide show using above information
\begin{frame}
  \titlepage
\end{frame}
\note{Talk for 30 minutes} % Add notes to yourself that will be displayed when typeset with the notes class option.

%% Creates table of contents slide incorporating all \section and \subsection commands.
%\begin{frame}
%  \tableofcontents
%\end{frame}

\section{Descripci�n}
\begin{frame}
\frametitle{Descripci�n de la situaci�n}
\textit{El alcalde Bornes} hace seis semanas ha asumido el mando y no tiene idea de que ocurre en sus departamentos. La situaci�n es peor de lo que imaginaba y necesita saber que tiene que hacer para poder remediar la situaci�n actual.
\end{frame}

\section{Preguntas}
\begin{frame}
\frametitle{Preguntas}
\begin{itemize}
\item \textbf{No se conoce el presupuesto gastado este a�o por departamento. Retraso importante (6 meses) en reportes de gastos.} Motivo por el cual esto se ha producido. �Desde cu�ndo esta situaci�n est� as� de mal?

\item \textbf{No se conocen las actividades ejecutadas por departamento.} �Cada departamento posee v�as de comunicaci�n hacia los dem�s? �Entregan informaci�n a su unidad superior?
\end{itemize}
\end{frame}

\begin{frame}
\frametitle{Preguntas}
\begin{itemize}
\item \textbf{Desconocimiento de mecanismos de sistema (indicadores) para evaluar situaci�n y desempe�o.} �Anteriormente hubieron indicadores? �Cu�les fueron los resultados de las evaluaciones? �Los departamentos controlan el desempe�o de manera interna?

\item \textbf{No existe retroalimentaci�n.} �Existe retroalimentaci�n por parte de la organizaci�n, interna o de departamentos superiores? �Que se espera lograr con esta? �Est� sistematizada y con normas claras y definidas?
\end{itemize}
\end{frame}


\section{Descripci�n del tipo de control}
\begin{frame}
\frametitle{Descripci�n del tipo de control - Fomento de avance}

Claramente se puede identificar que es necesario establecer un est�ndar a seguir para los tiempos de respuesta de cada departamento y mecanismos de retroalimentaci�n para poder tener conocimiento de la situaci�n global. Se deben de implementar normas, fechas l�mite de termino para actividades las cuales deben estar claramente definidas y con un responsable.\\

\end{frame}
\begin{frame}
\frametitle{Descripci�n del tipo de control - Concurrente}
Para el control concurrente se consideran las supervisiones de actividades que se desarrollan de manera conforme a las normas establecidas, identificando dificultades que pudiesen afectar el proceso. 

\end{frame}
\begin{frame}
\frametitle{Descripci�n del tipo de control - Retroalimentaci�n}
Con esto, una vez establecidos los canales de comunicaci�n, cada departamento (y municipio) debe establecer sus m�tricas e indicadores, los cuales le permiten evaluar el desempe�o. A su vez, la retroalimentaci�n debe ser basada en normas y ser sistematizada, basada en hechos enfatizando aspectos que pueden ser cambiados a nivel de la organizaci�n.

\end{frame}

\begin{frame}
\begin{center}
Fin
\end{center}
\end{frame}

\begin{frame}
  \titlepage
\end{frame}

\end{document}
