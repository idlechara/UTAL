\section{Problema del logaritmo}
\subsection{Aproximación usando sumatoria por integral de $ln(n!)$}

Podemos descomponer $ln(n!)$ usando la propiedad aditiva, lo cual nos queda:
\begin{equation}
	ln(n!) = ln(1) + ln(2) + ln(3) + \ldots + ln(n-1) + ln(n) 
	=
	\sum_{i=1}^{n}ln(n)
\end{equation}\label{EQ:LOGARITMO_ADITIVA}
\begin{equation}\label{EQ:INTEGRAL_SUMATORIA}
	\sum_{}^{}ln(n) \simeq \int ln(n)\,dn
\end{equation}
\begin{equation}\label{EQ:LOGARITMO_COTA_INFERIOR_1}
	\int_{2}^{n+1} \! ln(n) \, \mathrm{d}n 
	\leq
	\sum_{i=1}^{n}ln(n)
\end{equation}
\begin{equation}\label{EQ:LOGARITMO_COTA_SUPERIOR_2}
	\sum_{i=1}^{n}ln(n)
	\leq
	\int_{1}^{n} \! ln(n) \, \mathrm{d}n 
\end{equation}

Sumatoria que usando el concepto de integral queda reescrita como la forma de la integral indefinida \ref{EQ:INTEGRAL_SUMATORIA}, permitiendo acotar la sumatoria anterior por \ref{EQ:LOGARITMO_COTA_INFERIOR_1} y \ref{EQ:LOGARITMO_COTA_SUPERIOR_2}.

%%\subsection{Desarrollo de la Ecuaci'on \ref{EQ:INTEGRAL_SUMATORIA}}
Usando integración por partes tenemos:
\begin{equation}
	\int ln(n)dn = n\;(ln(n)-1)
\end{equation}

\subsection{Cota inferior}
\begin{align}
	\int_{2}^{n+1} \! ln(n)\;dn &= n(ln(n)-1) \bigg|_{2}^{n+1} \nonumber \\
	&=(n+1)\;\left(ln(n+1)-1\right) - 2\;ln(2) - 1	\nonumber \\
	&=(n+1)\;\left(ln\left(\frac{n+1}{e}\right)\right)-2\;ln\left(\frac{2}{e}\right) \nonumber \\
	&=ln\left(\frac{(n+1)^{n+1}}{4e^{n-1}}\right) \label{EQ:CONCLUSION_COTA_INFERIOR}
\end{align}

\subsection{Cota superior}
\begin{align}
	\int_{1}^{n} \! ln(n)\;dn &= n(ln(n)-1) \bigg|_{1}^{n} \nonumber \\
	&=n\;\left(ln(n)-1\right) - 1 \nonumber \\
	&=n\;ln\left(\frac{n}{e}\right) - ln(e) \nonumber \\
	&=ln\left(\frac{n^{n}}{e^{n-1}}\right) \label{EQ:CONCLUSION_COTA_SUPERIOR}
\end{align}

\subsection{Conclusión}
Usando las Ecuación \ref{EQ:CONCLUSION_COTA_INFERIOR} y la Ecuación \ref{EQ:CONCLUSION_COTA_SUPERIOR}, podemos concluir que:
\begin{align}
	ln\left(\frac{n^{n}}{e^{n-1}}\right)
	\leq
	ln(n!)
	\leq
	ln\left(\frac{(n+1)^{n+1}}{4e^{n-1}}\right) \nonumber
\end{align}
