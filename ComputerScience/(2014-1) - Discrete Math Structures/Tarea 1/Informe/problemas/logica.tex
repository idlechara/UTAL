\section{Inducción sobre fórmulas lógicas}
\subsubsection{Base inductiva}
Definiremos los operadores lógicos $\vee$, $\wedge$, $\Rightarrow$ y $\Leftrightarrow$:

Sea $1=Verdadero$ y $0=Falso$:

\[ A \vee B =
 \begin{cases}
 0,	\qquad& \text{si $A=0$ y $B=0$}.\\
 1,	\qquad& \text{en otro caso}.\\
\end{cases} \]

\[ A \wedge B =
 \begin{cases}
 1, 	\qquad& \text{si $A=1$ y $B=1$}.\\
 0,	 	\qquad& \text{en otro caso}.\\
\end{cases} \]

\[ A \rightarrow B =
 \begin{cases}
 0,	 	\qquad& \text{si $A=1$ y $B=0$}.\\
 1,		\qquad& \text{en otro caso}.\\
\end{cases} \]

\[ A \Leftrightarrow B =
 \begin{cases}
 1, 	\qquad& \text{si $A$ y $B$ son iguales}.\\
 0,	 	\qquad& \text{en otro caso}.\\
\end{cases} \]


\begin{align}
\left(\bigvee\limits_{i=1}^{1}X_i\right)\rightarrow Y &= x_1 \rightarrow Y \Leftrightarrow \bigwedge\limits_{i=1}^{1}(X_i\rightarrow Y) = x_1\rightarrow Y \label{EQ:ORATORIA_BI_1}\\
\end{align}

\subsection{Hipotésis inductiva} 
\begin{align}
\left(\bigvee\limits_{i=1}^{n}X_i\right)\rightarrow Y \Leftrightarrow \bigwedge\limits_{i=1}^{n}(X_i\rightarrow Y) 
\end{align}

Lo cual demuestra nuestra base.


\subsection{Tésis inductiva}


Resolviendo la tabla de verdad para $(A \rightarrow C) \wedge (B \rightarrow C)$:\\
\begin{center}
\begin{tabular}{| l | l | l | l | l | l |}
	A	&B	&C	&A$\rightarrow$C	&B$\rightarrow$C	&Resultado\\
	\hline
	1       &1       &1       &1       &1       &1\\
	1       &1       &0       &0       &0       &0\\
	1       &0       &1       &1       &1       &1\\
	1       &0       &0       &0       &1       &0\\
	0       &1       &1       &1       &1       &1\\
	0       &1       &0       &1       &0       &0\\
	0       &0       &1       &1       &1       &1\\
	0       &0       &0       &1       &1       &1\\
\end{tabular}
\end{center}


\[ (A \rightarrow Y) \wedge (B \rightarrow Y) =
 \begin{cases}
 0, 	\qquad& \text{si $Y=0$ y $(A\vee B) = 1$ }.\\
 1,	 	\qquad& \text{en otro caso}.\\
\end{cases} \]

Por lo cual podemos asociar $A$ y $B$ de esta manera:
\begin{equation}
(A \rightarrow Y) \wedge (B \rightarrow Y) = (A\vee B) \rightarrow Y = u \rightarrow Y
\end{equation}\label{EQ:ANDATORIA_PROPIEDAD_1}

Entonces, tomando nuestra hipótesis:
\begin{align}
\left(\bigvee\limits_{i=1}^{n}X_i\right)\rightarrow Y &= (x_1\vee x_2\vee \ldots \vee x_{n-1} \vee x_n)\rightarrow Y\label{EQ:ORATORIA_TI_1}\\
\bigwedge\limits_{i=1}^{n}(X_i\rightarrow Y) &= (x_1\rightarrow Y) \wedge (x_2\rightarrow Y) \wedge \ldots \wedge (x_{n-1}\rightarrow Y) \wedge (x_n\rightarrow Y) \label{EQ:ANDATORIA_TI_1}
\end{align}

Usando el cambio de variable \ref{EQ:ANDATORIA_PROPIEDAD_1} sobre la Operatoria \ref{EQ:ORATORIA_TI_1} y haciendo uso de la propiedad asociativa del operador $\vee$ tenemos:

\begin{align}
\bigwedge\limits_{i=1}^{n}(X_i\rightarrow Y) &= ((x_1\vee x_2)\rightarrow Y) \wedge ((x_3\vee x_4)\rightarrow Y) \wedge \ldots \wedge ((x_{n-1}\vee x_n)\rightarrow Y)
\nonumber\\
\bigwedge\limits_{i=1}^{n}(X_i\rightarrow Y) &= ((x_1\vee x_2\vee x_3 \vee x_4)\rightarrow Y) \wedge ((x_5\vee x_6\vee x_7 \vee x_8)\rightarrow Y) \wedge \ldots \wedge ((x_{n-3}\vee x_{n-2}\vee x_{n-1}\vee x_n)\rightarrow Y)
\nonumber\\
&\vdots\nonumber\\
\bigwedge\limits_{i=1}^{n}(X_i\rightarrow Y) &= ((x_1\vee x_2\vee \ldots \vee x_{n-1} \vee x_n)\rightarrow Y) \label{EQ:ANDATORIA_TI_2}
\end{align}

Es obvio que la Operatoria \ref{EQ:ANDATORIA_TI_2} es idéntica a la Operatoria \ref{EQ:ORATORIA_TI_1}, con lo cual, hemos demostrado que ambas expresiones son lógicamente equivalentes.
