\section{Plumíferos}
\subsection{$\Sigma$}

\subsubsection{Ningun pato está dispuesto a bailar cumbia.}
\begin{equation}
\forall x, \neg (P(X) \rightarrow V(x))
\end{equation}\label{PRED:2}

\subsubsection{Ningún oficial rechazaría bailar cumbia.}
\begin{equation}
\forall x, \neg(A(X) \rightarrow \neg V(x))
\end{equation}\label{PRED:2}


\subsubsection{Todas mis aves de corral son patos.} 
\begin{equation}
\forall x, C(x)  \rightarrow P(x)
\end{equation}\label{PRED:3}

\subsection{Demostración: Mis aves de corral no son agentes de policía}
Dado que todas están con cuantificador universal, podemos obviarlo:\\

\begin{tabular}{ c  c  l }
1:& $\neg (P(x) \rightarrow V(x))$ & pertenece a $\Sigma$\\
2:& $\neg (\neg P(x) \vee V(x))$ & \textit{implicación material} sobre 1\\
3:& $P(x) \wedge \neg V(x)$ & \textit{Teorema de Morgan} en 2\\
4:& $\neg(A(X) \rightarrow \neg V(x))$ & pertenece a $\Sigma$\\
5:& $\neg (\neg A(x) \vee V(x))$ & \textit{implicación material} 4\\
6:& $\neg A(x) \wedge \neg V(x)$ & \textit{Teorema de Morgan} en 5\\
7:& $P(x) \wedge \neg V(x) \wedge \neg A(x) \wedge \neg V(x)$ & \textit{conjunción} de 3 y 6\\
8:& $P(x) \wedge \neg A(x)$ & \textit{simplificación} de 7\\
9:& $C(x)  \rightarrow P(x)$ & pertenece a $\Sigma$\\
10:& $C(x)  \wedge \neg A(x)$ & \textit{Modus Ponens} de 8 y 9\\
\end{tabular}

Error.

Es obvio que $C(x)  \wedge \neg A(x)$ es igual a $\neg (C(x) \rightarrow \neg A(x))$, lo cual no es en nada parecido a $C(x) \rightarrow \neg A(x)$ la cual es la afirmación a verificar. Esto nos lleva a pensar que quizás hay una condición mal escrita.



\subsection{$\Sigma$ segunda versión}

En este caso, usaremos expresiones lógicamente equivalentes a las anteriores.

\subsubsection{Ningun pato está dispuesto a bailar cumbia. Todos los patos no están dispuestos a bailar cumbia.}
\begin{equation}
\forall x, P(X) \rightarrow \neg V(x)
\end{equation}\label{PRED:2_1}

\subsubsection{Ningún oficial rechazaría bailar cumbia. O mejor dicho, todos los oficiales están dispuestos a bailar cumbia.}
\begin{equation}
\forall x, A(X) \rightarrow V(x)
\end{equation}\label{PRED:2_1}


\subsubsection{Todas mis aves de corral son patos.} 
\begin{equation}
\forall x, C(x)  \rightarrow P(x)
\end{equation}\label{PRED:3_1}

\subsection{Demostración: Mis aves de corral no son agentes de policía}
Dado que todas están con cuantificador universal, podemos obviarlo:\\

\begin{tabular}{ c  c  l }
1:& $C(x) \rightarrow P(x)$ & pertenece a $\Sigma$\\
2:& $P(x) \rightarrow \neg V(x)$ & pertenece a $\Sigma$\\
3:& $C(x) \rightarrow \neg V(x)$ & \textit{Siglogismo Hipotético} en 1 y 2\\
4:& $A(x) \rightarrow V(x)$ & pertenece a $\Sigma$\\
5:& $C(x) \rightarrow \neg A(x)$ & \textit{Modus Tollens} en 3 y 4
\end{tabular}

Lo cual demuestra nuestra premisa: \textit{Mis aves de corral no son agentes de policía}.
