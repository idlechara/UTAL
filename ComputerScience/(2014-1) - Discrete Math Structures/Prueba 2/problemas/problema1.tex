\section{Pregunta 1}

\subsection{¿Al Menos Tan sabio?}

\begin{tabular}{ c | c | c }
  \hline                       
  $alMenosTanSabioQue(a,b)$ & $alMenosTanSabioQue(b,a)$ & $alMenosTanSabioQue(a,b) \vee alMenosTanSabioQue(b,a) $ \\
  \hline                
  Verdadero 	& Falso 		& Verdadero \\
  Verdadero 	& Verdadero 	& Verdadero \\  
  Falso 		& Falso 		& Falso \\
  Falso 		& Verdadero 	& Verdadero \\
  \hline  
\end{tabular}

\subsubsection{Conclusión}
Contingencia.

\subsubsection{Argumento}
No conocemos cual es el criterio para determinar que tan sabio puede ser una persona en comparaci'on a la otra. Dado que $a$ y $b$ son arbitrarios, la combinaci'on se resume a los resultados que pueda arrojar la caja negra $alMenosTanSabioQue(x,y)$. No se especifica conocimiento alguno de las reglas para determinar sus valores tampoco. Por ende, se asume que las entradas son las propias salidas de esas funciones.


\subsection{¿Un primo es impar?}

\begin{tabular}{ c | c | c | c }
  \hline                       
  $primo(a)$ & $impar(a)$ & $impar(a) \rightarrow primo(a)$  & $primo(a) \rightarrow (impar(a) \rightarrow primo(a))$\\
  \hline                
  Verdadero 	& Falso 		& Verdadero 	&Verdadero \\
  Verdadero 	& Verdadero 	& Verdadero 	&Verdadero \\  
  Falso 		& Falso 		& Verdadero 	&Verdadero \\
  Falso 		& Verdadero 	& Falso 		&Verdadero \\
  \hline  
\end{tabular}

\subsubsection{Conclusión}
Tautología.

\subsubsection{Argumento}
Idem al caso anterior.




\subsection{¿Es una cosa mejor que la otra?}

\begin{tabular}{ c | c | c | c }
  \hline                       
  $mejorQue(a,b)$ & $mejorQue(b,a)$ & $\neg mejorQue(b,a)$ & $mejorQue(a,b)\rightarrow \neg mejorQue(b,a)$\\
  \hline                
  Verdadero 	& Falso 		& Verdadero 	&Verdadero \\
  Verdadero 	& Verdadero 	& Falso	 		&Verdadero \\  
  Falso 		& Falso 		& Verdadero 	&Falso 		\\
  Falso 		& Verdadero 	& Falso 		&Verdadero \\
  \hline  
\end{tabular}

\subsubsection{Conclusión}
Contingencia.

\subsubsection{Argumento}
Idem al caso anterior.
