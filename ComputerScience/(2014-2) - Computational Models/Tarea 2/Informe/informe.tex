\documentclass[11pt]{utalcaDoc}
\usepackage{alltt}
\usepackage{underscore}
\usepackage[latin1]{inputenc}
\usepackage[activeacute,spanish]{babel}
\usepackage{verbatim}
\usepackage{graphicx}
\usepackage{ae}
\usepackage{amsmath}
\usepackage{mathtools}
\usepackage{algorithm}
\usepackage{algorithmic}
\usepackage{amsfonts}
\usepackage{algo}
\floatname{algorithm}{Algoritmo}
\renewcommand{\listalgorithmname}{Lista de algoritmos}
\renewcommand{\algorithmicrequire}{\textbf{Precondici\'{o}n:}}
\renewcommand{\algorithmicensure}{\textbf{Postcondici\'{o}n:}}
\renewcommand{\algorithmicend}{\textbf{fin}}
\renewcommand{\algorithmicif}{\textbf{si}}
\renewcommand{\algorithmicthen}{\textbf{entonces}}
\renewcommand{\algorithmicelse}{\textbf{si no}}
\renewcommand{\algorithmicelsif}{\algorithmicelse,\ \algorithmicif}
\renewcommand{\algorithmicendif}{\algorithmicend\ \algorithmicif}
\renewcommand{\algorithmicfor}{\textbf{para}}
\renewcommand{\algorithmicforall}{\textbf{para todo}}
\renewcommand{\algorithmicdo}{\textbf{hacer}}
\renewcommand{\algorithmicendfor}{\algorithmicend\ \algorithmicfor}
\renewcommand{\algorithmicwhile}{\textbf{mientras}}
\renewcommand{\algorithmicendwhile}{\algorithmicend\ \algorithmicwhile}
\renewcommand{\algorithmicloop}{\textbf{repetir}}
\renewcommand{\algorithmicendloop}{\algorithmicend\ \algorithmicloop}
\renewcommand{\algorithmicrepeat}{\textbf{repetir}}
\renewcommand{\algorithmicuntil}{\textbf{hasta que}}
\renewcommand{\algorithmicprint}{\textbf{imprimir}} 
\renewcommand{\algorithmicreturn}{\textbf{devolver}} 
\renewcommand{\algorithmictrue}{\textbf{verdadero }} 
\renewcommand{\algorithmicfalse}{\textbf{falso }} 


\title{{\bf Modelos de Computabilidad}\\Tarea 2 - Informe}
\author{Erik Regla\\ eregla09@alumnos.utalca.cl}
\date{10 de Diciembre del 2014}

\begin{document}
\renewcommand{\figurename}{Figura~}
\renewcommand{\tablename}{Tabla~}

\maketitle

\section{Introducci�n}
Dada una calculadora infija proporcionada como base, dise�ar una calculadora prefija que soporte suma, resta, multiplicaci�n, divisi�n, potencia, negaci�n, asignaci�n de variables y soporte para numeros de doble prescici�n.
\section{An�lisis del problema}

\subsection{Desambiguaci�n del Lenguaje}

\begin{figure}
\begin{center}
\begin{verbatim}
operacion   :   NUM                      { $$ = $1;      }
            |   '+' lista_suma expresion { $$ = $2 + $3; }
			;
			
lista_suma  :   lista_suma  expresion    { $$ = $1 + $2; }
            |   expresion                { $$ = $1;      }
            ;

\end{verbatim}
\end{center}
\caption{Ejemplo de desambiguaci�n para la operaci�n \textit{suma}.}
\label{DEFINICION_LISTA}
\end{figure} 

\subsubsection{Descripci�n del problema}
Las operaciones solicitadas piden que se ejecuten las operaciones en un formato similar al de \textit{Racket}, cada operaci�n rodeada de par�ntesis, el primer miembro siendo el operador en caso de estar disponible. El problema se presenta en que dado que el operador se presenta en un extremo de la \textit{expresion}, entonces no tenemos forma de saber que operaci�n le precede a cada elemento. Para lo cual se implement� una lista inherente a cada tipo de operaci�n, de modo tal que permita alojar expresiones de forma recursiva y siempre aplique la misma operaci�n a los miembros de su nivel (ver ejemplo en la Figura \ref{DEFINICION_LISTA}). Esta misma desambiguaci�n fue realizada para las operaciones \textit{resta}, \textit{multiplicaci�n} y \textit{divisi�n}. 

\section{Implementaci�n}
\subsection{Parser L�xico}


\begin{algorithm}
\begin{algorithmic}[1]
\REQUIRE $W$ es un puntero a un caracter del lenguaje
\ENSURE $NUM$ si se encuentra un n�mero, un caracter si se encuentra alguna otra cosa, error en caso contrario.

\STATE consumir espacios en blanco de $W$
\IF{$W$ es un d�gito o un punto}
	\STATE $NUM \leftarrow $leer numero en $W$
	\RETURN $NUM$
\ELSIF{$W$ es un caracter}
	\RETURN $W$
\ENDIF
\RETURN error
\end{algorithmic}
\caption{BFS}\label{ALG:LEX_PARSER}
\end{algorithm}

\subsection{Variables}
Las variables son almacenadas en un arreglo unidimensional de tipo double, Todas las variables si no han sido inicializadas tienen como valor por defecto 0. El tratamiento de la asignaci�n se hace en la definici�n de \textit{variable}, de modo de mantener la coherencia del resto de la sint�xis.

\section{Notas}
\begin{itemize}
\item Las fuentes de \textit{Blumenkranz}, el programa que implementa los algoritmos anteriormente descritos, est� alojado en  \texttt{https://bitbucket.org/eregla/ecd2014-1/} al igual que la prueba y tarea anterior.
\item El programa est� licenciado bajo una licencia abierta MIT.
\end{itemize}
%\newpage
%\bibliographystyle{alpha}
%\bibliography{informe}
\end{document}
