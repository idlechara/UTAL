\section{Fibonacci recursivo}
\subsection{Resolución analítica}
\begin{align}
	t_n = t_{n-1}t_{n-2}\nonumber\\
	t_1 = 1 ,\; t_2=2 \nonumber 	
\end{align}

Aplicando la propiedad aditiva de logaritmo:

\begin{align}
	&log_2(t_n)=log_2(t_{n-1} \; t_{n-2}) \nonumber\\
	&log_2(t_n)=log_2(t_{n-1})+log_2(t_{n-2}) \label{EQ:FIBO_RECURSIVO_1}\\
	&log_2(t_1=0)=1 \;,\; log_2(t_2=1)=2 \nonumber\\
\end{align}

Aplicando la transformación $f(n) = log(T(n))$ a la Ecuación \ref{EQ:FIBO_RECURSIVO_1}:

\begin{align}
	f(n) = f(n-1) + f(n-2) \label{EQ:FIBONACCI} \\
	f(n+2) = f(n+1) + f(n) \label{EQ:FIBONACCI_DESPLAZADO}\\
	f(0) = 1 \;,\; f(1) = 2 \nonumber
\end{align}

En donde la Ecuación Recursiva \ref{EQ:FIBONACCI} es posible resolverse usando la tecnica \textit{polinonio característico}.\footnote{Esta recursión es más conocida cómo la \textit{sucesión de fibonacci}} Para esto, se hace avanzar la Ecuación \ref{EQ:FIBONACCI} en dos pasos, lo cual nos deja con la Ecuación \ref{EQ:FIBO_RECURSIVO_1} a la cual, procedemos a extraer el polinomio caracteristico.

\begin{align}
	\lambda^{n+2} - \lambda^{n+1} - \lambda^{n} &= 0 \nonumber\\
	\lambda^{n}\;(\lambda^{2} - \lambda - 1) &= 0 \label{EQ:POLINOMIO_FIBONACCI_1}\\	
	\lambda^{2} - \lambda - 1 &= 0 \label{EQ:POLINOMIO_FIBONACCI_SIMP}\\
	\nonumber\\
	\lambda_1 = \frac{1+\sqrt{5}}{2} \nonumber\\
	\lambda_2 = \frac{1-\sqrt{5}}{2} \nonumber
\end{align}

De la Ecuación \ref{EQ:POLINOMIO_FIBONACCI_1} se tiene que $\lambda=0$ es solución del polinomio, sin embargo, por esto mismo es despreciado, lo cual nos deja con la Ecuación \ref{EQ:POLINOMIO_FIBONACCI_SIMP} como polinomio característico de la Ecuación \ref{EQ:FIBONACCI}. Entonces, ahora reemplazando en la matriz los casos base de la Ecuación \ref{EQ:FIBO_RECURSIVO_1} los valores obtenidos para $\lambda_1$ y $\lambda_2$ tenemos lo siguiente:

 
\begin{align}
	c_1 + c_2 &= f_0 = 0 \label{EQ:FIBO_MATRIX_1} \\
	c_1\;\lambda_{1} + c_2\;\lambda_{2} &= f_1 = 1 \label{EQ:FIBO_MATRIX_2}
\end{align}
Para la Ecuación \ref{EQ:FIBO_MATRIX_1} tenemos:
\begin{align}
	c_1 = -c_2 \label{EQ:FIBO_MATRIX_3}
\end{align}
Aplicando la Ecuación \ref{EQ:FIBO_MATRIX_3} en \ref{EQ:FIBO_MATRIX_2}:
\begin{align}
	c_1\lambda_1 - c_1\lambda_2&=1 \nonumber\\
	c_1\;(\lambda_1 +\lambda_2) &= 1\nonumber\\
	c_1 &= \frac{1}{\sqrt{5}} \nonumber\\
	c_2 &= -\frac{1}{\sqrt{5}} \nonumber
\end{align}

Ahora reemplazamos los valores obtenidos para $\lambda_1, \lambda_2, c_1, c_2$ en la fórmula del polinomio característico (\ref{EQ:POLINOMIO_CARACTERISTICO_DEFINICION}), dando lugar a la Ecuación \ref{EQ:VALOR_FIBONACCI}.
\begin{align}
	f_n &= c_1 \lambda_{1}^{n} + c_2 \lambda_{2}^{n} + \ldots + c_n \lambda_{n}^{n}\ \label{EQ:POLINOMIO_CARACTERISTICO_DEFINICION}\\	
	f_n &= c_1 \lambda_{1}^{n} + c_2 \lambda_{2}^{n}\nonumber\\	
	f_n &= \frac{1}{\sqrt{5}}\;\left(\frac{1+\sqrt{5}}{2}\right)^n -\frac{1}{\sqrt{5}}\;\left(\frac{1-\sqrt{5}}{2}\right)^n \label{EQ:VALOR_FIBONACCI}
\end{align}

Deshaciendo la transformación $f(n) = log(T(n))$:
\begin{align}
log_2(t_n)&= \frac{1}{\sqrt{5}}\;\left(\frac{1+\sqrt{5}}{2}\right)^n -\frac{1}{\sqrt{5}}\;\left(\frac{1-\sqrt{5}}{2}\right)^n \nonumber\\
t_n &= 2^{\left(\frac{1}{\sqrt{5}}\;\left(\frac{1+\sqrt{5}}{2}\right)^n -\frac{1}{\sqrt{5}}\;\left(\frac{1-\sqrt{5}}{2}\right)^n\right)}\nonumber\\
t_n&=2^{f(n)} \\
t_1 = f(0) = 1 \;&,\; t_2 = f(1) = 2
\end{align}

Donde $f_n$ es el $n$-ésimo término de la sucesión de fibonacci. Para efectos de cálculo, se usará directamente la recursión de fibonacci para obtener el valor a calcular, dado que por motivos de programación entera, el cálculo de la raíces provoca un error de precisión el cual arruina los resultados.

Usando el programa adjunto, se puede comprobar que el $25$-ésimo término de la recursión es aproximadamente $1.4415806\times10^{13951}$.

Para realizar la implementación del programa, se utilizó GNU MP\footnote{https://gmplib.org/manual/}. Esta es una librería escrita en C, orientada principalmente al trabajo con numeros racionales, enteros y flotantes de precisión infinita, enfocandose en el rendimiento independiente del tamaño del número a calcular. Hay actualmente incluso versiones en \textit{Assembler} optimizadas para muchos procesadores. Es software libre licenciado bajo la licencia \textbf{GNU LGPL v3}\footnote{https://www.gnu.org/licenses/lgpl.html} y \textbf{GNU GPL v2.0}\footnote{https://www.gnu.org/licenses/gpl-2.0.html}.