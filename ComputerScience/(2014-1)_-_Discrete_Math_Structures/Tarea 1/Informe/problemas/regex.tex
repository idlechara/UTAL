\section{Inducción sobre expresiones regulares}
\subsection{Demostrar que $(0\cdot(1\cdot0)\star\cdot1)|(0\cdot1)$ es una expresión regular.}
\begin{eqnarray*}
	(0\cdot(1\cdot0)\star\cdot1)&|&(0\cdot1)\\
	(0\cdot (ER)\star\cdot1)&|&(ER)\\
	(0\cdot ER\star\cdot1)&|& ER\\
	(0\cdot ER\cdot1)&|& ER\\
	(0\cdot ER)&|& ER\\
	(ER)&|& ER\\
	ER &|& ER\\
   	&ER&\\
\end{eqnarray*}

\subsection{Explique por qué $(0(0|1\star|00)$ no es una expresión regular. Explique como corregir estos problemas.}
 \begin{eqnarray*}
 	(0( 0| 1\star &|& 00)\\
 	(0( 0|ER &|& 00)\\
 	(0( ER &|& 00)\\
 \end{eqnarray*}
	De esto se desprende que ER|00 no es una expresión regular, para ello, debemos ańadir una definición para $\Sigma$ que permita juntar 2 o más caracteres de éste. Así:
 \begin{equation}
 	\forall \text{ c, d } \in \Sigma \text{ , cd es una ER }
 \end{equation} \label{EQ:CD}
	Y luego, de vuelta al ejercicio:
 \begin{eqnarray*}
 	(0(ER&|&00)\\
 	(0(ER&|&ER)\\
 	(0&(ER)&\\
 	&(0ER&\\
 	&(ER&\\
 \end{eqnarray*}
 	Y ahora, nuevamente hemos llegado a algo que no es expresión regular ("(ER"). Para solucionar este nuevo problema, debemos generar una nueva definición para $\Sigma$.
 \begin{equation}
 	\text{Si } E_1 \text{es una ER, (}E_1 \wedge E_1\text{)} \text{ son ERS.}
 \end{equation}\label{EQ:PARENTESIS_INDIVIDUALES}
 
 Luego, finalmente, las nuevas definiciones requeridas para comprobar que $(0(0|1\star|00)$ es una expresión regular son \ref{EQ:CD} y \ref{EQ:PARENTESIS_INDIVIDUALES}.

\subsection{Demostración del número de operadores y paréntesis}
Contraejemplo:

Si para toda expresión regular válida se cumple la Propiedad \ref{EQ:REGEX_RULE}, entonces, para $0****$ también debería cumplirse.

\begin{align}
0****\label{EQ:REGEX_00}\\
\overbrace{0*}^\text{ER}&*** \label{EQ:REGEX_01}\\
\overbrace{ER*}^\text{ER}&** \nonumber\\
\overbrace{ER*}^\text{ER}&* \nonumber\\
\overbrace{ER*}^\text{ER}& \nonumber\\
ER&\nonumber
\end{align}


Dado que la Expresión Regular \ref{EQ:REGEX_01} es una expresión regular válida:
\begin{align}
\#operadores &\leq \#par\'{e}ntesis + \#carácteres\_de\_\Sigma \label{EQ:REGEX_RULE}\\
4 &\leq 0+1 \nonumber\\
4 &\leq 1 \label{EQ:REGEX_02}
\end{align}

Por lo cual, podemos afirmar usando \ref{EQ:REGEX_02}, que la Expresión Regular \ref{EQ:REGEX_00}, no cumple con \ref{EQ:REGEX_RULE}, por ende no es posible demostrar esta regla.

\subsection{Bonus}


\begin{align}
\Phi \label{EQ:REGEX_BONUS}\\
\overbrace{\Phi}^\text{ER} \nonumber\\
ER\nonumber
\end{align}

Dado que la Expresión Regular \ref{EQ:REGEX_BONUS} es una expresión regular válida:
\begin{align}
\#operadores &\leq \#par\'{e}ntesis + \#carácteres\_de\_\Sigma \nonumber\\
0 &\leq 0 + 0 \nonumber\\
0 &\leq 0 \label{EQ:REGEX_BONUS_1}
\end{align}

Por lo cual, usando \ref{EQ:REGEX_BONUS_1}, podemos afirmar que la Expresión Regular \ref{EQ:REGEX_BONUS} tiene la misma cantidad de paréntesis que de elementos de $\Sigma$.