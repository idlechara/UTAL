\documentclass[12pt]{utalcaDoc}

\usepackage{fullpage}
\usepackage[activeacute,spanish]{babel}
\usepackage{graphicx}
\usepackage{ae}

\title{Prueba Unidad 1, parte 1}
\author{Profesor: Rodrigo Paredes. Ayudante: Manuel Hoffhein}
\date{20 de octubre de 2015\\
2:00 horas.}


\newcommand{\Pauta}[1]{\subsection*{Respuesta}#1}
\renewcommand{\Pauta}[1]{}

%\usepackage{algo}
 
\begin{document}

\maketitle

Esta prueba luego se promedia con la prueba de la parte 2 de la unidad (algoritmos
inductivos) para calcular la nota de la prueba de la Unidad 1.

\paragraph{Pregunta 1 (2 ptos)}

Resuelva la siguiente ecuaci�n de recurrencia:
$$T(n) = 5 T\left(\frac{n}{9}\right) + \frac{1}{3} \sqrt{n}, ~~~T(1)= 1$$
Utilizando el m�todo de la telesc�pica (1.6 pts).
Revise sus resultados comprobando los primeros 4 valores de la serie (0.4 pts).

\paragraph{Pregunta 2 (2 ptos)}

Resuelva la siguiente ecuaci'on de recurrencia:
$$t(n) = 5t(n/5) + 10 n,~~t(1) = 1$$
Utilizando el m�todo del polinomio caracter�stico (1.6 pts).
Revise sus resultados comprobando los primeros 4 valores de la serie (0.4 pts).

\Pauta{

}

\paragraph{Pregunta 3 (2 ptos)}

Resuelva la siguiente ecuaci'on de recurrencia:
$$t(n) = 5t(n/5) + 10 n,~~t(1) = 1$$
Utilizando el m�todo de la funci�n generatriz (1.6 pts).
Revise sus resultados comprobando los primeros 4 valores de la serie (0.4 pts).
Note que es la misma ecuaci�n de la pregunta 2.

\Pauta{

}



\end{document}