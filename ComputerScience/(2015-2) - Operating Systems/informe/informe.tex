\documentclass[11pt]{utalcaDoc}
\usepackage{alltt}
\usepackage{underscore}
\usepackage[latin1]{inputenc}
\usepackage[activeacute,spanish]{babel}
\usepackage{verbatim}
\usepackage[pdftex]{graphicx}
\usepackage{epstopdf}
\usepackage{ae}
\usepackage{enumerate}
\usepackage{amsmath}
\usepackage{amsfonts}
\usepackage{algorithm}
\usepackage{algorithmic}
\usepackage[stable]{footmisc}
\floatname{algorithm}{Algoritmo}
\renewcommand{\listalgorithmname}{Lista de algoritmos}
\renewcommand{\algorithmicrequire}{\textbf{Precondici\'{o}n:}}
\renewcommand{\algorithmicensure}{\textbf{Postcondici\'{o}n:}}
\renewcommand{\algorithmicend}{\textbf{fin}}
\renewcommand{\algorithmicif}{\textbf{si}}
\renewcommand{\algorithmicthen}{\textbf{entonces}}
\renewcommand{\algorithmicelse}{\textbf{si no}}
\renewcommand{\algorithmicelsif}{\algorithmicelse,\ \algorithmicif}
\renewcommand{\algorithmicendif}{\algorithmicend\ \algorithmicif}
\renewcommand{\algorithmicfor}{\textbf{para}}
\renewcommand{\algorithmicforall}{\textbf{para todo}}
\renewcommand{\algorithmicdo}{\textbf{hacer}}
\renewcommand{\algorithmicendfor}{\algorithmicend\ \algorithmicfor}
\renewcommand{\algorithmicwhile}{\textbf{mientras}}
\renewcommand{\algorithmicendwhile}{\algorithmicend\ \algorithmicwhile}
\renewcommand{\algorithmicloop}{\textbf{repetir}}
\renewcommand{\algorithmicendloop}{\algorithmicend\ \algorithmicloop}
\renewcommand{\algorithmicrepeat}{\textbf{repetir}}
\renewcommand{\algorithmicuntil}{\textbf{hasta que}}
\renewcommand{\algorithmicprint}{\textbf{imprimir}} 
\renewcommand{\algorithmicreturn}{\textbf{devolver}} 
\renewcommand{\algorithmictrue}{\textbf{verdadero }} 
\renewcommand{\algorithmicfalse}{\textbf{falso }} 


\title{{\bf Sistemas Operativos}\\Tarea 1}
\author{Erik Regla\\ \texttt{eregla09@alumnos.utalca.cl}\\}
\date{11 de Agosto del 2015}

\begin{document}
\renewcommand{\figurename}{Figura~}
\renewcommand{\tablename}{Tabla~}

\maketitle

\section{Investigue sobre los siguientes aspectos de hardware que debe conocer a cabalidad}
\begin{enumerate}[a)]
	\item Componentes b�sicos de un sistema computacional (Arquitectura de Von Neumman)\\
	\emph{R: } Unidad de procesamiento central dividida en unidad aritmetico-l�gica y registros de procesador, unidad de control con registro de instrucciones y contador de programa y memoria principal.
	
	\item Ciclo de ejecuci�n de las instrucciones
	\begin{enumerate}[1.-]
		\item \emph{Cargar instruccion}: La siguiente instruccion pasa del contador de programa al registro de instrucciones. 
		\item \emph{Decodificar instruccion}: La instruccion previamente cargada es interpretada.
		\item \emph{Leer direcci�n de memoria}: La instruccion a ejecutar tiene un bloque de memoria objetivo, el cual debe ser leido antes de realizar cambios.
		\item \emph{Ejecutar instruccion}: La instruccion se ejecuta.
		\item repetir.
	\end{enumerate}
	
	\item Interrupciones, �qu� son? �por qu� son �tiles, en el contexto de la E/S? �Cuales tipos existen?\\
	\emph{R: } Las interrupciones son se�ales que indican que ha ocurrido un evento que requiere atenci�n inmediata. Son especialmente �tiles en microcontroladores para ejecutar fragmentos cortos de c�digo cuando este se sabe que el evento que lo desencadena es recurrente. Existen interrupciones por hardware que utilizan alertas electr�nicas e interrupciones por software que tienden a ser mensajes que el mismo sistema operativo env�a usualmente para controlar errores.
	
	\item Jerarqu�a de memoria, principio de localidad de referencia\\
	\emph{R: } T�rmino utilizado para discutir materias de desempe�o en memoria. Esta var�a de procesador en procesador. El principio de localidad de referencia indica que si $M$ es un arreglo unidimensional y $\mathbb{P}_i$ la probabilidad de leer un bloque de memoria $i$ en $M$, entonces,  $\mathbb{P}_{i+1} > \mathbb{P}_{i+x}, \forall x \in [0..|M|]-\{i+1\}$
	
	\item Acceso directo a memoria (DMA)\\
	\emph{R: } Caracter�stica que permite a dispositivos acceder a direcciones de memoria en RAM directamente sin pasar por el procesador.
	
	\item Arquitecturas SMP, solo la organizaci�n incluidos los cach�s\\
	\emph{R: } Cada unidad de procesamiento posee un acceso directo a memoria utilizando el bus de sistema, sin embargo, la cach� es individual para cada uno de estos. La cach� por motivos de coherencia est� permamentemente monitoreando la actividad del bus. La entrada y salida junto con la RAM tambi�n van directamente conectadas al bus de sistema, siendo este controlado por su propio bus.

	 \item computadores multicore, diferencia con SMP convencional\\
	 \emph{R: } La organizaci�n de la cach� es diferente, siendo esta dividida en diferentes niveles de acuerdo a su tama�o y nivel de acceso compartido el cual no existe en SMP.
	\end{enumerate}
\end{document}