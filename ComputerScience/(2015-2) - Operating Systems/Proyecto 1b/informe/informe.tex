\documentclass[11pt]{utalcaDoc}
\usepackage{alltt}
\usepackage{underscore}
\usepackage[latin1]{inputenc}
\usepackage[activeacute,spanish]{babel}
\usepackage{verbatim}
\usepackage[pdftex]{graphicx}
\usepackage{epstopdf}
\usepackage{ae}
\usepackage{enumerate}
\usepackage{amsmath}
\usepackage{amsfonts}
\usepackage{algorithm}
\usepackage{algorithmic}
\usepackage[stable]{footmisc}
\floatname{algorithm}{Algoritmo}
\renewcommand{\listalgorithmname}{Lista de algoritmos}
\renewcommand{\algorithmicrequire}{\textbf{Precondici\'{o}n:}}
\renewcommand{\algorithmicensure}{\textbf{Postcondici\'{o}n:}}
\renewcommand{\algorithmicend}{\textbf{fin}}
\renewcommand{\algorithmicif}{\textbf{si}}
\renewcommand{\algorithmicthen}{\textbf{entonces}}
\renewcommand{\algorithmicelse}{\textbf{si no}}
\renewcommand{\algorithmicelsif}{\algorithmicelse,\ \algorithmicif}
\renewcommand{\algorithmicendif}{\algorithmicend\ \algorithmicif}
\renewcommand{\algorithmicfor}{\textbf{para}}
\renewcommand{\algorithmicforall}{\textbf{para todo}}
\renewcommand{\algorithmicdo}{\textbf{hacer}}
\renewcommand{\algorithmicendfor}{\algorithmicend\ \algorithmicfor}
\renewcommand{\algorithmicwhile}{\textbf{mientras}}
\renewcommand{\algorithmicendwhile}{\algorithmicend\ \algorithmicwhile}
\renewcommand{\algorithmicloop}{\textbf{repetir}}
\renewcommand{\algorithmicendloop}{\algorithmicend\ \algorithmicloop}
\renewcommand{\algorithmicrepeat}{\textbf{repetir}}
\renewcommand{\algorithmicuntil}{\textbf{hasta que}}
\renewcommand{\algorithmicprint}{\textbf{imprimir}} 
\renewcommand{\algorithmicreturn}{\textbf{devolver}} 
\renewcommand{\algorithmictrue}{\textbf{verdadero }} 
\renewcommand{\algorithmicfalse}{\textbf{falso }} 


\title{{\bf Sistemas Operativos}\\Proyecto 1b}
\author{Erik Regla\\ \texttt{eregla09@alumnos.utalca.cl}\\}
\date{24 de Septiembre del 2015}

\begin{document}
\renewcommand{\figurename}{Figura~}
\renewcommand{\tablename}{Tabla~}

\maketitle

\section{Tablas}
\begin{table}[h]
\begin{tabular}{|l|l|l|l|l|}
\hline
Par�metro medido                & OSP1                                                                                        & OSP2                                                                                     & OSP3                                                             & OSP4                                                             \\ \hline
CPU Utilization                 & 89.974\%                                                                                    & 96.994\%                                                                                 & 80.278\%                                                         & 17.29\%                                                          \\ \hline
Average service time per thread & 19702.97                                                                                    & 3086.6335                                                                                & 15435.1875                                                       & 34413.3                                                          \\ \hline
Total number of tasks           & 4                                                                                           & 2                                                                                        & 6                                                                & 5                                                                \\ \hline
Threads summary                 & \begin{tabular}[c]{@{}l@{}}18 alive:\\ 1 running, \\ 12 suspended, \\ 5 ready)\end{tabular} & \begin{tabular}[c]{@{}l@{}}3 alive:\\ 1 running, \\ 0 suspended, \\ 2 ready\end{tabular} & \begin{tabular}[c]{@{}l@{}}20 alive:\\ 20 suspended\end{tabular} & \begin{tabular}[c]{@{}l@{}}14 alive:\\ 14 suspended\end{tabular} \\ \hline
\end{tabular}
\end{table}

\section{Comparaci�n}
La diferencia ente los distintos archivos de configuraci�n es �nicamente el par�metro ``ThreadLifeExpectancy'' siendo 5000, 500, y 2500 para params1, params2 y params3 respectivamente. (Autoexplicativo, lo que se espera que posea como tiempo de vida de cada hilo).

\section{params4}
El cambio m�s notorio al cambiar el par�metro ``memory events'' y ``resource events'' que ajustan el donde ocurren dichos eventos durante la simulaci�n (se  junto con la frecuencia a la cual los eventos ocurren. La mayor cantidad de eventos generados fueron efectivamente para memoria sin embargo, dado que muchas operaciones eran de asignaci�n de recursos, los hilos estaban mucho tiempo esperando, lo cual llev� a deadlock en algunos casos cuando se desactiva el detector (se puede notar por la cantidad de hilos esperando y por que la cantidad de invocaciones de petici�n de recursos son mucho m�s que las de liberaci�n, junto con la tabla de alocaci�n y de espera).
\end{document}