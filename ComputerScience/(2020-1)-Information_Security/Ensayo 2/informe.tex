\documentclass[11pt]{utalcaDoc}
\usepackage{alltt}
\usepackage{underscore}
\usepackage[utf8]{inputenc}
\usepackage[activeacute,spanish]{babel}
\usepackage{verbatim}
\usepackage[pdftex]{graphicx}
\usepackage{ae}
\usepackage{amsmath}
\usepackage{amsfonts}
\usepackage{pdflscape}
\usepackage{inconsolata}
\usepackage{url}
\usepackage{CJKutf8}
\usepackage{hyperref}
\usepackage{listings}
% \usepackage{placeins}
\usepackage[section]{placeins}
\usepackage[stable]{footmisc}


\title{{\bf Seguridad Informática}\\ Ensayo 2 \\\textit{El anonimato no es un derecho}}
\author{Erik Regla\\ eregla09@alumnos.utalca.cl}
\date{\today}
\lstset{language=SH, 
		basicstyle=\ttfamily\tiny, 
		showspaces=false, 
		numbers=left, 
		breaklines=true,
		frame=shadowbox
		}

\begin{document}
\maketitle

%Qué es el ciber acoso?
El acoso -como tal- siempre ha existido. Da lo mismo en el momento de la historia que te encuentres, siempre existen personas que desean simplemente desde exponer su punto de vista -a la fuerza- como también personas que solo intentan reafirmar su seguridad en si mismos ejerciendo violencia sobre otras personas. Pero el acoso, el acto de hostigar a una persona siempre ha estado presente en nuestra sociedad y en la naturaleza.

Pero reduzcamos el alcance del término y centrémonos en el \textit{ciberacoso}, ahora término bastante mal utilizado ya que en realidad para efectos de de este fundo con departamentos llamado Chile, es una adaptación del término \textit{cyberbullying}, el cual se restringe casi completamente al acoso escolar pero en su formato virtual.

En materia legal, el año pasado se anunció un proyecto de ley que modifica las penas y las configuraciones de los delitos que clasifican como ciberacoso, para extender su definición a lo que normalmente conocemos como acoso. Esta ley, de origen en el año 2009 en el contexto de la creciente preocupación por estudiantes que estaban siendo agredidos no solo físicamente si no también hostigados mediante mecanismos digitales por sus pares dio origen a la ley Nro. 20.370\footnote{https://especial.mineduc.cl/wp-content/uploads/sites/31/2018/04/LEY-20370_12-SEP-2009.pdf}, con tal de poder frenar estos intentos. Sin embargo, esta nueva modificación a la ley anteriormente mencionada extiende la configuración y tipificación del ciberacoso a un contexto mas general, en las cuales se consideran el hostigamiento (61 días a 3 años de cárcel), exhibición no consentida de registros de contenido sexual (hasta 5 años de cárcel) y difusión no consentida de imágenes o datos personales (hasta 541 días de cárcel).

Como contexto, de punto de vista del netizen\footnote{https://es.wikipedia.org/wiki/Netizen} siempre han existido puntos de encuentro en la red. Canales de IRC, foros, \textit{warez}, luego de la llegada de internet al país y su creciente disponibilidad las personas lentamente comenzaron no solo a adquirir una identidad física si no también a crear lo que conocemos como la huella digital -digital en todo sentido de la palabra-.\footnote{https://blog.mozilla.org/internetcitizen/2018/07/26/this-is-your-digital-fingerprint/}.

Dentro de todas estas comunidades había una en específico `nido.org' en la cual para el netizen fue siempre como un antro de perdición. Cualquier persona perteneciente a otras comunidades sabia que nada bueno sale de ese lugar, digamoslo de otra forma como que a muchos le faltaban un par de palos para el puente. Basado en el formato de \begin{CJK}{UTF8}{min}ふたば(双葉)☆ちゃんねる\end{CJK} (FutabaChannel o 2ch), un BBS (tabloide de noticias) e IB (imageboard), al igual que 4chan (la versión fruna de 2ch hecha por estadounidenses), este mostraba los mismos problemas que su origen.

Ahora este no era el único BBS en Chile, Latinoamérica o el mundo, pero al final todos se basan en el concepto fundamental de la privacidad. Los BBS y los IB a diferencia de los foros en línea tradicionales no necesitan registro previo de su usuario, cualquier persona puede llegar y participar sin necesidad de exponer información personal. Por esto mismo muchas comunidades tienen reglas internas, algunas de manera explicita y otras de forma implícita respecto al código de conducta, porque si bien tu identidad está protegida, no quiere decir que no sea moderado. Prevenir usos indebidos si bien no se almacena información personal del usuario, si se tienen registros de IP para poder ejecutar \textit{baneos},investigar incidentes en caso de ser necesario u otros fines.

En estos casos, la moderación por lo general era llevada a cabo por la misma comunidad, no solo en materia de conducta si no también de contenido ya que los BBS no necesariamente pueden ser temáticos aunque si pueden tener sub-tabloides entre medio. Lo más cercano a este formato que es consumido de manera popular en estos días es Reddit, pero claro, no respeta el principio de la anonimidad.

Y bueno, Chile tiene una larga tradición por ser un país reactivo en muchas materias y el caso de esta modificación a la ley no fue la excepción. El principal gatillante de estas modificaciones esto fue el caso de acoso en `nido.org' ocurrido durante el año 2018 -pero para quienes han pasado más tiempo frente a un computador que con personas sabemos que se arrastra desde hace tiempo pero nadie legalmente podía hacer algo al respecto.

En ese incidente aquel IB salió a la luz debido a la gran cantidad de material sexual digital de origen no consentido que albergaba. En la red se suele dar que hasta que no se llega a un lugar, nadie sabe de su existencia, pero la eventual denuncia de una persona logró sacar a ese lugar de su anonimato y adicionalmente personas fuera de aquel circulo comenzaron una caza de brujas en la cual terminaron por alcanzar en el fuego cruzado a otras que nada de relación tenían con ellos.

Yo en lo personal soy parte de una comunidad en línea hace mas de 10 años (y que la verdad es bastante chica) en la cual contamos con BBS y IB, por lo que sabemos que el formato es especialmente llamativo para algunas personas. Sin embargo a nosotros cuando comenzó la crisis de nido.org, esta nos comenzó a golpear de manera indirecta. Nuestra comunidad tiene sus normas y su conducta, desde sus origenes (hace mas de 10 años) hemos tenido una postura de respeto a toda persona que llega, de no discriminar a nadie, sin embargo siendo para ese momento de los pocos BBS restantes en el formato sufrimos una avalancha de personas nuevas. Nosotros sabíamos que \textit{eran del nido}.

De la misma manera, comenzó una avalancha de estigmatización contra el formato, porque bueno, el formato de la comunidad ha estado prácticamente intacto desde hace 10 años, lo cual se volvió complicado lidiar con gente que solo venía a intentar generar daño dentro de la comunidad por un evento completamente ajeno a nosotros.

Dentro de las mismas deliberaciones de la comunidad se adoptó la postura de responder a las preguntas de todos, porque claro muchas personas llegaban creyendo que teníamos algo que ver con aquel infame sitio pero luego de muchos threads se daban cuenta que en realidad solo tenían en frente a un grupo de ñoños que solo hablan de computadores, humanidades, política y unix todo el día y de de vez en cuando realizaban compilaciones musicales en donde uno que otro aprendía a tocar un instrumento solo para eso.

Afortunadamente esa visión es compartida dentro de la comunidad aunque también nos obligó (inconsientemente desde mi punto de vista) a dejar de ser tan blandos con las personas que comenzaron a llegar en el momento. Realmente no era el tiempo ni el lugar para mantenernos con la política de no discriminación y de que las normas de la comunidad evolucionan implícitamente con ella, ya que muchas personas estaban llegando desde nido.org son la mera intención de \textit{tomar el control} de nuestro espacio.

El espacio está definido por su comunidad y la comunidad por sus personas. Entonces ahora teníamos un conflicto dentro de los mismos principios de dieron lugar a nuestro lugar porque ya no podíamos simplemente aceptar toda actividad que comenzase a llegar. No podíamos simplemente tener una avalancha de usuarios con tales intenciones de distorsionar aquel rincón en la red. Esto provocó un cambio de la mentalidad con miras a preservar la comunidad de respetar las normas y la conducta original del sitio. No fue hasta unos cuatro meses después que la situación se normalizó y bueno volvimos a ser la misma comunidad de siempre. De hecho mirando al pasado, aquel incidente también nos supuso un gran alivio ya que la comunidad choroy (usuarios del nido) por fin dejó de ser \textit{una piedra en el zapato} luego de muchos años.

Quizás uno de los episodios más emblemáticos fue un día en que comenzaron a llegar usuarios a preguntar cosas de nido en la cual uno de los usuarios (molesto) le dijo que hiciera su pega de periodista y buscase bien. Junto con un hilo de más de 50 respuestas respecto a la historia de las comunidades digitales en Chile.

Hay más comunidades allá afuera en la red, cada una con sus normas y códigos, su temática y sus participantes por lo cual este tipo de eventos en el cual se toman en el fuego cruzado realmente es problemático. Muchas de estas cosas podrían haber sido evitadas si una ley como tal hubiera permitido frenarlas. El daño de estos eventos se propaga más allá de la misma comunidad, termina llegando a personas reales.

Pero, estas cosas tampoco ocurrirían si hubiera una base cultural, legal al mismo tiempo que una consiencia respecto a la privacidad de las personas. Esta no se limita a proteger los datos personales de cada individuo por cada uno de sus medios, también se necesita que el resto lo haga de manera proactiva. Se necesita una base en la educación que permita entender en materia digital lo que implica tener una huella en la red, siendo algo que puede ser usado para bien o mal. La protección de la privacidad de los datos personales es más importante que cualquier otra cosa hoy en día estén en la red o no.

Dentro de aquel espacio virtual llamado internet puedes ser cualquier cosa. Puedes hacer lo que esté a tu alcance y que realmente no es poco. Las personas tras una máscara se sienten con el derecho de hacer cualquier cosa sin miedo a represalias y al final del día somos animales entrenados a palos y shocks eléctricos para prevenir incurrir en conductas que atentan contra el bienestar de las demás personas. Entonces, sobre la base de que el anonimato no es lo mismo que la privacidad, ¿Tenemos realmente algun motivo de fuerza para proteger el anonimato en la red?

\end{document}