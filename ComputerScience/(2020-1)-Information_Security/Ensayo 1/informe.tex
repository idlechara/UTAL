\documentclass[11pt]{utalcaDoc}
\usepackage{alltt}
\usepackage{underscore}
\usepackage[utf8]{inputenc}
\usepackage[activeacute,spanish]{babel}
\usepackage{verbatim}
\usepackage[pdftex]{graphicx}
\usepackage{ae}
\usepackage{amsmath}
\usepackage{amsfonts}
\usepackage{pdflscape}
\usepackage{inconsolata}
\usepackage{url}
\usepackage{hyperref}
\usepackage{listings}
% \usepackage{placeins}
\usepackage[section]{placeins}
\usepackage[stable]{footmisc}


\title{{\bf Seguridad Informática}\\ Ensayo 1 \\\textit{Chile, país bananero.}}
\author{Erik Regla\\ eregla09@alumnos.utalca.cl}
\date{\today}
\lstset{language=SH, 
		basicstyle=\ttfamily\tiny, 
		showspaces=false, 
		numbers=left, 
		breaklines=true,
		frame=shadowbox
		}

\begin{document}
\maketitle

Uno de los ambientes laborales más comunes en Chile es el ambiente de la consultoría de software. Menciono común en Chile, porque debido a nuestro curriculum, en general el \textit{jardinero de software}\footnote{https://medium.com/@cmaitchison/are-you-a-software-gardener-f79eba5b7fb7} termina saliendo de planes de estudio ridiculamente largos (sobre los 6 años) con un skill-set que va lejos de se apĺicable en la realidad. No solo porque la profundidad es baja, si no también porque es demasiado amplio y en realidad, no es lo que se necesita en una industria especializada.

Siendo ese el caso, Chile, un país con una industria tecnológica en expansión, no tiene empresas especializadas y cuando lo llegan a ser, suelen ser empresas netamente tecnológicas. Entonces, claro, es una bendición tener personal que sepa un poco de todo a tener gente que se especialice en algun área en específico, siempre puedes pagar menos bajo ese pretexto. De ese punto, la consultoría de software en Chile no solo apunta a resolver problemas de los clientes locales, también hay mucha industria que resuelve problemas en clientes en el extranjero, por lo mismo, se presenta como una opción bastante interesante también del punto de vista monetario (claro, también apuntando que llegaste a un buen lugar).

Tenemos un problema grande: las empresas se proyectan típicamente como no son en realidad, no solo en el punto de vista privado. Aún recuerdo cuando en uno de mis primeros trabajos como desarrollador (en ese tiempo aún no realizaba consultoría) me tocó ir con mi jefe, el cual era un ``ingeniero comercial'' para preguntarle por los términos y condiciones de el servicio que nosotros ofrecíamos. Como contexto, lo que ofrecía la empresa era una suerte de asistencia para las empresas que necesitaban ejecutar fáctoring para así liquidar los pagos, pero claro, uno maneja información sensible de ellos en el proceso. En ese tiempo la ley general de protección de datos (GDPR) estaba recién entrando en aplicación en la Unión Europea\footnote{https://gdpr-info.eu/} a lo que recibo una respuesta del estilo ``pero toma los términos y condiciones de X empresa y adaptemoslos, total son todos iguales''.

Esta práctica que para en ese momento solo pensé que era un acto de ignorancia de quien ese entonces era mi jefe, en realidad resultó ser una práctica que luego de comenzar a trabajar como consultor me tocó ver en un sin fin de lados. Prácticamente montones de empresas chilenas hacen eso mismo al momento de generar términos y condiciones, porque claro, en especial las startups y quienes les dirigen quieren abaratar costos lo más posible y entre que ven innecesario el preocuparse de eso desde el comienzo, que ven caro el contar con un abogado que se dedique a revisar esos aspectos hasta que piensan en levantar una startup para venderla al poco tiempo después luego de que el humo esté flotando en la ciudad.

Si nos ponemos a revisar la ley chilena respecto a la protección de datos que está en una constitución de más de 30 años en la cual no hay actualización de la misma y fue escrita en un momento que con suerte se sabía que internet se podía usar para comunicar información en un laboratorio, nos damos cuenta que de manera local si desarrollamos algo para el territorio no es necesario cumplir con prácticamente ninguna ley en especial. 

Tenemos la ley 19.223 sobre delitos informáticos y sistemas de información (la cual está tan mal redactada que incluso si estas haciendo trabajos de WhiteHat la empresa te puede denunciar incluso si estás con contrato), la ley 19.955 respecto a protección de derechos del consumidor (lo único que obliga a las empresas a escribir una sección de términos y condiciones del servicio, aunque no sirva de nada) y finalmente la ley 19.628 sobre la protección de datos de carácter personal. Hay otras leyes pero que en realidad solo afectan a ciertas instituciones (como los bancos respecto a los respaldos a ejecutar), sin embargo parece ridículo que solo diga que es lo que hay que hacer, pero no establece pena alguna de cara a la empresa ya que basta con que deje explicito en algún lado oculto cuales son sus términos y condiciones como también que ''pinche en una cajita para indicar que acepta los términos y condiciones``. Haces eso y estás excento de responsabilidad. \footnote{https://www.leychile.cl/Navegar?idNorma=141599}

Una empresa pequeña está mucho mas vulnerable a esto por la poca accesibilidad a asesoria que esta puede tener. Por otro lado una empresa más grande puede mover la maquinaria necesaria para incluso si se le ``llega a pasar'' pueda salir como si nada.

Dejando el lado privado de lado, démosle una mirada a el sector público el problema no deja de estar presente. Esto es porque el sector público trabaja casi en su totalidad externalizando recursos tecnológicos, en especial realizando outsourcing de personal especializado para llevar a cabo tareas que deberían realizarse dentro. Sin embargo, la misma legislación indica que basta con que haya una sola persona supervisando el actuar de un externo y con eso es suficiente para poder tener personal externo a una organización de carácter público trabajando con datos personales. De hecho, uno puede revisar la información de un año cualquiera para ver la cantidad de audiencias que las empresas solicitan para poder trabajar con entidades estatales, como por ejemplo, el Ministerio Público \footnote{https://www.leylobby.gob.cl/instituciones/AE003/audiencias/2018/146962}.

Desde ese punto de vista, nos da igual que tengamos una ley de proteccion de datos (que hubo un intento en el 2018 en Chile también con la ley anteriormente mencionada), no hay empresas ni organizaciones de ningún sector realmente dispuestas a acatar la norma y el hecho que esté mal hecha solo hace que la situación sea peor.

Ahora, claro, uno ve este panorama dentro de nuestro país como una realidad distante, muy lejana del primer mundo donde tienen normativas sólidas al respecto que provocaron tal impacto que empresas a lo largo del mundo estuvieron obligadas a cambiar el mecanismo con el cual controlan y trabajan sus datos, pero acá en Chile al parecer nunca ocurrió nada.

Quizás uno de los aspectos más tristes es que cuando tienes que trabajar con proyectos de externos, te das cuenta que un mismo artefacto ha pasado por montones de organizaciones diferentes, cada una con sus propios fallos, en lo que al final en vez de ser una actividad en la cual tu puedes desempeñar tu trabajo bien, se transforma en pasar una pelota cada vez mas pesada con responsabilidades de un proveedor a otro. La subcontratación de consultoras entre si cuando ocurren estas cosas es un fenómeno extremadamente común, por lo que es normal ver que siendo que si bien un único proveedor es el encargado de desarrollar una solución, sus bases han pasado por muchas mas organizaciones, las cuales no necesariamente tienen que ser compilantes con las políticas de el cliente original. Debido a esto, cuando ocurre un problema, el problema se pierde en la oscuridad de encontrar donde ocurrió.

El impacto pues lo podemos tener desde tres aspectos, interno, externo y transitivo. De un punto de vista interno, consideremos empresas pequeñas que aspiran a apelar al mercado tecnológico europeo, aunque este no sea el principal. Debido a las regulaciones de aquel lugar, estas empresas están técnicamente imposibilitadas de poder ofrecer sus servicios ya que la legislación local les clasifica como ilegales. Ese fue el caso de muchos servicios de internet provistos por otros países que estaban disponibles sobre territorio europeo (como Facebook\footnote{https://gdpr.eu/the-gdpr-meets-its-first-challenge-facebook/}) las cuales simplemente no tuvieron mas opción que optar por adoptarla o bien simplemente dejar de ofrecer los servicios. Aún asi ¿Hemos visto alguna noticia tan rimbombante con una empresa chilena hasta el día de hoy?

La siguiente dimensión es la externa, ya que de la misma manera que nosotros quisieramos ofrecer servicios, hay otras empresas que también eventualmente podrían querer traer desde territorio europeo sus servicios a Chile, pero el problema es que debido a que dentro del gobierno, los bancos y cualquier otro servicio que por transitividad sea necesario utilizar dentro de nuestro territorio podría (o no) estar apegado a la norma, esta incertidumbre al respecto impide el avance de la colaboración. 

Ahora esto no elimina la posibilidad que las mismas organizaciones implementen sus propias políticas para el trabajo de los datos, de hecho, muchas organizaciones dentro de nuestro territorio que implementan productos del exterior ya vienen con esa política por nuevamente, transitividad. En ese caso, claro, no existe mayor problema porque si el servicio viene desde afuera, el problema de la seguridad está hecho, pero por otro lado, si ellos llegasen a requerir el servicio de algún local, entonces ¿Qué otra opcion les queda? Entonces otro aspecto que impacta negativamente en la industria y en la economía es la limitancia al respecto de las opciones a tomar y la oferta disponible en el lugar de destino.

Finalmente, tenemos una dimensión transitiva, ya que en el mundo actual no todas las empresas vienen de un solo lugar. Tenemos empresas de todos los lugares del mundo ofreciendo sus servicios acá, incluso algunas con filiales locales -\textit{branches}- las cuales también desempeñan funciones bajo el alero de esta gran empresa, pero que por otro lado tiene sus propias libertades. Es el caso de las automotoras, las cuales tienen sus propias unidades de desarrollo, legales, marketing, etc, diseñadas especificamente a apuntar a un público específico del lugar donde se encuentran.

Pero tomemos el ejemplo de una campaña de marketing, operación tipica en muchos negocios independientes de su escala. Durante estas campañas, procesos que van desde el descubrimiento del público objetivo, el minado de información y la persistencia de los mismos para futuras acciones por parte de estos grupos. ¿Y donde va a parar esa información? Entonces una empresa externa que venga a ofrecer sus servicios en Chile, si ya cuenta con clientes dentro de la Unión Europea, estará forzada a dos cosas. Forzar la adopción de dichas politicas de resguardo para los datos personales dentro del branch Chileno -lo cual en realidad es bastante bueno para nosotros-, pero también por transitividad impulsa a las empresas del ambiente local a adoptarlas, porque ejecutar dicha adopción les abre las puertas a nuevos lead de negocio. En realidad bajo esta dimensión todos ganamos, pero también es que es un efecto indirecto.

Irónicamente, la GDPR ha generado un boom de la \textit{consultoría exportada} por parte del territorio europeo, por una razón muy simple. Como ellos están immersos dentro del mismo territorio donde las leyes mas fuerte pegan, los servicios que ofrecen tienen que estar sujetos a dicha normativa y la consultoria tanto como sus productos per-se tienen que ser compilantes, aplicando este fenómeno no solo al ámbito de desarrollo. Debido a esto y a la rigidez de la misma, es muy facil para otros paises adoptar sus productos porque bueno, a excepción de China, no hay mas lugares con políticas así de estrictas.

Como consultor muchas veces te toca escuchar a clientes que hacen la pregunta de como  sus datos son tratados y cuando te cae la pregunta además de tener que redirigirlos con el CTO para que vieran el asunto, no es mucho lo que se puede hacer si es que no quieres meterte dentro de la parte administrativa. Diferente es el caso cuando te preguntan como alcanzar una compilancia, ya que ahi es parte de tu trabajo.

Recuerdo que una vez para poder cerrar un trato había que ser compilantes con tres normas diferentes. Ahora, la cara se me deformó cuando después de las conversaciones me dijeron que pasaron por mas de otras diez oficinas y ninguna de ellas tenia idea de que era eso o bien no sabían como garantizarla -plot twist: la reunión salió bien pero al final por temas \textit{administrativos} o cobardía decidieron no ejecutar el proyecto-.

Sin embargo, volviendo al punto principal de que en realidad Chile tiene politicas muy pobres, podríamos decir que en realidad no es ese el problema. Chile es un país bananero\footnote{https://es.wikipedia.org/wiki/República_bananera}, la industria tecnológica tiene un océano de posibilidades pero elige malas prácticas enrraigadas desde los inicios de nuestra cultura, junto con una oligarquia que principalmente reina en la industria nacional donde se toman las decisiones teniendo nulo conocimiento del contexto o bien de la materia sobre la cual deliberan.

Seguimos solo pudiendo exportar \textit{talento técnico} pero aún nos cuesta salir afuera para poder ofrecer productos y GDPR mas que un riesgo es la oportunidad que muchas industrias locales necesitan, pero que nadie está dispuesto a tomar. ¿O sea, a quién le interesaría \textit{gastar dinero} contratando un SOC o un asesor legal para poder ver como ofrecer productos fuera del territorio nacional? ¿A quien le importaría tener que dejar de utilizar un mecanismo precario laboralmente como el outsourcing para así garantizar la calidad del producto y velar por su compilancia? No, es mucha plata.

\end{document}