\documentclass[11pt]{utalcaDoc}
\usepackage{alltt}
\usepackage{underscore}
\usepackage[utf8]{inputenc}
\usepackage[activeacute,spanish]{babel}
\usepackage{verbatim}
\usepackage[pdftex]{graphicx}
\usepackage{ae}
\usepackage{amsmath}
\usepackage{amsfonts}
\usepackage{pdflscape}
\usepackage{inconsolata}
\usepackage{url}
\usepackage{hyperref}
\usepackage{listings}
\usepackage{tabularx}
\usepackage{makecell}
% \usepackage{placeins}
\usepackage[section]{placeins}
\usepackage[stable]{footmisc}
\usepackage{keyval}% http://ctan.org/pkg/keyval
\usepackage{minted}

\title{{\bf Seguridad Informática}\\ Proyecto 1}
\author{Erik Regla\\ eregla09@alumnos.utalca.cl}
\date{\today}



\newcommand{\informationResource}[9] {
    \begin{tabularx}{\linewidth}{|r|X|}
        \hline
        \textbf{Nombre}      & #1  \\ 
        \hline
        \textbf{Descripción} & #2  \\ 
        \hline
        \textbf{Categoría}   & #3  \\ 
        \hline
        \textbf{Ubicación}   & #4  \\ 
        \hline
        \textbf{Propietario} & #5  \\ 
        \hline
        \textbf{Valoración} &   Confidencialidad: #6 \\
                            &   Integridad: #7 \\
                            &   Disponibilidad: #8 \\
        \hline
        \textbf{Vulnerabilidades y Amenazas}  &  #9 \\ 
        \hline
    \end{tabularx}
}
    
    %% For offices
    \newcommand{\repetatOfficeInformationResource}[9]{
        \informationResource{#1\_001}{#2}{#3}{#4}{Jefe de #5}
        {#6}{#7}{#8}{#9}\\
    
        \informationResource{#1\_002}{#2}{#3}{#4}{Secretario de #5}
        {#6}{#7}{#8}{#9}\\
    
        \informationResource{#1\_003}{#2}{#3}{#4}{Ejecutivo de #5}
        {#6}{#7}{#8}{#9}
    }
    %% For depart
    \newcommand{\repeatDepartmentInformationResource}[9]{
        \informationResource{#1\_001}{#2}{#3}{#4}{Jefe de #5}
        {#6}{#7}{#8}{#9}\\
    
        \informationResource{#1\_002}{#2}{#3}{#4}{Secretario de #5}
        {#6}{#7}{#8}{#9}\\
    
        \informationResource{#1\_003}{#2}{#3}{#4}{Ejecutivo de #5}
        {#6}{#7}{#8}{#9}

        \informationResource{#1\_003}{#2}{#3}{#4}{Encargado de TI de #5}
        {#6}{#7}{#8}{#9}
    }

    %% For secretary
    \newcommand{\repeatSecretaryInformationResource}[9]{
        \informationResource{#1\_101}{#2}{#3}{#4}{Dirección de #5}
        {#6}{#7}{#8}{#9}\\

        \informationResource{#1\_101}{#2}{#3}{#4}{Secretario de #5}
        {#6}{#7}{#8}{#9}\\
    
        \informationResource{#1\_201}{#2}{#3}{#4}{Ejecutivo de #5}
        {#6}{#7}{#8}{#9}
    }
    

\newcommand{\repeatOnEachOfficeBase}[3]{
	\repetatOfficeInformationResource
	{NTB\_#1}
	{Thinkpad T490 series, equipo corporativo}
	{Hardware TI}
	{#2 - primer piso - recurso estático}
	{#2}
	{4}{5}{3}
	{
		\threatInformationBreach
		\threatResourceLost
		\threatHumanIntervention
		\threatRemoteIntervention
		\threatNaturalDisaster
		\threatHumanDisaster
	}

	\repetatOfficeInformationResource
	{EML\_#1}
	{Cuenta de correo corporativa}
	{Activo de información tangible}
	{EXE\_EXCHA\_001}
	{#2}
	{4}{5}{3}
	{
		\threatInformationBreach
		\threatRemoteIntervention
		\threatTransitive
	}

	\informationResource
	{NAS\_#1\_001}
	{Cisco NSS324 NAS, NAS local para equipo de la #2}
	{Hardware TI}
	{#2 - primer piso - recurso estático}
	{Departamento de TI - Encargado TI de #3}
	{4}{5}{3}
	{
		\threatCVE{CVE-2017-7494}{Ejecución remota de código}
		\threatInformationBreach
		\threatResourceLost
		\threatRemoteIntervention
		\threatNaturalDisaster
		\threatHumanDisaster
		\threatInterest
	}

	\informationResource
	{CAB\_#1\_001}
	{Armario de archivos para #2}
	{Infraestructura TI}
	{#2 - primer piso - recurso estático}
	{Jefe de #2}
	{5}{5}{5}
	{
		\threatNaturalDisaster
		\threatHumanDisaster
		\threatInterest
	}

	\informationResource
	{OFI\_#1\_001}
	{#2 - Instancia física}
	{Infraestructura TI}
	{#2 - primer piso - recurso estático}
	{Administración del edificio}
	{5}{5}{5}
	{
		\threatNaturalDisaster
		\threatHumanDisaster
	}

	\informationResource
	{RNG\_#1\_001}
	{Alarma de #2}
	{Control de entorno}
	{#2 - primer piso - recurso estático}
	{Administración del edificio}
	{5}{5}{5}
	{
		\threatHumanIntervention
		\threatRemoteIntervention
	}


	\informationResource
	{ARC\_#1\_001}
	{Archivo de #2 - registro de documentos}
	{Activos tangibles / Activos intangibles}
	{#2 - primer piso}
	{Jefe de #2}
	{5}{5}{4}
	{
		\threatNoPhysicalBackup
		\threatNoDigitalBackup
		\threatHumanIntervention
		\threatValue
	}
}
\newcommand{\repeatOnEachSecretaryBase}[3]{
	\repeatSecretaryInformationResource
	{NTB\_#1}
	{Thinkpad T490 series, equipo corporativo}
	{Hardware TI}
	{#2 - primer piso - recurso estático}
	{#2}
	{4}{5}{3}
	{
		\threatInformationBreach \\ &
		\threatResourceLost \\ &
		\threatHumanIntervention \\ &
		\threatNaturalDisaster \\ &
		\threatHumanDisaster
	}

	\repeatSecretaryInformationResource
	{EML\_#1}
	{Cuenta de correo corporativa}
	{Activo de información tangible}
	{EXE\_EXCHA\_001}
	{#2}
	{4}{5}{3}
	{
		\threatInformationBreach \\ &
		\threatHumanIntervention \\ &
		\threatTransitive \\ &
		\riskNameRolesNoDefinidos
		\riskNameQuiebreAutenticacionDeLlaveSeguridad \\ &
		\riskNameTransaccionRota \\ &
		\riskNamePhishing \\ &
	}

	\informationResource
	{NAS\_#1\_001}
	{Cisco NSS324 NAS, NAS local para equipo de la #2}
	{Hardware TI}
	{#2 - primer piso - recurso estático}
	{Departamento de TI - Encargado TI de #3}
	{4}{5}{3}
	{
		\threatCVE{CVE-2017-7494}{Ejecución remota de código} \\ &
		\threatInformationBreach \\ &
		\threatResourceLost \\ &
		\threatHumanIntervention \\ &
		\threatNaturalDisaster \\ &
		\threatHumanDisaster
	}

	\informationResource
	{CAB\_#1\_001}
	{Armario de archivos para #2}
	{Infraestructura TI}
	{#2 - primer piso - recurso estático}
	{Jefe de #2}
	{5}{5}{5}
	{
		\threatNaturalDisaster \\ &
		\threatHumanDisaster
	}

	\informationResource
	{OFI\_#1\_001}
	{#2 - Instancia física}
	{Infraestructura TI}
	{#2 - primer piso - recurso estático}
	{Administración del edificio}
	{5}{5}{5}
	{
		\threatNaturalDisaster \\ &
		\threatHumanDisaster
	}

	\informationResource
	{RNG\_#1\_001}
	{Alarma de #2}
	{Control de entorno}
	{#2 - primer piso - recurso estático}
	{Administración del edificio}
	{5}{5}{5}
	{
		\threatHumanIntervention
	}


	\informationResource
	{ARC\_#1\_001}
	{Archivo de #2 - registro de documentos}
	{Activos tangibles / Activos intangibles}
	{#2 - primer piso}
	{Jefe de #2}
	{5}{5}{4}
	{
		\threatNoPhysicalBackup \\ &
		\threatNoDigitalBackup \\ &
		\threatHumanIntervention
	}
}
\newcommand{\repeatOnEachDepartmentBase}[3]{
	\informationResource
	{SWI\_#1\_0001}
	{Switch general Cisco Catalyst 2960 para específico del departamento}
	{Hardware TI}
	{#2 - #3}
	{Departamento de TI - Encargado TI de #2}
	{1}{5}{5}
	{
		\threatCVE{CVE-2017-3881}{Ejecución arbitraria de código (resuelto)} \\ &
		\threatResourceLost \\ &
		\threatHumanIntervention \\ &
		\threatNaturalDisaster \\ &
		\threatHumanDisaster
	}


	\repeatDepartmentInformationResource
	{NTB\_#1}
	{Thinkpad T490 series, equipo corporativo}
	{Hardware TI}
	{#2 - #3 - recurso estático}
	{#2}
	{4}{5}{3}
	{
		\threatInformationBreach \\ &
		\threatResourceLost \\ &
		\threatHumanIntervention \\ &
		\threatNaturalDisaster \\ &
		\threatHumanDisaster
	}

	\repeatDepartmentInformationResource
	{EML\_#1}
	{Cuenta de correo corporativa}
	{Activo de información tangible}
	{EXE\_EXCHA\_001}
	{#2}
	{4}{5}{3}
	{
		\threatInformationBreach \\ &
		\threatHumanIntervention \\ &
		\threatTransitive \\ &
		\riskNameRolesNoDefinidos
		\riskNameQuiebreAutenticacionDeLlaveSeguridad \\ &
		\riskNameTransaccionRota \\ &
		\riskNamePhishing \\ &
	}

	\informationResource
	{NAS\_#1\_001}
	{Cisco NSS324 NAS, NAS local para equipo de la #2}
	{Hardware TI}
	{#2 - #3 - recurso estático}
	{Departamento de TI - Encargado TI de #2}
	{4}{5}{3}
	{
		\threatCVE{CVE-2017-7494}{Ejecución remota de código} \\ &
		\threatInformationBreach \\ &
		\threatResourceLost \\ &
		\threatHumanIntervention \\ &
		\threatNaturalDisaster \\ &
		\threatHumanDisaster
	}

	\informationResource
	{CAB\_#1\_001}
	{Armario de archivos para #2}
	{Infraestructura TI}
	{#2 - #3 - recurso estático}
	{Jefe de #2}
	{5}{5}{5}
	{
		\threatNaturalDisaster \\ &
		\threatHumanDisaster \\ &
		\riskNameRecuperacionDesastres
	}

	\informationResource
	{OFI\_#1\_001}
	{#2 - Instancia física}
	{Infraestructura TI}
	{#2 - #3 - recurso estático}
	{Administración del edificio}
	{5}{5}{5}
	{
		\threatNaturalDisaster \\ &
		\threatHumanDisaster
	}

	\informationResource
	{RNG\_#1\_001}
	{Alarma de #2}
	{Control de entorno}
	{#2 - #3 - recurso estático}
	{Administración del edificio}
	{5}{5}{5}
	{
		\threatHumanIntervention
	}


	\informationResource
	{ARC\_#1\_001}
	{Archivo de #2 - registro de documentos}
	{Activos tangibles / Activos intangibles}
	{#2 - #3}
	{Jefe de #2}
	{5}{5}{4}
	{
		\threatNoPhysicalBackup \\ &
		\threatNoDigitalBackup \\ &
		\threatHumanIntervention
	}
}

\makeatletter
\define@key{phpappbase}{appdefinition}{\def\php@appdefinition{#1}}
\define@key{phpappbase}{apprefr}{\def\php@apprefr{#1}}
\define@key{phpappbase}{srvrefr}{\def\php@srvrefr{#1}}
\define@key{phpappbase}{dbsrvrefr}{\def\php@dbsrvrefr{#1}}
\define@key{phpappbase}{dbsrvrefconfi}{\def\php@dbsrvrefconfi{#1}}
\define@key{phpappbase}{dbsrvrefinteg}{\def\php@dbsrvrefinteg{#1}}
\define@key{phpappbase}{dbsrvrefavail}{\def\php@dbsrvrefavail{#1}}
\define@key{phpappbase}{dbinstrefr}{\def\php@dbinstrefr{#1}}
\define@key{phpappbase}{dbinstrefconfi}{\def\php@dbinstrefconfi{#1}}
\define@key{phpappbase}{dbinstrefinteg}{\def\php@dbinstrefinteg{#1}}
\define@key{phpappbase}{dbinstrefavail}{\def\php@dbinstrefavail{#1}}
\define@key{phpappbase}{dbinstrefcve}{\def\php@dbinstrefcve{#1}}
\define@key{phpappbase}{phpinstrefr}{\def\php@phpinstrefr{#1}}
\define@key{phpappbase}{phpinstrefconfi}{\def\php@phpinstrefconfi{#1}}
\define@key{phpappbase}{phpinstrefinteg}{\def\php@phpinstrefinteg{#1}}
\define@key{phpappbase}{phpinstrefavail}{\def\php@phpinstrefavail{#1}}

\setkeys{phpappbase}{
appdefinition=appdefinition,
apprefr=apprefr,
srvrefr=srvrefr,
dbsrvrefr=dbsrvrefr,
dbsrvrefconfi=dbsrvrefconfi,
dbsrvrefinteg=dbsrvrefinteg,
dbsrvrefavail=dbsrvrefavail,
dbinstrefr=dbinstrefr,
dbinstrefconfi=dbinstrefconfi,
dbinstrefinteg=dbinstrefinteg,
dbinstrefavail=dbinstrefavail,
dbinstrefcve=dbinstrefcve,
phpinstrefr=phpinstrefr,
phpinstrefconfi=phpinstrefconfi,
phpinstrefinteg=phpinstrefinteg,
phpinstrefavail=phpinstrefavail
}

\newcommand{\phpApplicationBaseStructure}[2][]{%
  \setkeys{phpappbase}{#1}% Set the keys

	\php@appdefinition

	\informationResource
	{\php@dbsrvrefr}
	{Servidor MySQL 6.0.9 Beta3 para \php@apprefr}
	{Software}
	{\php@srvrefr}
	{Departamento de TI}
	{\php@dbsrvrefconfi}{\php@dbsrvrefinteg}{\php@dbsrvrefavail}
	{
		\threatCVE{CVE-2009-0819}{Denegación de servicio}
		\threatCVE{CVE-2008-7247}{Bypass de restricciones RBAC}
		\threatHumanIntervention
		\threatRemoteIntervention
	}

	\informationResource
	{\php@dbinstrefr}
	{Base de datos MySQL en \php@dbsrvrefr para \php@apprefr}
	{Software}
	{\php@srvrefr}
	{Departamento de TI}
	{\php@dbinstrefconfi}{\php@dbinstrefinteg}{\php@dbinstrefavail}
	{
		\php@dbinstrefcve
	}
	
	\informationResource
	{\php@phpinstrefr}
	{Servidor PHP 7.3.6 para \php@apprefr}
	{Software}
	{\php@srvrefr}
	{Departamento de TI}
	{\php@phpinstrefconfi}{\php@phpinstrefinteg}{\php@phpinstrefavail}
	{
		\threatCVE{CVE-2019-11042}{Buffer overflow causado por información EXIF}
		\threatCVE{CVE-2008-7247}{Buffer overflow causado por información EXIF}
		\threatHumanIntervention
		\threatRemoteIntervention
	}

  #2
}

\makeatother

% {EXE\_WPRES\_001}{SRV_SHARE_001}
% {EXE\_MYSQL\_001}{1}{3}{3} %7
% {EXE\_MYSDB\_001}{1}{3}{3}{
% 	\threatTransitive
% 	\threatValue
% }%12
% {EXE\_PHPSR\_001}{1}{3}{3}

%
% \newcommand{\phpApplicationBaseStructure}[16]{

	
% }
\newcommand{\threatCVE}[2]{
	#1\footnote{https://www.cvedetails.com/cve/#1/}#2.\\
}
\newcommand{\threatInformationBreach}{
	Robo de información digital.\\
}\newcommand{\threatResourceLost}{
	Pérdida del equipo.\\
}\newcommand{\threatHumanIntervention}{
	Pérdida de integridad física o lógica por intervención física.\\
}\newcommand{\threatRemoteIntervention}{
	Pérdida de integridad física o lógica por intervención remota.\\
}\newcommand{\threatNaturalDisaster}{
	Puede ser sujeto de desastres de origen natural.\\
}\newcommand{\threatHumanDisaster}{
	Puede ser sujeto de desastres de origen humano.\\
}\newcommand{\threatInterest}{
	Posee información de alto interés para un grupo específico.\\
}\newcommand{\threatValue}{
	Posee un gran valor por reducción de especies.\\
}\newcommand{\threatCVEHigh}{
	Posee un gran número de vulnerabilidades CVE.\\
}\newcommand{\threatTransitive}{
	Está sujeta a vulnerabilidades de manera transitiva.\\
}




\begin{document}
\maketitle

\newpage

\section{Introducción}

\section{Estructura organizacional}
El alcance de este trabajo abarca solo las siguientes divisiones:
\begin{itemize}
	\item {
		\textbf{Departamento de asuntos municipales}. Perteneciente a la secretaría municipal. Este se compone de las siguientes oficinas:
		\begin{itemize}
			\item Oficina de partes alcaldía
			\item Sección resolutiva
			\item Sección administrativa
		\end{itemize}
	}
	\item {
		\textbf{Departamento de certificación y archivo}.  Perteneciente a la secretaría municipal. Este se compone de las siguientes oficinas:
		\begin{itemize}
			\item Oficina de registro municipal de transferencias
			\item Oficina de control de archivo y reoresentación.
		\end{itemize}
	}
	\item {
		\textbf{Departamento de asuntos concejo cosoc y otros}.  Perteneciente a la secretaría municipal. Este se compone de las siguientes secciones:
		\begin{itemize}
			\item Sección cosoc
			\item Sección consejo municipal
		\end{itemize}
	}
	\item {
		\textbf{Departamento de revisión de procesos de contratación pública}.   Perteneciente a la dirección de control.
	}
	\item {
		\textbf{Departamento de auditoría operativa}.  Perteneciente a la dirección de control.
	}
	\item {
		\textbf{Departamento Revisión de procesos de pago, bienes y servicios}.  Perteneciente a la dirección de control. 
	}
\end{itemize}

Se establece para cada departamento la siguiente estructura base:
\begin{itemize}
	\item Un(a) Jefe(a) de departamento.
	\item Un(a) Secretario(a) general de departamento.
	\item Uno o más ejecutivos de departamento.
	\item Un encargado de TI del departamento.
\end{itemize}


Se establece para cada oficina la siguiente estructura base:
\begin{itemize}
	\item Un(a) Jefe(a) de oficina.
	\item Un(a) Secretario(a) general.
	\item Uno o más ejecutivos de oficina.
\end{itemize}

Se establece para cada secretaría la siguiente estructura base:
\begin{itemize}
	\item Un(a) Secretario(a) general.
	\item Uno o más ejecutivos de oficina.
\end{itemize}

\section{Identificación de activos}
Durante la identificación de activos esta se ha limitado a activos que puedan presentar riesgos de seguridad de la información, ignorando los activos humanos y los activos de servicios de TI ya que escapan a situaciones bajo el control directo y supervisión de el equipo de TI. Adicionalmente está especificado en la especificación del proyecto que dichos factores no deben de ser incluidos.

\subsection{Activos de carácter transversal}
A continuación se listan los activos de caracter transversal, quiere decir, cuyo uso se extiende por más de una sola oficina.

\informationResource
{RTR\_PRINC\_001}
{Router principal Cisco 2901, gateway externo perteneciente a la municipalidad}
{Hardware TI}
{Sala de servidores - primer piso}
{Departamento de TI}
{1}{5}{5}
{
	\threatCVE{CVE-2013-1241}{Autenticación inválida en cabeceras del módulo ISM}
	\threatResourceLost
	\threatHumanIntervention
	\threatRemoteIntervention
	\threatNaturalDisaster
	\threatHumanDisaster
}

\informationResource
{RTR\_SECUN\_001}
{Router secundario Cisco 2901, utilizado de punto intermedio hacia la red interna}
{Hardware TI}
{Sala de servidores - primer piso}
{Departamento de TI}
{1}{5}{5}
{
	\threatCVE{CVE-2017-3881}{Ejecución arbitraria de código (resuelto)}
	\threatResourceLost
	\threatHumanIntervention
	\threatRemoteIntervention
	\threatNaturalDisaster
	\threatHumanDisaster
}

\informationResource
{SWI\_NODES\_001}
{Switch general Cisco Catalyst 2960, para nodo base del arbol de conectividad}
{Hardware TI}
{Sala de servidores - primer piso}
{Departamento de TI}
{1}{5}{5}
{
	\threatCVE{CVE-2017-3881}{Ejecución arbitraria de código (resuelto)}
	\threatResourceLost
	\threatHumanIntervention
	\threatRemoteIntervention
	\threatNaturalDisaster
	\threatHumanDisaster
}

\informationResource
{SRV\_SHARE\_001}
{Dell PowerEdge R520 750W E5 2440}
{Hardware TI}
{Sala de servidores - primer piso}
{Departamento de TI}
{5}{5}{5}
{
	\threatHumanIntervention
	\threatRemoteIntervention
	\threatNaturalDisaster
	\threatHumanDisaster
	\threatInterest
}


\informationResource
{OSS\_WINDO\_001}
{Windows Server 2019 Datacenter Edition}
{Sistemas Operativos}
{SRV_SHARE_001}
{Departamento de TI}
{2}{5}{5}
{
	Mas de 390 vulnerabilidades detectadas\footnote{https://www.cvedetails.com/product/50662/Microsoft-Windows-Server-2019.html?vendor_id=26}\\
	\threatHumanIntervention
	\threatRemoteIntervention
}


\informationResource
{EXE\_EXCHA\_001}
{Módulo servidor para Microsoft Exchange 2016, para uso de correos corporativos de los funcionarios de la municipalidad.}
{Software}
{SRV_SHARE_001}
{Departamento de TI}
{5}{5}{5}
{
	\threatCVE{CVE-2018-8374}{Tampering Vulnerability existente al momento de \\un fallo en la información de los perfiles}
	\threatCVE{CVE-2018-8302}{Ejecución de código remota debido al fallo de \\
	manipulación de objetos en memoria, resultante en control total}
	\threatCVE{CVE-2018-8159}{ XSS resultante en elevación de privilegios por \\
	medio de requests web }
	\threatCVE{CVE-2018-8154}{Ejecución de código remota debido a la corrupción \\
	del manejo de objetos en memoria, resultante en control total}
	\threatCVE{CVE-2018-8153}{ Spoofing }
	\threatCVE{CVE-2018-8152}{ Elevación de privilegios }
	\threatCVE{CVE-2018-8151}{ Corrupción de memoria }
	\threatHumanIntervention
	\threatRemoteIntervention
	\threatInterest
}

\informationResource
{ARC\_LOCAL\_001}
{Archivo general de la municipalidad - registro de documentos}
{Activos tangibles / Activos intangibles}
{Archivo - Primer piso}
{Departamento de Certificación y Archivos}
{5}{4}{2}
{
	\threatNoPhysicalBackup
	\threatNoDigitalBackup
	\threatHumanIntervention
}

\phpApplicationBaseStructure[appdefinition={
	\informationResource
	{EXE\_WPRES\_001}
	{Servidor Wordpress 5.1 Beta3 para página institucional}
	{Software}
	{SRV\_SHARE\_001}
	{Departamento de TI}
	{1}{3}{3}
	{
		\threatCVE{CVE-2019-9787}{Ejecución remota de código por medio de CRSRF}
		\threatCVE{CVE-2018-8302}{Sanitización de wp\_validate manipula redirects}
		\threatHumanIntervention
		\threatRemoteIntervention
	}
},
apprefr=EXE\_WPRES\_001,
srvrefr=SRV\_SHARE\_001,
dbsrvrefr=EXE\_MYSQL_001,
dbsrvrefconfi=1,
dbsrvrefinteg=3,
dbsrvrefavail=3,
dbinstrefr=EXE\_SQLTB_001,
dbinstrefconfi=1,
dbinstrefinteg=3,
dbinstrefavail=3,
dbinstrefcve={
	\threatTransitive
	\threatInterest
},
phpinstrefr=EXE\_PHPSR_001,
phpinstrefconfi=1,
phpinstrefinteg=3,
phpinstrefavail=3]



\phpApplicationBaseStructure[appdefinition={
	\informationResource
	{EXE\_ADMIN\_002}
	{Servidor con aplicativo de administración propia para municipio}
	{Software}
	{SRV\_SHARE\_002}
	{Departamento de TI}
	{5}{5}{5}
	{
		\threatUnkown
		\threatTransitive
		\threatInterest
	}
},
apprefr=EXE\_ADMIN\_002,
srvrefr=SRV\_SHARE\_002,
dbsrvrefr=EXE\_MYSQL_002,
dbsrvrefconfi=4,
dbsrvrefinteg=5,
dbsrvrefavail=4,
dbinstrefr=EXE\_SQLTB_002,
dbinstrefconfi=5,
dbinstrefinteg=5,
dbinstrefavail=5,
dbinstrefcve={
	\threatTransitive
	\threatInterest
},
phpinstrefr=EXE\_PHPSR_002,
phpinstrefconfi=4,
phpinstrefinteg=5,
phpinstrefavail=4]


\phpApplicationBaseStructure[appdefinition={
	\informationResource
	{EXE\_ADMIN\_003}
	{Servidor con aplicativo de administración para archivo de municipio}
	{Software}
	{SRV\_SHARE\_003}
	{Departamento de TI}
	{5}{5}{5}
	{
		\threatUnkown
		\threatTransitive
		\threatInterest
	}
},
apprefr=EXE\_ADMIN\_003,
srvrefr=SRV\_SHARE\_003,
dbsrvrefr=EXE\_MYSQL_003,
dbsrvrefconfi=4,
dbsrvrefinteg=5,
dbsrvrefavail=4,
dbinstrefr=EXE\_SQLTB_003,
dbinstrefconfi=5,
dbinstrefinteg=5,
dbinstrefavail=5,
dbinstrefcve={
	\threatTransitive
	\threatInterest
},
phpinstrefr=EXE\_PHPSR_003,
phpinstrefconfi=4,
phpinstrefinteg=5,
phpinstrefavail=4]



\subsection{Activos de carácter específico}
A continuación se listan los activos de caracter específico, quiere decir, cuyo uso es solo de un oficina, departamento o sección en particular.

\repeatOnEachOfficeBase{OF001}{Oficina de Partes Alcaldía}{ Departamento de Asuntos Municipales}

\repeatOnEachOfficeBase{OF002}{Oficina de Registro Municipal de Transferencias}{ Departamento de Certificación y Archivo}

\repeatOnEachOfficeBase{OF003}{Oficina de Control de Archivo y reorsentación}{ Departamento de certificación y archivo}

\repeatOnEachSecretaryBase{SE001}{Sección Resolutiva}{Departamento de Asuntos Municipales}

\repeatOnEachSecretaryBase{SE002}{Sección Administrativa}{Departamento de Asuntos Municipales}

\repeatOnEachDepartmentBase{DP001}{Departamento de Asuntos Municipales}{primer piso}

\repeatOnEachDepartmentBase{DP002}{Departamento de Certificación y Archivo}{segundo piso}

\repeatOnEachDepartmentBase{DP003}{Departamento de Asuntos Concejo Cosoc y Otros.}{tercer piso}

\repeatOnEachDepartmentBase{DP004}{Departamento de Revisión de Procesos de Contratación Pública}{cuarto piso}

\repeatOnEachDepartmentBase{DP005}{Departamento de Auditoría Operativa} {quinto piso}

\repeatOnEachDepartmentBase{DP006}{Departamento de Revisión de Pagos de Bienes y Servicios}{sexto piso}

\section{Análisis de riesgos}
\section{Matriz de riesgos}
\section{Política de seguridad}


\begin{thebibliography}{9}
	\bibitem{REF:crc}
	Bill McDaniel.
	\textit{An Algorithm for Error Correcting Cyclic Redundance Checks}.
	https://www.drdobbs.com/an-algorithm-for-error-correcting-cyclic/184401662, 2002.

	\bibitem{REF:gnupg}
	Linux Man pages.
	\textit{gpg(1) - Linux man page}.
	https://linux.die.net/man/1/gpg.


	\bibitem{REF:veracrypt}
	VeraCrypt.
	\textit{Documentation}.
	https://www.veracrypt.fr/en/Documentation.html.


	\bibitem{REF:mellon}
	Koopman, Philip and Chakravarty, T.
	\textit{Cyclic redundancy code (CRC) polynomial selection for embedded networks}.
	10.1109/DSN.2004.1311885. 2004.


\end{thebibliography}

\end{document}