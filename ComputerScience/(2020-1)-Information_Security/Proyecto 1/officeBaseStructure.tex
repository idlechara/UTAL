
\newcommand{\repeatOnEachOfficeBase}[3]{
	\repetatOfficeInformationResource
	{NTB\_#1}
	{Thinkpad T490 series, equipo corporativo}
	{Hardware TI}
	{#2 - primer piso - recurso estático}
	{#2}
	{4}{5}{3}
	{
		\threatInformationBreach
		\threatResourceLost
		\threatHumanIntervention
		\threatRemoteIntervention
		\threatNaturalDisaster
		\threatHumanDisaster
	}

	\repetatOfficeInformationResource
	{EML\_#1}
	{Cuenta de correo corporativa}
	{Activo de información tangible}
	{EXE\_EXCHA\_001}
	{#2}
	{4}{5}{3}
	{
		\threatInformationBreach
		\threatRemoteIntervention
		\threatTransitive
	}

	\informationResource
	{NAS\_#1\_001}
	{Cisco NSS324 NAS, NAS local para equipo de la #2}
	{Hardware TI}
	{#2 - primer piso - recurso estático}
	{Departamento de TI - Encargado TI de #3}
	{4}{5}{3}
	{
		\threatCVE{CVE-2017-7494}{Ejecución remota de código}
		\threatInformationBreach
		\threatResourceLost
		\threatRemoteIntervention
		\threatNaturalDisaster
		\threatHumanDisaster
		\threatInterest
	}

	\informationResource
	{CAB\_#1\_001}
	{Armario de archivos para #2}
	{Infraestructura TI}
	{#2 - primer piso - recurso estático}
	{Jefe de #2}
	{5}{5}{5}
	{
		\threatNaturalDisaster
		\threatHumanDisaster
		\threatInterest
	}

	\informationResource
	{OFI\_#1\_001}
	{#2 - Instancia física}
	{Infraestructura TI}
	{#2 - primer piso - recurso estático}
	{Administración del edificio}
	{5}{5}{5}
	{
		\threatNaturalDisaster
		\threatHumanDisaster
	}

	\informationResource
	{RNG\_#1\_001}
	{Alarma de #2}
	{Control de entorno}
	{#2 - primer piso - recurso estático}
	{Administración del edificio}
	{5}{5}{5}
	{
		\threatHumanIntervention
		\threatRemoteIntervention
	}
}