% %% tengo que agregar las siguientes plataformas

% Remuneraciones
% Registro comunal de permisos de circulación
% Licencias de conducir
% Patentes municipalesControl de bienes
% Adquisisión y control de bodegas
% Banco de datos propiedas comunales
% Oficina de partes
% DepartamentoTecnologiasJuzgados de polícia local
% mantencion y administración de cartografia digital


% Administracion red departamente licencias de conducir
% administracion red departamento de juzgados de policia local
% partes empadronados departamento de seguridad ciudadana
% administración red departamento patentes municpales y tesorería
% administración red del departamento de seguridad ciudadana
% administración de dirección de obras municipales
% plataforma de egresos
% administracion de sistemas de desarrollo comunitario
% sistemas computaciones por internet y sistemas sociales


% administración de sistemas de egresos, recursos humanos y remuneraciones **
% Contabilidad gubernamental **

% % %% Sistema contabilidad gubernamental
% %     Sistema  contabilidad  gubernamental.
% %     Ingreso de cuentas  contables,  programas  y centro  de costos.
% %     Ingreso   de  tablas   para  el  funcionamiento    del  sistema   (meses,áreas,  tipo  de  comprobantes   contables,  tipo  de  documentos,   tablacentro  costos,  programas,  parámetros  y proveedores).
% %     Ingreso de presupuesto  inicialymodificaciones   presupuestarias.
% %     Ingreso d eobligaciones(contratos,orden de compra,adjudicaciones  y factibilidades).
% %     Ingreso  de devengados  por proveedor  (facturas).
% %     Confección  de órdenes  de pago.
% %     Ingreso y contabilización   de documentos  contables,  rendiciones  decuentas.

% % %% Sistema de tesoreria municipal
% % Boletas  de garantía.
% %     Mantención  de garantías.
% %     Consulta  documento  en garantía.
% %     Ingreso de contratos
% % Egresos.
% %     Emisión  de cheques  de distintas  cuentas  corrientes.
% %     Emisión   de  listados   de  información,   como   cuenta   corriente   deproveedor.
% %     Generación  de listado  de conciliaciones   bancarias  y retenciones  deimpuesto.
% %     Contabilización  de movimientos  contables
% % Ingresos.
% %     Apertura  y cierre de cajas.
% %     Anulación  de ingresos.
% %     Cuadraturas  de cajas.
% %     Contabilización  de ingresos.
% %     Conciliación  de ingresos.
% %     Pagos a través de Internet.
% %     Emisión  informes  varios.
% %     Consulta  de recaudación  por cajas.

% % %% Sistema patentes comerciales
% %     Consulta  de patentes.
% %     Listar patentes  CIPA, según tipo.
% %     Administrar  solicitud  de patente.
% %     Mantención  del maestro  de patentes.
% %     Cálculo  de patentes.
% %     Anulación  de patentes  y/o giros.


% % %% Sistema permisos de circulación
% %     Generación  de giro para pago de permisos  de circulación.
% %     Generación  de duplicado  de permisos  de circulación.
% %     Emisión  de giros de fondos  a terceros
% %     Bloqueo  por sistema  de placas  patentes.
% %     Consultas   de  pagos  años  anteriores,   de  registro  de  multas,   deincorporaciones  y de traslados.
% %     Generación  de giros de sellos.
% %     Mantención  de traslado.
% %     Asignación  de código de S.1.1.
% %     Anulación  de giros mal emitidos

\section{Riesgos asociados a factores no tecnológicos}
%% desastres naturales
\riskItemTable[
    riskTitle={Desastres naturales},
    riskAuthor={Ariel Valenzuela},
    riskDate={12/06/2020},
    riskDescription={ Al materializarse este riesgo, distintas estructuras (departamentos) podrían verse dañadas parcial o totalmente. },
    riskResourceOwner={},
    riskAssociatedProcess={
        
    },
    riskSubArea={ 
        \riskSubAreaAll
    },
    riskSubAreaDependencies={   },
    riskVulnDetails={
        Dado que Chile está ubicado en una zona potencialmente sísmica, la integridad de los recursos físicos se encuentra en un peligro constante.
    },
    riskThreatDetails={
       
    },
    riskResponse={ },
    riskApproval={  }
]{}

%% Multas por servicios operaciones (como agua, luz)
    riskTitle={Multas por Servicios Básicos (Agua, luz)},
    riskAuthor={Ariel Valenzuela},
    riskDate={12/06/2020},
    riskDescription={ Al materializarse este riesgo,  },
    riskResourceOwner={},
    riskAssociatedProcess={
        
    },
    riskSubArea={ 
        \riskSubAreaAll
    },
    riskSubAreaDependencies={   },
    riskVulnDetails={
        dd
    },
    riskThreatDetails={
        dd
    },
    riskResponse={ },
    riskApproval={  }
]{}

%% Pérdida de soporte de proyectos licitados
riskTitle={Pérdidas de soporte de proyectos licitados},
    riskAuthor={Ariel Valenzuela},
    riskDate={12/06/2020},
    riskDescription={ Al materializarse este riesgo,  },
    riskResourceOwner={},
    riskAssociatedProcess={
        
    },
    riskSubArea={ 
        \riskSubAreaAll
    },
    riskSubAreaDependencies={   },
    riskVulnDetails={
        dd
    },
    riskThreatDetails={
        dd
    },
    riskResponse={ },
    riskApproval={  }
]{}
%% Fin de facturaciones
riskTitle={Fin de facturaciones},
    riskAuthor={Ariel Valenzuela},
    riskDate={12/06/2020},
    riskDescription={ Al materializarse este riesgo,  },
    riskResourceOwner={},
    riskAssociatedProcess={
        
    },
    riskSubArea={ 
        \riskSubAreaAll
    },
    riskSubAreaDependencies={   },
    riskVulnDetails={
        dd
    },
    riskThreatDetails={
        dd
    },
    riskResponse={ },
    riskApproval={  }
]{}
%% Desastres lógicos
riskTitle={Desastres lógicos},
    riskAuthor={Ariel Valenzuela},
    riskDate={12/06/2020},
    riskDescription={ Al materializarse este riesgo,  },
    riskResourceOwner={},
    riskAssociatedProcess={
        
    },
    riskSubArea={ 
        \riskSubAreaAll
    },
    riskSubAreaDependencies={   },
    riskVulnDetails={
        dd
    },
    riskThreatDetails={
        dd
    },
    riskResponse={ },
    riskApproval={  }
]{}
%% Divulgación y copiado de informacion
riskTitle={Divulgación y copiado de información},
    riskAuthor={Ariel Valenzuela},
    riskDate={12/06/2020},
    riskDescription={ Al materializarse este riesgo,  },
    riskResourceOwner={},
    riskAssociatedProcess={
        
    },
    riskSubArea={ 
        \riskSubAreaAll
    },
    riskSubAreaDependencies={   },
    riskVulnDetails={
        dd
    },
    riskThreatDetails={
        dd
    },
    riskResponse={ },
    riskApproval={  }
]{}
%% falta de stock de recursos tecnológicos
riskTitle={Falta de stock de recursos tecnológicos},
    riskAuthor={Ariel Valenzuela},
    riskDate={12/06/2020},
    riskDescription={ Al materializarse este riesgo,  },
    riskResourceOwner={},
    riskAssociatedProcess={
        
    },
    riskSubArea={ 
        \riskSubAreaAll
    },
    riskSubAreaDependencies={   },
    riskVulnDetails={
        dd
    },
    riskThreatDetails={
        dd
    },
    riskResponse={ },
    riskApproval={  }
]{}
%% candados de seguridad
riskTitle={Candados de Seguridad},
    riskAuthor={Ariel Valenzuela},
    riskDate={12/06/2020},
    riskDescription={ Al materializarse este riesgo,  },
    riskResourceOwner={},
    riskAssociatedProcess={
        
    },
    riskSubArea={ 
        \riskSubAreaAll
    },
    riskSubAreaDependencies={   },
    riskVulnDetails={
        dd
    },
    riskThreatDetails={
        dd
    },
    riskResponse={ },
    riskApproval={  }
]{}

\section{Riesgos asociados a procesos municpales}
%% Falta de documentación e implantación de políticas para envío de correos masivos
%% Falta de punto de contacto
%% Roles no definidos
%% Falta de monitoreo



\section{Riesgos asociados procesos de atención municipal}

\riskItemTable[
    riskTitle={\riskNamePermisosCirculacionVulnerables},
    riskAuthor={Erik Regla},
    riskDate={12/06/2020},
    riskDescription={ Al materializarse este riesgo, el sistema de permisos de circulación ejecuta operaciones sin necesidad inmediata ni forzada de ingresar datos. },
    riskResourceOwner={Miguel Jorquera},
    riskAssociatedProcess={
        \riskProcessRegistroComunalDePermisosDeCirculacion
    },
    riskSubArea={ 
        \riskSubAreaPago
        \riskSubAreaAsuntosMunicipales
        \riskProcessRegistroComunalDePermisosDeCirculacion
    },
    riskSubAreaDependencies={ \riskPOIEmpresaExternaC },
    riskVulnDetails={
        Es posible que debido aun fallo interno del sistema externalizado de permisos de circulación, emite decretos de pago inválidos o con información errónea.
    },
    riskThreatDetails={
        Al estar externalizado el servicio, no hay control sobre el funcionamiento interno de esta plataforma.
    },
    riskResponse={ \riskRepsonseCompensate },
    riskApproval={ \riskPOIDepartamentoTecnologias }
]{}



\riskItemTable[
    riskTitle={\riskNamePatentesVulnerables},
    riskAuthor={Erik Regla},
    riskDate={12/06/2020},
    riskDescription={ Al materializarse este riesgo, el sistema dede patentes comerciales ejecuta operaciones sin necesidad inmediata ni forzada de ingresar datos. },
    riskResourceOwner={Miguel Jorquera},
    riskAssociatedProcess={
        \riskProcessPatentesMunicipalesB
    },
    riskSubArea={ 
        \riskSubAreaPago
        \riskSubAreaAsuntosMunicipales
        \riskProcessPatentesMunicipales
    },
    riskSubAreaDependencies={ \riskPOIEmpresaExternaA },
    riskVulnDetails={
        Es posible que debido aun fallo interno del sistema externalizado de patentes municipales, emite decretos de pago inválidos o con información errónea.
    },
    riskThreatDetails={
        Al estar externalizado el servicio, no hay control sobre el funcionamiento interno de esta plataforma.
    },
    riskResponse={ \riskRepsonseCompensate },
    riskApproval={ \riskPOIDepartamentoTecnologias }
]{}



\riskItemTable[
    riskTitle={\riskNameEmisionDecretosVulneravbles},
    riskAuthor={Erik Regla},
    riskDate={12/06/2020},
    riskDescription={ Al materializarse este riesgo, el sistema de tesorería municipal o el sistema externalizado de contabilidad gubernamental pueden emitir decretos sin requerir la información. },
    riskResourceOwner={Miguel Jorquera},
    riskAssociatedProcess={
        \riskProcessContabilidadGubernamental 
    },
    riskSubArea={ 
        \riskSubAreaPago
        \riskSubAreaAsuntosMunicipales
        \riskSubAreaRevisionProcesos
    },
    riskSubAreaDependencies={ \riskPOIEmpresaExternaA },
    riskVulnDetails={
        Es posible que debido aun fallo interno del sistema externalizado de contabilidad gubernamental, emite decretos de pago inválidos o con información errónea.
    },
    riskThreatDetails={
        Al estar externalizado el servicio, no hay control sobre el funcionamiento interno de esta plataforma.
    },
    riskResponse={ \riskRepsonseCompensate },
    riskApproval={ \riskPOIDepartamentoTecnologias }
]{}

\riskItemTable[
    riskTitle={\riskNameDecretosDePagoImputadosNoCorresponden},
    riskAuthor={Erik Regla},
    riskDate={12/06/2020},
    riskDescription={ Al materializarse este riesgo, el sistema de tesorería municipal puede emitir cheques a terceros con cargo a la municipalidad. },
    riskResourceOwner={Miguel Jorquera},
    riskAssociatedProcess={
        \riskProcessContabilidadGubernamental 
    },
    riskSubArea={ 
        \riskSubAreaPago
        \riskSubAreaAsuntosMunicipales
        \riskSubAreaRevisionProcesos
    },
    riskSubAreaDependencies={ \riskPOIEmpresaExternaA },
    riskVulnDetails={
        Es posible que debido aun fallo interno del sistema externalizado de contabilidad gubernamental, emite decretos de pago inválidos o con información errónea.
    },
    riskThreatDetails={
        Al estar externalizado el servicio, no hay control sobre el funcionamiento interno de esta plataforma.
    },
    riskResponse={ \riskRepsonseCompensate },
    riskApproval={ \riskPOIDepartamentoTecnologias }
]{}

\riskItemTable[
    riskTitle={\riskNameEmisionDeChequeIndividualVulnerable},
    riskAuthor={Erik Regla},
    riskDate={12/06/2020},
    riskDescription={ Al materializarse este riesgo, el sistema de contabilidad gubernamental puede emitir cheques a terceros con cargo a la municipalidad. },
    riskResourceOwner={Miguel Jorquera},
    riskAssociatedProcess={
        \riskProcessContabilidadGubernamental 
    },
    riskSubArea={ 
        \riskSubAreaPago
    },
    riskSubAreaDependencies={ \riskPOIEmpresaExternaA },
    riskVulnDetails={
        Es posible que debido aun fallo interno del sistema externalizado de contabilidad gubernamental, este pueda ser utilizado para la extraccion de fondos de manera ilícita.
    },
    riskThreatDetails={
        Al estar externalizado el servicio, no hay control sobre el funcionamiento interno de esta plataforma.
    },
    riskResponse={ \riskRepsonseCompensate },
    riskApproval={ \riskPOIDepartamentoTecnologias }
]{}

\riskItemTable[
    riskTitle={Recepción de pagos con cálculos de intereses y multas fuera de período},
    riskAuthor={Erik Regla},
    riskDate={12/06/2020},
    riskDescription={ Al materializarse este riesgo, es posible ingresar un pago con cálculos erroneos en el sistema, los cuales se ven reflejados posteriormente en el sistema interno de contabilidad. },
    riskResourceOwner={Miguel Jorquera},
    riskAssociatedProcess={
        \riskProcessSistemasComputacionesPorInternetYSistemasSociales
        \riskProcessAdministracionDeSistemasDeEgresosRecursosHumanosYRemuneraciones \\ &
        \riskProcessContabilidadGubernamental \\ &
        \riskProcessTramitesDeOficinaDePartes \\ &
        \riskProcessPatentesMunicipales \\ &
    },
    riskSubArea={ 
        \riskSubAreaAsuntosMunicipales \\ &
        \riskSubAreaRevisionProcesos \\ &
        \riskSubAreaPago
    },
    riskSubAreaDependencies={ \riskPOIEmpresaExternaA },
    riskVulnDetails={
        Es posible que debido aun fallo interno del sistema externalizado para pagos de la tesorería municipal puedan ejecutarse pagos sin respetar los calculos establecidos para la contabilidad de multas y demases.
    },
    riskThreatDetails={
        Al estar externalizado el servicio, no hay control sobre el funcionamiento interno de esta plataforma, en especial considerando su estatus de plataforma legada.
    },
    riskResponse={ \riskRepsonseCompensate },
    riskApproval={ \riskPOIDepartamentoTecnologias }
]{}

\section{Riesgos asociados a control del personal}

\riskItemTable[
    riskTitle={Quiebre autenticación de tarjeta magnética},
    riskAuthor={Erik Regla},
    riskDate={12/06/2020},
    riskDescription={ Al materializarse este riesgo, las llaves magnéticas que utilizadas como barrera para personal no autorizado quedan sin efecto. },
    riskResourceOwner={Miguel Jorquera},
    riskAssociatedProcess={
        \riskProcessSeguridadInfraestructura
    },
    riskSubArea={ 
        \riskSubAreaDepartamentoTecnologias
    },
    riskSubAreaDependencies={ \riskPOIDepartamentoTecnologias },
    riskVulnDetails={
        De ser quebrada la autenticación de las tarjetas magnéticas utilizadas para acceder a las salas de servidores, la integridad completa de los dispositivos queda comprometida.
    },
    riskThreatDetails={
        Este tipo de amenzas se presenta principalmente por un factor físico dada la dificultad de intervenir las cerraduras magnéticas. Copias de las tarjetas, duplicaciones, generaciones de maestros son algunas de las amenazas posibles las cuales pueden ser ejecutadas si existe una brecha de información respecto a la infraestructura de las cerraduras, la pérdida de una tarjeta o el robo de esta.
    },
    riskResponse={ \riskRepsonsePrevent },
    riskApproval={ \riskPOIDepartamentoTecnologias }
]{}

\riskItemTable[
    riskTitle={Copia de llaves},
    riskAuthor={Erik Regla},
    riskDate={12/06/2020},
    riskDescription={ Al materializarse este riesgo, las llaves físicas que utilizadas como barrera para personal no autorizado quedan sin efecto. },
    riskResourceOwner={Miguel Jorquera},
    riskAssociatedProcess={
        \riskProcessSeguridadInfraestructura
        \riskProcessSeguridadPersonal
    },
    riskSubArea={ 
        \riskSubAreaAll
    },
    riskSubAreaDependencies={ \riskPOIAlcalde },
    riskVulnDetails={
        Al efectuarse una copia física de las llaves, la seguridad que estas proveen queda inutilizable. Por tanto ya no es posible contar con la seguridad de control de personal que estas ofrecen.
    },
    riskThreatDetails={
        El acceso físico no es solo relevante por el compromiso de infraestructura que este pueda poseer, si no por la rapidez con la que esta puede generar problemas alternos (como la propagación de la misma copia) y a su vez deja en peligro al personal ya que seguridad no puede desempeñar correctamente sus funciones.
    },
    riskResponse={ \riskRepsonsePrevent },
    riskApproval={ \riskPOIDepartamentoTecnologias }
]{}



\riskItemTable[
    riskTitle={Quiebre autenticación de llave de seguridad},
    riskAuthor={Erik Regla},
    riskDate={12/06/2020},
    riskDescription={ Al materializarse este riesgo, la información de autenticación queda comprometida a terceras partes },
    riskResourceOwner={Miguel Jorquera},
    riskAssociatedProcess={
        \riskProcessIntegridad
    },
    riskSubArea={ 
        \riskSubAreaAll
    },
    riskSubAreaDependencies={ \riskPOIAlcalde },
    riskVulnDetails={
        El compromiso de autenticación de un usuario compromete en su totalidad todos los niveles de seguridad permitidos, poniendo en riesgo la confidencialidad, integridad y disponiblidad de información y servicios.
    },
    riskThreatDetails={
        Puede ocurrir al dejar contraseñas escritas en medios físicos como post-it, almacenadas dentro de archivos planos en unidades extraibles o en equipos personales, etc.
    },
    riskResponse={ \riskRepsonsePrevent },
    riskApproval={ \riskPOIDepartamentoTecnologias }
]{}



\section{Riesgos asociados de índole técnica}

\riskItemTable[
    riskTitle={\riskNameInteroperabilidadConEstandarInexistente},
    riskAuthor={\riskAuthorErik},
    riskDate={12/06/2020},
    riskDescription={ Al materializarse este riesgo, los datos son corrompidos debido a un mal diseño de alguna estructura de datos subyacente a la aplicación. },
    riskResourceOwner={
        \riskPOIEmpresaExternaA \\&
        \riskPOIEmpresaExternaB \\&
        \riskPOIEmpresaExternaC
    },
    riskAssociatedProcess={
        \riskProcessRegistroComunalDePermisosDeCirculacion \\ &
        \riskProcessContabilidadGubernamental \\ &
        \riskProcessPatentesMunicipales
    },
    riskSubArea={ 
        \riskSubAreaPago
        \riskSubAreaAsuntosMunicipales
        \riskSubAreaAuditoria
        \riskSubAreaRevisionProcesos
    },
    riskSubAreaDependencies={ 
        \riskPOIEmpresaExternaA \\&
        \riskPOIEmpresaExternaB \\&
        \riskPOIEmpresaExternaC },
    riskVulnDetails={
        Un manejo deficiente de las estructuras de datos involucradas en el desarrollo de las aplicaciones o de las bases de datos puede llevar a corrupcion de los datos debido a múltiples factores.
    },
    riskThreatDetails={
        Actualmente la municipalidad externaliza gran parte de los servicios informaticos como tambien la mantención de muchos de sus equipos. Debido a esto, en caso de que uno de los proveedores entregue un servicio desarollado de manera deficiente, resulta en un incremento de la probabilidad de un evento de perdida de datos.
    },
    riskResponse={ \riskRepsonseCompensate },
    riskApproval={ \riskPOIDepartamentoTecnologias }
]{}


\riskItemTable[
    riskTitle={\riskNameInteroperabilidadConEstandarInexistente},
    riskAuthor={\riskAuthorErik},
    riskDate={12/06/2020},
    riskDescription={ Al materializarse este riesgo, los datos son corrompidos debido a un mal diseño de alguna estructura de datos subyacente a la aplicación. },
    riskResourceOwner={
        \riskPOIDepartamentoTecnologias
    },
    riskAssociatedProcess={
        \riskProcessGeneral
    },
    riskSubArea={ 
        \riskSubAreaAll
    },
    riskSubAreaDependencies={ \riskPOIDepartamentoTecnologias },
    riskVulnDetails={
        Un manejo deficiente de las estructuras de datos involucradas en el desarrollo de las aplicaciones o de las bases de datos puede llevar a corrupcion de los datos debido a múltiples factores.
    },
    riskThreatDetails={
        Actualmente la municipalidad externaliza gran parte de los servicios informaticos como tambien la mantención de muchos de sus equipos. Debido a esto, en caso de que uno de los proveedores entregue un servicio desarollado de manera deficiente, resulta en un incremento de la probabilidad de un evento de perdida de datos .
    },
    riskResponse={ \riskRepsonseCorrect },
    riskApproval={ \riskPOIDepartamentoTecnologias }
]{}


\riskItemTable[
    riskTitle={\riskNameIntegridadDeInformacionPorEstructuraDeDatos},
    riskAuthor={\riskAuthorErik},
    riskDate={12/06/2020},
    riskDescription={ Al materializarse este riesgo, los datos son corrompidos debido a un mal diseño de alguna estructura de datos subyacente a la aplicación. },
    riskResourceOwner={
        \riskPOIEmpresaExternaA \\&
        \riskPOIEmpresaExternaB \\&
        \riskPOIEmpresaExternaC
    },
    riskAssociatedProcess={
        \riskProcessRegistroComunalDePermisosDeCirculacion \\ &
        \riskProcessContabilidadGubernamental
    },
    riskSubArea={ 
        \riskSubAreaPago
        \riskSubAreaAsuntosMunicipales
        \riskSubAreaAuditoria
        \riskSubAreaRevisionProcesos
    },
    riskSubAreaDependencies={ \riskPOIDepartamentoTecnologias },
    riskVulnDetails={
        Un manejo deficiente de las estructuras de datos involucradas en el desarrollo de las aplicaciones o de las bases de datos puede llevar a corrupcion de los datos debido a múltiples factores.
    },
    riskThreatDetails={
        Actualmente la municipalidad externaliza gran parte de los servicios informaticos como tambien la mantención de muchos de sus equipos. Debido a esto, en caso de que uno de los proveedores entregue un servicio desarollado de manera deficiente, resulta en un incremento de la probabilidad de un evento de perdida de datos .
    },
    riskResponse={ \riskRepsonseCompensate },
    riskApproval={ \riskPOIDepartamentoTecnologias }
]{}


\riskItemTable[
    riskTitle={Carencia de licencias},
    riskAuthor={\riskAuthorErik},
    riskDate={12/06/2020},
    riskDescription={ Al materializarse este riesgo, la carencia de una licencia podría limitar la  disponibilidad de un servicio que dependa de esta. },
    riskResourceOwner={\riskPOIGeneral},
    riskAssociatedProcess={
        \riskProcessGeneral
    },
    riskSubArea={ 
        \riskSubAreaAll
    },
    riskSubAreaDependencies={ \riskPOIDepartamentoTecnologias },
    riskVulnDetails={
        Para el caso de software que opera con licencias (Office 365 por ejemplo), la no disponibilidad de las mismas puede ocasionar problemas al momento de utilizar otros servicios
    },
    riskThreatDetails={
        Actualmente debido al convenio con Microsoft vigente por parte del gobierno actual, muchos softwares están a merced de que estas licencias estén disponibles. Sin embargo, la no caducidad no tiene relación alguna con las licencias asignadas, ya que dependiendo del tier involucrado en las licencias asignadas, son los servicios disponibles.
    },
    riskResponse={ \riskRepsonsePrevent },
    riskApproval={ \riskPOIDepartamentoTecnologias, \riskPOIPago, \riskPoiRevisionProcesos }
]{}

\riskItemTable[
    riskTitle={Dependencia de licencias},
    riskAuthor={\riskAuthorErik},
    riskDate={12/06/2020},
    riskDescription={ Al materializarse este riesgo, la invalidación de una licencia podría provocar problemas de seguridad o bien interrupciones en la disponibilidad de un servicio. },
    riskResourceOwner={\riskPOIGeneral},
    riskAssociatedProcess={
        \riskProcessUsoDeSitioWeb
    },
    riskSubArea={ 
        \riskSubAreaAll
    },
    riskSubAreaDependencies={ \riskPOIDepartamentoTecnologias, \riskPOIEmpresasExternas  },
    riskVulnDetails={
        Para el caso de software que opera con licencias (Office 365 por ejemplo), la caducidad de las mismas puede generar interrupciones o bien dejar de dar soporte a nuevas amenazas
    },
    riskThreatDetails={
        Actualmente debido al convenio con Microsoft vigente por parte del gobierno actual, muchos softwares están a merced de que estas licencias no caduquen. Esto podría producirse por múltiples factores, no disponibilidad del retailer, cambio de versiones, no soporte de cambios, olvido de pagos, etc.
    },
    riskResponse={ \riskRepsonsePrevent },
    riskApproval={ \riskPOIDepartamentoTecnologias, \riskPOIPago, \riskPoiRevisionProcesos }
]{}

\riskItemTable[
    riskTitle={Transaccion rota},
    riskAuthor={\riskAuthorErik},
    riskDate={12/06/2020},
    riskDescription={ Al materializarse este riesgo, es posible leer la información directamente desdfe el medio en que se encuentra sin ninguna barrera de seguridad },
    riskResourceOwner={\riskPOIDepartamentoTecnologias},
    riskAssociatedProcess={
        \riskProcessIntegridad
    },
    riskSubArea={ 
        \riskSubAreaAll
    },
    riskSubAreaDependencies={ \riskPOIDepartamentoTecnologias, \riskPOIEmpresasExternas  },
    riskVulnDetails={
        Actualmente no hay ningun mecanismo de respaldo para operaciones de índole transaccional, lo cual puede provocar pérdidas de información.
    },
    riskThreatDetails={
        Al no haber un registro de comunicaciones llevadas a cabo de manera transaccional, en el momento de existir peticiones a los distintos serivcios que pudan provocar un conflicto, este puede resultar en inconsistencias, corrupción y pérdida de datos.

        Sin embargo, Dado que los riesgos son mínimos de por el momento y no ha ocurrido no se le da mayor importancia, a excepción del sistema de pago.
    },
    riskResponse={ \riskRepsonseCompensate },
    riskApproval={ \riskPOIDepartamentoTecnologias }
]{}

\riskItemTable[
    riskTitle={Falta de encriptado},
    riskAuthor={\riskAuthorErik},
    riskDate={12/06/2020},
    riskDescription={ Al materializarse este riesgo, es posible leer la información directamente desdfe el medio en que se encuentra sin ninguna barrera de seguridad },
    riskResourceOwner={\riskPOIDepartamentoTecnologias},
    riskAssociatedProcess={
        \riskProcessResguardoInformacionPersonal \\&
        \riskProcessResguardoInformacionInstitucional \\&
    },
    riskSubArea={ 
        \riskSubAreaAll
    },
    riskSubAreaDependencies={ \riskPOIDepartamentoTecnologias, \riskPOIEmpresasExternas  },
    riskVulnDetails={
        Una auditoria realizada por contraloría reveló que no existe encruiptación de los datos almacenados digitalmente salvo en la capa de transporte.
    },
    riskThreatDetails={
        La falta de encriptación puede producir fuga de información sensible.
    },
    riskResponse={ \riskRepsonseCorrect },
    riskApproval={ \riskPOIAlcalde }
]{}


\riskItemTable[
    riskTitle={Residuos de información},
    riskAuthor={\riskAuthorErik},
    riskDate={12/06/2020},
    riskDescription={ Al materializarse este riesgo, información que haya sido borrada sigue disponible dentro de una base de datos sin ser detectada },
    riskResourceOwner={\riskPOIDepartamentoTecnologias},
    riskAssociatedProcess={
        \riskProcessResguardoInformacionPersonal \\&
        \riskProcessResguardoInformacionInstitucional \\&
    },
    riskSubArea={ 
        \riskSubAreaAll
    },
    riskSubAreaDependencies={ \riskPOIDepartamentoTecnologias, \riskPOIEmpresasExternas  },
    riskVulnDetails={
        Una auditoria realizada por contraloría reveló que no existen protocolos de eliminación de la información de manera interna y esta tampoco forma parte de los servicios contratados.
    },
    riskThreatDetails={
        Al no existir un protocolo de eliminado de información claro, es altamente probable que la información no pueda ser eliminada de manera efectiva ya sea de plataformas, dispositivos, medios extraibles, etc. Este problema aplica también a los archivos físicos que no cuenten con respaldo y que dentro de las operaciones vigentes consideren su eliminación. 
    },
    riskResponse={ \riskRepsonseCorrect },
    riskApproval={ \riskPOIAlcalde }
]{}

\riskItemTable[
    riskTitle={Residuos de información},
    riskAuthor={\riskAuthorErik},
    riskDate={12/06/2020},
    riskDescription={ Al materializarse este riesgo, información que haya sido borrada sigue disponible dentro de una base de datos sin ser detectada },
    riskResourceOwner={\riskPOIDepartamentoTecnologias},
    riskAssociatedProcess={
        \riskProcessResguardoInformacionPersonal \\&
        \riskProcessResguardoInformacionInstitucional \\&
    },
    riskSubArea={ 
        \riskSubAreaAll
    },
    riskSubAreaDependencies={ \riskPOIDepartamentoTecnologias, \riskPOIEmpresasExternas  },
    riskVulnDetails={
        Una auditoría realizada por contraloría reveló que no existen protocolos de eliminación de la información de manera interna y esta tampoco forma parte de los servicios contratados.
    },
    riskThreatDetails={
        Al no existir un protocolo de eliminado de información claro, es altamente probable que la información no pueda ser eliminada de manera efectiva ya sea de plataformas, dispositivos, medios extraíbles, etc. Este problema aplica también a los archivos físicos que no cuenten con respaldo y que dentro de las operaciones vigentes consideren su eliminación. 
    },
    riskResponse={ \riskRepsonseCorrect },
    riskApproval={ \riskPOIAlcalde }
]{}

\riskItemTable[
    riskTitle={Mantenimiento preventivo externalizado ejecutado deficientemente},
    riskAuthor={\riskAuthorErik},
    riskDate={12/06/2020},
    riskDescription={ Al materializarse este riesgo, las plataformas sujetas a mantenimiendo por parte de una empresa externa podrían quedar expuestas a vulnerabilidades },
    riskResourceOwner={\riskPOIEmpresasExternas},
    riskAssociatedProcess={
        \riskProcessResguardoInformacionPersonal \\&
        \riskProcessResguardoInformacionInstitucional \\&
        \riskProcessResguardoPlataforma
    },
    riskSubArea={ 
        \riskSubAreaAsuntosMunicipales
        \riskSubAreaPago
        \riskSubAreaAuditoria
    },
    riskSubAreaDependencies={ \riskPOIDepartamentoTecnologias, \riskPOIEmpresasExternas  },
    riskVulnDetails={
        Mantenciones negligentes, omitidas, incompletas.
    },
    riskThreatDetails={
        Una mantención negligente de las plataformas puede llevar a un uso malicioso de estas, las cuales pueden perjudicar enormemente el servicio entregado por la municipalidad como también poner en riesgo los datos disponibles en esta. Actualmente debido a la normativa actual, todas las aplicaciones están alojadas en servidores de la municipalidad, sin embargo, no implica que el código esté necesariamente abierto o que el personal propio del departamento de tecnologías pueda tener el conocimiento suficiente sobre este para tomar control completo.
    },
    riskResponse={ \riskRepsonsePrevent },
    riskApproval={ \riskPOIDepartamentoTecnologias }
]{}

\riskItemTable[
    riskTitle={Denegación de servicio},
    riskAuthor={\riskAuthorErik},
    riskDate={11/06/2020},
    riskDescription={
        Al materializarse este riesgo, el sitio web de la municipalidad deja de quedar disponible para todo público.},
    riskResourceOwner={\riskPOIDepartamentoTecnologias},
    riskAssociatedProcess={
        Nivel general, \riskProcessUsoDeSitioWeb
    },
    riskSubArea={ 
        \riskSubAreaDepartamentoTecnologias 
    },
    riskSubAreaDependencies={ \riskPOIDepartamentoTecnologias, \riskPOIEmpresasExternas  },
    riskVulnDetails={
        La denegación de servicio es un tipo de ataque cuyo fin es eliminar temporal o parcialmente la disponibilidad de un servicio, usualmente por medios como ICMP Flood.
    },
    riskThreatDetails={
        Si bien la aplicación está funcionando con las últimas versiones de PHP y de MYQSL disponibles, la infraestructura al ser local y no contar con un WAF, no hay filtro respecto a las peticiones que son resueltas en el servidor. Debido a esto, en caso de llegar un número importante de peticiones las cuales no pudiesen resolverse simultaneamente, podría ocurrir un problema de overflow de memoria colapsando el proceso.

        Cabe destacar que esto también puede ocurrir de manera orgánica en situaciones de alta demanda. Y debido a los acuerdos internos de desarrollo estandarizado, está presente en todas las plataformas desarolladas para uso interno.
    },
    riskResponse={ \riskRepsonseCorrect },
    riskApproval={ \riskPOIDepartamentoTecnologias }
]{}


\riskItemTable[
    riskTitle={Ejecución remota de código},
    riskAuthor={\riskAuthorErik},
    riskDate={11/06/2020},
    riskDescription={
        Al materializarse este riesgo, el atacante ejecuta código en el navegador del cliente sin previo consentimiento.},
    riskResourceOwner={\riskPOIDepartamentoTecnologias},
    riskAssociatedProcess={
        Nivel general, \riskProcessUsoDeSitioWeb
        % \threatInterest \\&
        % \threatHumanIntervention
    },
    riskSubArea={ 
        \riskSubAreaAll
    },
    riskSubAreaDependencies={ \riskPOIDepartamentoTecnologias  },
    riskVulnDetails={
        La ejecución remota de código permite que un usuario no autorizado ejecute instrucciones en otro equipo.
    },
    riskThreatDetails={
        Esta amenaza está atribuida a CVE-2019-9787, el cual especifica una vulnerabilidad sobre la ejecución remota de código por medio de CRSRF. Este tipo de ataque fuerza al usuario a ejecutar código utilizando sus credenciales ya cargadas en la aplicación.
    },
    riskResponse={ \riskRepsonsePrevent },
    riskApproval={ \riskPOIDepartamentoTecnologias }
]{}

\riskItemTable[
    riskTitle={Manipulación de redirecciones},
    riskAuthor={\riskAuthorErik},
    riskDate={11/06/2020},
    riskDescription={
        Al materializarse este riesgo, el atacante fuerza la redirección a un sitio externo.},
    riskResourceOwner={\riskPOIDepartamentoTecnologias},
    riskAssociatedProcess={
        Nivel general, \riskProcessUsoDeSitioWeb \\&
        Nivel general, \riskProcessConfiabilidadDeSitioWeb
    },
    riskSubArea={ 
        \riskSubAreaAll
    },
    riskSubAreaDependencies={ \riskPOIDepartamentoTecnologias  },
    riskVulnDetails={
        La ejecución remota de código permite que un usuario no autorizado ejecute instrucciones en otro equipo.
    },
    riskThreatDetails={
        Esta amenaza está atribuida a CVE-2019-16220, el cual especifica una vulnerabilidad sobre la ejecución remota de código por medio de CRSRF. Este tipo de ataque fuerza al usuario a ejecutar código utilizando sus credenciales ya cargadas en la aplicación.
    },
    riskResponse={ \riskRepsonseMitigate },
    riskApproval={ \riskPOIDepartamentoTecnologias }
]{}


\riskItemTable[
    riskTitle={Ejecución de malwarepor falta de software AV},
    riskAuthor={\riskAuthorErik},
    riskDate={11/06/2020},
    riskDescription={
        Al materializarse este riesgo, el servidor principal de la municipalidad queda comprometido.},
    riskResourceOwner={\riskPOIDepartamentoTecnologias},
    riskAssociatedProcess={
    },
    riskSubArea={ 
        \riskSubAreaAll
    },
    riskSubAreaDependencies={ \riskPOIDepartamentoTecnologias  },
    riskVulnDetails={
        Ataques por randomware, gusanos, trojanos, etc.
    },
    riskThreatDetails={
        Debido al alto número de vulnerabilidades presentes en el sistema operativo, es posible que la materalización de un riesgo en un equipo de una red adyacente pueda propagar procesos de terceros y estos comprometan el servidor principal.
    },
    riskResponse={ \riskRepsonsePrevent },
    riskApproval={ \riskPOIDepartamentoTecnologias }
]{}


\section{Riesgos generales asociados a ingenería social}

\riskItemTable[
    riskTitle={Phishing},
    riskAuthor={\riskAuthorErik},
    riskDate={11/06/2020},
    riskDescription={
        Al materializarse este riesgo, un usuario ingresa información institucional a un sitio falso.},
    riskResourceOwner={\riskPOIDepartamentoTecnologias},
    riskAssociatedProcess={
    },
    riskSubArea={ 
        \riskSubAreaAll
    },
    riskSubAreaDependencies={ \riskPOIDepartamentoTecnologias  },
    riskVulnDetails={
        Un usuario recibe un correo con un mensaje falso pero con apariencia visual creíble, de esta manera para tentar al usuario a ejecutar alguna acción que pueda comprometer la seguridad, ya sea filtrando credenciales o información sensible.
    },
    riskThreatDetails={
        Un ataque de Phishing implica la personificación de otro individuo o entidad, la cual actúa como emisor de un mensaje el cual puede ser de interés del usuario. En este caso la apuesta es que el lector del correo hará caso del call to action antes de verificar la veracidad del contenido, por lo que este tipo de ataques está dirigido a un público no técnico.
    },
    riskResponse={ \riskRepsonsePrevent },
    riskApproval={ \riskPOIDepartamentoTecnologias }
]{}
