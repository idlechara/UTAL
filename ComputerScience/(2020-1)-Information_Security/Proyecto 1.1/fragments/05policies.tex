% %% dependencies
% %% Approbal 
\newcommand{\riskPOIAsuntosMunicipales}{
	Jefe de Oficina de Asuntos Municipales.
}
\newcommand{\riskPOIConsejoMunicipal}{
	Jefe de Consejo municipal.
}
\newcommand{\riskPOISecretariaMunicipal}{
	Jefe de Secretaría Municipal
}
\newcommand{\riskPOIDireccionControl}{
	Jefe de Dirección de Control.
}
\newcommand{\riskPOIDireccionComunicaciones}{
	Jefe de Dirección de Comunicaciones.
}
\newcommand{\riskPOIDireccionAsistenciaJuridica}{
	Jefe de Dirección de Asesoría Jurídica.
}
\newcommand{\riskPOIDepartamentoTecnologias}{
	Jefe de Departamento de Tecnologías de la Información.
}
\newcommand{\riskPOISecretariaComunalDePlanificacion}{
	Jefe de Secretaría Comunal de Planificación.
}
\newcommand{\riskPOIDireccionTransitoTransportePublico}{
	Jefe de Dirección de Tránsito y Transporte Público.|
}
\newcommand{\riskPOIDesarolloComunitario}{
	Jefe de Dirección de Desarrollo Comunitario
}
\newcommand{\riskPOIDireccionObrasMunicipales}{
	Jefe de Dirección de Obras Municipales
}
\newcommand{\riskPOIDireccionAdministracionFinanzas}{
	Jefe de Dirección de Administración y Finanzas
}
\newcommand{\riskPOIDireccionFiscalizacion}{
	Jefe de Dirección de Fiscalización
}
\newcommand{\riskPOIDireccionDeOperaciones}{
	Jefe de Dirección de Operaciones y Servicios Urbanos
}
\newcommand{\riskPOIDireccionCulturaTurismo}{
	Jefe de Dirección de Cultura y Turismo
}
\newcommand{\riskPOIDireccionSeguridadVecinal}{
	Jefe de Dirección de Seguridad Vecinal y Resguardo
}
\newcommand{\riskPOIDireccionGestionPersonas}{
	Jefe de Dirección de Gestión de Personas
}
\newcommand{\riskPOIDireccionSalud}{
	Jefe de Dirección de Salud
}
\newcommand{\riskPOIDireccionEducacion}{
	Jefe de Dirección de Educación
}
\newcommand{\riskPOIEmpresaExterna1}{
	Empresa Externa 1
}
\newcommand{\riskPOIEmpresaExterna2}{
	Empresa Externa 2
}
\newcommand{\riskPOIEmpresaExterna3}{
	Empresa Externa 3
}
\newcommand{\riskPOIEmpresasExternas}{
	Empresas externas
}
\newcommand{\riskPOIAlcalde}{
	Alcalde
}
\newcommand{\riskPOIGeneral}{
	Dominio General
}
\newcommand{\riskPOIArchivo}{
	Jefe de Departamento de Certificación y Archivo
}

\newcommand{\riskPOIDepartamentoTecnologias}{
	Jefe de Departamento de Tecnologías de la Información.
}
\newcommand{\riskPOICartografia}{
	Jefe de Departamento de cartografía
}

\newcommand{\riskPOIAuditoria}{
	Jefe de Departamento de auditoria operativa
}
\newcommand{\riskPOIPago}{
	Jefe de Departamento revisión de procesos de pago, bienes y servicios
}

\newcommand{\riskPoiRevisionProcesos}{
	Jefe de Departamento de revisión de procesos de contratación pública
}

% 
%% Authors
\newcommand{\riskAuthorErik}{
	Erik Regla
}\newcommand{\riskAuthorAriel}{
	Ariel Valenzuela
}\newcommand{\riskAuthorMiguel}{
	Miguel Jorquera
}

%% Response 
\newcommand{\riskRepsonsePrevent}{
	PREVENIR
}\newcommand{\riskRepsonseMitigate}{
	MITIGAR
}\newcommand{\riskRepsonseCorrect}{
	CORREGIR
}\newcommand{\riskRepsonseCompensate}{
	COMPENSAR
}\newcommand{\riskRepsonseTransfer}{
	TRANSFERIR
}

% %% Processes 
\newcommand{\riskProcessGetAppointment}{
	Obtención de hora de atención.
}\newcommand{\riskProcessNewSuggestion}{
	Entrega de sugerencias.
}\newcommand{\riskProcessNewClaim}{
	Entrega de reclamos.
}\newcommand{\riskProcessGetInformation}{
	Solicitud de información.
}\newcommand{\riskProcessUsoDeSitioWeb}{
	Disponibilidad sitio web.
}\newcommand{\riskProcessConfiabilidadDeSitioWeb}{
	Confiabilidad sitio web.
}\newcommand{\riskProcessResguardoInformacionPersonal}{
	Resguardo de información personal.
}\newcommand{\riskProcessResguardoInformacionInstitucional}{
	Resguardo de información institucional.
}
% 
\makeatletter
\define@key{riskItem}{riskTitle}{\def\risk@riskTitle{#1}}
\define@key{riskItem}{riskAuthor}{\def\risk@riskAuthor{#1}}
\define@key{riskItem}{riskDate}{\def\risk@riskDate{#1}}
\define@key{riskItem}{riskDescription}{\def\risk@riskDescription{#1}}
\define@key{riskItem}{riskResourceOwner}{\def\risk@riskResourceOwner{#1}}
\define@key{riskItem}{riskAssociatedProcess}{\def\risk@riskAssociatedProcess{#1}}
\define@key{riskItem}{riskSubArea}{\def\risk@riskSubArea{#1}}
\define@key{riskItem}{riskSubAreaDependencies}{\def\risk@riskSubAreaDependencies{#1}}
\define@key{riskItem}{riskVulnDetails}{\def\risk@riskVulnDetails{#1}}
\define@key{riskItem}{riskThreatDetails}{\def\risk@riskThreatDetails{#1}}
\define@key{riskItem}{riskResponse}{\def\risk@riskResponse{#1}}
\define@key{riskItem}{riskApproval}{\def\risk@riskApproval{#1}}

\setkeys{riskItem}{
riskTitle=riskTitle,
riskAuthor=riskAuthor,
riskDate=riskDate,
riskDescription=riskDescription,
riskResourceOwner=riskResourceOwner,
riskAssociatedProcess=riskAssociatedProcess,
riskSubArea=riskSubArea,
riskSubAreaDependencies=riskSubAreaDependencies,
riskVulnDetails=riskVulnDetails,
riskThreatDetails=riskThreatDetails,
riskResponse=riskResponse,
riskApproval=riskApproval
}

\newcommand{\riskItemTable}[2][]{%
    \setkeys{riskItem}{#1}% Set the keys

    \begin{tabularx}{\linewidth}{|r|X|}
        \hline
        \textbf{Título de Riesgo}      & \risk@riskTitle  \\ 
        \hline
        \textbf{Autor} & \risk@riskAuthor  \\ 
        \hline
        \textbf{Fecha de Levantamiento}   & \risk@riskDate  \\ 
        \hline
        \textbf{Descripción}   & \risk@riskDescription  \\ 
        \hline
        \textbf{Dueño del Activo} & \risk@riskResourceOwner  \\ 
        \hline
        \textbf{Proceso}      & \risk@riskAssociatedProcess  \\ 
        \hline
        \textbf{Sub Área} & \risk@riskSubArea  \\ 
        \hline
        \textbf{Dependencia}   & \risk@riskSubAreaDependencies  \\ 
        \hline
        \textbf{Detalle de la Vulnerabilidad}   & \risk@riskVulnDetails  \\ 
        \hline
        \textbf{Detalle de la amenaza} & \risk@riskThreatDetails  \\ 
        \hline
        \textbf{Respuesta}   & \risk@riskResponse  \\ 
        \hline
        \textbf{Aprobación} & \risk@riskApproval  \\ 
        \hline
        %\textbf{Valoración}  &  \makecell[l]{Confidencialidad: #6\\ Integridad: #7\\ Disponibilidad: #8}\\
        %\hline
        %\textbf{\makecell[r]{Vulnerabilidades y \\Amenazas}}  &  \makecell[l]{#9}\\ 
        %\hline
    \end{tabularx}
    #2
}

\makeatother
% %% SubAreas y dependencias
\newcommand{\riskSubAreaAsuntosMunicipales}{
	Departamento de Asuntos Municipales.
}
\newcommand{\riskSubAreaArchivo}{
	Departamento de Certificación y Archivo
}

\newcommand{\riskSubAreaDepartamentoTecnologias}{
	Departamento de Tecnologías de la Información.
}
\newcommand{\riskSubAreaCartografia}{
	Departamento de cartografía
}
\newcommand{\riskSubAreaRevisionProcesos}{
	Departamento de revisión de procesos de contratación pública
}
\newcommand{\riskSubAreaAuditoria}{
	Departamento de auditoria operativa
}
\newcommand{\riskSubAreaPago}{
	Departamento revisión de procesos de pago, bienes y servicios
}
\newcommand{\riskSubAreaNone}{
	Ninguna.
}
\newcommand{\riskSubAreaAll}{
	Todas.
}
\newcommand{\riskSubAreaMajor}{
	Alcalde.
}

% %%Begin of content

% \section{Riesgos asociados a la atención de público}
\riskItemTable[
    riskTitle={Imposibilidad de obtener una hora de atención a público},
    riskAuthor={\riskAuthorErik},
    riskDate={11/06/2020},
    riskDescription={
        Al materializarse este riesgo, una persona natural
        queda imposibilitada de poder solicitar una hora para su atención, 
        lo cual genera un cuello de botella.},
    riskResourceOwner={\riskPOIAsuntosMunicipales},
    riskAssociatedProcess={
        \riskProcessGetAppointment  \\&
        \riskProcessNewSuggestion   \\&
        \riskProcessNewClaim        \\&
        \riskProcessGetInformation
    },
    riskSubArea={ \riskSubAreaOIRS },
    riskSubAreaDependencies={ \riskSubAreaAsuntosMunicipales },
    riskVulnDetails={
        Bla bla bla, el programa utilizado en bla bla...
        
        Debido a que la versión del sistema operativo utilizado tiene una gran cantidad de CVEs activos, existe la posibilidad de que un usuario malicioso intente generar un ataque de denegación de servicio dentro de la misma municipalidad.
    },
    riskThreatDetails={
        Bla bla bla, el programa utilizado en bla bla...
        
        Debido a que la versión del sistema operativo utilizado tiene una gran cantidad de CVEs activos, existe la posibilidad de que un usuario malicioso intente generar un ataque de denegación de servicio dentro de la misma municipalidad.
    },
    riskResponse={ \riskRepsonsePrevent },
    riskApproval={ \riskPOIAsuntosMunicipales }
]{}

\section{Riesgos asociados a plataformas o tecnologías generalizados dentro de la organización}

\riskItemTable[
    riskTitle={Denegación de servicio},
    riskAuthor={\riskAuthorErik},
    riskDate={11/06/2020},
    riskDescription={
        Al materializarse este riesgo, el sitio web de la municipalidad deja de quedar disponible para todo público.},
    riskResourceOwner={\riskPOIDepartamentoTecnologias},
    riskAssociatedProcess={
        Nivel general, \riskProcessUsoDeSitioWeb
    },
    riskSubArea={ 
        \riskSubAreaDireccionComunicaciones \\&
        \riskSubAreaAsuntosMunicipales      \\&
        \riskSubAreaDepartamentoTecnologias 
    },
    riskSubAreaDependencies={ \riskPOIDepartamentoTecnologias  },
    riskVulnDetails={
        La denegación de servicio es un tipo de ataque cuyo fin es eliminar temporal o parcialmente la disponibilidad de un servicio, usualmente por medios como ICMP Flood.
    },
    riskThreatDetails={
        Si bien la aplicación está funcionando con las últimas versiones de PHP y de MYQSL disponibles, la infraestructura al ser local y no contar con un WAF, no hay filtro respecto a las peticiones que son resueltas en el servidor. Debido a esto, en caso de llegar un número importante de peticiones las cuales no pudiesen resolverse simultaneamente, podría ocurrir un problema de overflow de memoria colapsando el proceso.

        Cabe destacar que esto también puede ocurrir de manera orgánica en situaciones de alta demanda. Y debido a los acuerdos internos de desarrollo estandarizado, está presente en todas las plataformas desarolladas para uso interno.
    },
    riskResponse={ \riskRepsonseCorrect },
    riskApproval={ \riskPOIDepartamentoTecnologias }
]{}


\riskItemTable[
    riskTitle={Ejecución remota de código},
    riskAuthor={\riskAuthorErik},
    riskDate={11/06/2020},
    riskDescription={
        Al materializarse este riesgo, el atacante ejecuta código en el navegador del cliente sin previo consentimiento.},
    riskResourceOwner={\riskPOIDepartamentoTecnologias},
    riskAssociatedProcess={
        Nivel general, \riskProcessUsoDeSitioWeb
        % \threatInterest \\&
        % \threatHumanIntervention
    },
    riskSubArea={ 
        \riskSubAreaAll
    },
    riskSubAreaDependencies={ \riskPOIDepartamentoTecnologias  },
    riskVulnDetails={
        La ejecución remota de código permite que un usuario no autorizado ejecute instrucciones en otro equipo.
    },
    riskThreatDetails={
        Esta amenaza está atribuida a CVE-2019-9787, el cual especifica una vulnerabilidad sobre la ejecución remota de código por medio de CRSRF. Este tipo de ataque fuerza al usuario a ejecutar código utilizando sus credenciales ya cargadas en la aplicación.
    },
    riskResponse={ \riskRepsonsePrevent },
    riskApproval={ \riskPOIDepartamentoTecnologias }
]{}

\riskItemTable[
    riskTitle={Manipulación de redirecciones},
    riskAuthor={\riskAuthorErik},
    riskDate={11/06/2020},
    riskDescription={
        Al materializarse este riesgo, el atacante fuerza la redirección a un sitio externo.},
    riskResourceOwner={\riskPOIDepartamentoTecnologias},
    riskAssociatedProcess={
        Nivel general, \riskProcessUsoDeSitioWeb \\&
        Nivel general, \riskProcessConfiabilidadDeSitioWeb
    },
    riskSubArea={ 
        \riskSubAreaAll
    },
    riskSubAreaDependencies={ \riskPOIDepartamentoTecnologias  },
    riskVulnDetails={
        La ejecución remota de código permite que un usuario no autorizado ejecute instrucciones en otro equipo.
    },
    riskThreatDetails={
        Esta amenaza está atribuida a CVE-2019-16220, el cual especifica una vulnerabilidad sobre la ejecución remota de código por medio de CRSRF. Este tipo de ataque fuerza al usuario a ejecutar código utilizando sus credenciales ya cargadas en la aplicación.
    },
    riskResponse={ \riskRepsonseMitigate },
    riskApproval={ \riskPOIDepartamentoTecnologias }
]{}


\riskItemTable[
    riskTitle={Ejecución de malware en servidor principal},
    riskAuthor={\riskAuthorErik},
    riskDate={11/06/2020},
    riskDescription={
        Al materializarse este riesgo, el servidor principal de la municipalidad queda comprometido.},
    riskResourceOwner={\riskPOIDepartamentoTecnologias},
    riskAssociatedProcess={
    },
    riskSubArea={ 
        \riskSubAreaAll
    },
    riskSubAreaDependencies={ \riskPOIDepartamentoTecnologias  },
    riskVulnDetails={
        Ataques por randomware, gusanos, trojanos, etc.
    },
    riskThreatDetails={
        Debido al alto número de vulnerabilidades presentes en el sistema operativo, es posible que la materalización de un riesgo en un equipo de una red adyacente pueda propagar procesos de terceros y estos comprometan el servidor principal.
    },
    riskResponse={ \riskRepsonsePrevent },
    riskApproval={ \riskPOIDepartamentoTecnologias }
]{}


\section{Riesgos generales asociados a ingenería social}

\riskItemTable[
    riskTitle={Phishing},
    riskAuthor={\riskAuthorErik},
    riskDate={11/06/2020},
    riskDescription={
        Al materializarse este riesgo, un usuario ingresa información institucional a un sitio falso.},
    riskResourceOwner={\riskPOIDepartamentoTecnologias},
    riskAssociatedProcess={
    },
    riskSubArea={ 
        \riskSubAreaAll
    },
    riskSubAreaDependencies={ \riskPOIDepartamentoTecnologias  },
    riskVulnDetails={
        Un usuario recibe un correo con un mensaje falso pero con apariencia visual creible, de esta manera para tentar al usuario a ejecutar alguna acción que pueda comprometer la seguridad, ya sea filtrando credenciales o información sensible.
    },
    riskThreatDetails={
        Un ataque de Phishing implica la personificación de otro individuo o entidad, la cual actua como emisor de un mensaje el cual puede ser de interés del usuario. En este caso la apuesta es que el lector del correo hará caso del call to action antes de verificar la veracidad del contenido, por lo que este tipo de ataques está dirigido a un público no tecnico.
    },
    riskResponse={ \riskRepsonsePrevent },
    riskApproval={ \riskPOIDepartamentoTecnologias }
]{}



\section{Riesgos generales asociados a activos estándares}

\subsection{Objetivos de la política}
Esta política está diseñada para ayudar al proceso de estandarización y saneamiento de los problemas detectados durante la realización de este informe. Se espera que estas políticas sean leídas por el personal administrativo y general para informarse de los requisitos obligatorios para proteger la información del municipio.

Estas políticas están disponibles en nuestro sitio web oficial de la municipalidad como también son entregadas como documento adjunto al contrato de trabajo y debe ser entregada una copia firmada para notificar la lectura y aceptancia de estas políticas.

\subsection{Declaración de autoridad y alcance}
Esta politica es diseñada por parte del Departamento de Tecnologías de la Información del municipio para velar por la seguridad de TI desde un punto de vista general. Se espera que esta política establesca un lineamiento base respecto al uso de recursos y dispositivos a lo largo de la organización con tal de reducir el impacto de las vulnerabilidades de origen humano.

Dentro de esta política no se especificarán aspectos detallados sobre las contrataciones, mecanismos de licitación, mecanismos de auditoria ni ningún otro aspecto administrativo salvo el uso correcto de bienes y servicios por parte de los funcionarios, sin embargo, esta política base no excluye del efecto de otras políticas vigentes dentro de cada departamento, oficina y secretaria.

\subsection{Política de uso aceptable}
\begin{itemize}
    \item El usuario tiene prohibido extraer información de su dispositivo personal a otro medio físico.
    \item El usuario no puede utilizar aplicaciones externas o fuera del ecosistema utilizado por la organización.
    \item No esta permitido realizar cambios al sistema operativo, hardware o estructura interna del dispositivo institucional entregado .
\end{itemize}

\subsection{Política de identificación y autenticación}
\begin{itemize}
    \item Para poder ingresar a sus dispositivos será necesaria la utilización de una contraseña segura, alfanumérica con al menos un caracter especial de un largo no menor a 10 caracteres, como también debera utilizar en conjunto autenticación de doble factor biométrico.
    \item Todos los dispositivos no deben presentar medios para poder establecer cambios al sistema operativo ni a su entorno. 
    \item Todos los dispositivos deben contar con encendido automático cada cierto número de horas.
\end{itemize}

\subsection{Política de acceso a internet}
\begin{itemize}
    \item Todas las conexiones hacia los recursos de la empresa deben ser realizadas por medio de una VPN y por medio de conexión cableada mediante Ethernet.
    \item No está permitido el uso de redes inalámbricas, salvo la institucional, la cual para su uso se requiere un certificado el cual es provisto por el departamento de tecnologías de la información.
    \item No está permitido el acceso a sitios no autorizados u de ocio por medio de la red institucional.
\end{itemize}

\subsection{Política de acceso}
\begin{itemize}
    \item Para poder iniciar sesión en los recursos en linea de la empresa es necesario proveer de la contraseña de la cuenta corporativa y una verificación de dos pasos por medio de Office365 S2-E5.
    \item Las conexiones a recursos de la empresa solo pueden ser ejecutadas por medio de las aplicaciones stand-alone correspondientes.
    \item Todos los accesos a equipos corporativos deben hacerse por medio de Active Directory.
\end{itemize}


\subsection{Política de acceso remoto}
\begin{itemize}
    \item El acceso remoto solo está permitido por medio de el uso conjunto de una VPN y la red institucional.
    \item Las conexiones a recursos de la empresa solo pueden ser ejecutadas por medio de las aplicaciones stand-alone correspondientes.
    \item En caso de situaciones de fuerza mayor, las restricciones anteriores pueden ser levantadas y/o controladas.
\end{itemize}

\subsection{Políticas del manejo de incidentes}
En caso de robo o pérdida del equipo:
\begin{itemize}
    \item Se debe dar anuncio inmediato a el Departamento de Tecnologías de la Información para iniciar el procedimiento de traza y de ser necesario de borrado de información.
    \item El Departamento de Tecnologías de la Información debe dar anuncio inmediato del evento a las autoridades pertinentes e iniciar los trámites necesarios para la obtención del equipo. El individuo afectado también deberá concurrir junto con el Departamento de Tecnologías de la Información para tales efectos.
    \item La gerencia debera gestionar la entrega de un nuevo dispositivo en un plazo no mayor a 5 días hábiles e iniciará un procedimiento de eliminación e inhabilitación remota por medio del encendido automático.
\end{itemize}

En caso de mal uso:
\begin{itemize}
    \item En caso de detectarse mal uso el Departamento de Tecnologías de la Información deberá emitir una carta de amonestación a la persona, la cual deberá acusar su recibo con su jefe directo. Esta carta solo será entregada como máximo tres veces, luego de esto será considerado una violación al código de seguridad de la empresa.
    \item En caso de reiteradas violaciones al código de conducta, el caso será derivado a la gerencia de recursos humanos para sus respectivas medidas o sanciones que estimen pertinentes.
\end{itemize}

En caso de ser victima de un ataque informático u sospecha del mismo:
\begin{itemize}
    \item La persona afectada deberá dar cuenta al Departamento de Tecnologías de la Información de sus actividades y sospechas para que este pueda investigar en el problema.
    \item Se deberá hacer un security assesment para poder identificar los posibles afectados y comenzar con mitigaciones.
    \item De ser necesario afectado deberá hacer entrega de sus equipos digitales para ser examinados por el Departamento de Tecnologías de la Información y esclarecer el origen del problema y sus mitigaciones.
    \item El Departamento de Tecnologías de la Información deberá pasado el proceso de examen entregar los equipos limpios a su usuario para que este pueda retomar sus funciones.
    \item El incidente deberá ser archivado dentro de los registros del Departamento de Tecnologías de la Información. Dependiendo de la gravedad del incidente, este reporte puede ser elevado a otras unidades de la organización o bien a entes gubernamentales en acuerdo a la ley de protección de datos vigente actualmente en el país.
\end{itemize}