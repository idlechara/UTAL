% %% dependencies
% %% Approbal 
\newcommand{\riskPOIAsuntosMunicipales}{
	Jefe de Oficina de Asuntos Municipales.
}
\newcommand{\riskPOIConsejoMunicipal}{
	Jefe de Consejo municipal.
}
\newcommand{\riskPOISecretariaMunicipal}{
	Jefe de Secretaría Municipal
}
\newcommand{\riskPOIDireccionControl}{
	Jefe de Dirección de Control.
}
\newcommand{\riskPOIDireccionComunicaciones}{
	Jefe de Dirección de Comunicaciones.
}
\newcommand{\riskPOIDireccionAsistenciaJuridica}{
	Jefe de Dirección de Asesoría Jurídica.
}
\newcommand{\riskPOIDepartamentoTecnologias}{
	Jefe de Departamento de Tecnologías de la Información.
}
\newcommand{\riskPOISecretariaComunalDePlanificacion}{
	Jefe de Secretaría Comunal de Planificación.
}
\newcommand{\riskPOIDireccionTransitoTransportePublico}{
	Jefe de Dirección de Tránsito y Transporte Público.|
}
\newcommand{\riskPOIDesarolloComunitario}{
	Jefe de Dirección de Desarrollo Comunitario
}
\newcommand{\riskPOIDireccionObrasMunicipales}{
	Jefe de Dirección de Obras Municipales
}
\newcommand{\riskPOIDireccionAdministracionFinanzas}{
	Jefe de Dirección de Administración y Finanzas
}
\newcommand{\riskPOIDireccionFiscalizacion}{
	Jefe de Dirección de Fiscalización
}
\newcommand{\riskPOIDireccionDeOperaciones}{
	Jefe de Dirección de Operaciones y Servicios Urbanos
}
\newcommand{\riskPOIDireccionCulturaTurismo}{
	Jefe de Dirección de Cultura y Turismo
}
\newcommand{\riskPOIDireccionSeguridadVecinal}{
	Jefe de Dirección de Seguridad Vecinal y Resguardo
}
\newcommand{\riskPOIDireccionGestionPersonas}{
	Jefe de Dirección de Gestión de Personas
}
\newcommand{\riskPOIDireccionSalud}{
	Jefe de Dirección de Salud
}
\newcommand{\riskPOIDireccionEducacion}{
	Jefe de Dirección de Educación
}




% 
%% Authors
\newcommand{\riskAuthorErik}{
	Erik Regla
}\newcommand{\riskAuthorAriel}{
	Ariel Valenzuela
}\newcommand{\riskAuthorMiguel}{
	Miguel Jorquera
}

%% Response 
\newcommand{\riskRepsonsePrevent}{
	PREVENIR
}\newcommand{\riskRepsonseMitigate}{
	MITIGAR
}\newcommand{\riskRepsonseCorrect}{
	CORREGIR
}\newcommand{\riskRepsonseCompensate}{
	COMPENSAR
}

% %% Processes 
\newcommand{\riskProcessGetAppointment}{
	Obtención de hora de atención.
}\newcommand{\riskProcessNewSuggestion}{
	Entrega de sugerencias.
}\newcommand{\riskProcessNewClaim}{
	Entrega de reclamos.
}\newcommand{\riskProcessGetInformation}{
	Solicitud de información.
}\newcommand{\riskProcessUsoDeSitioWeb}{
	Disponibilidad del servicio.
}\newcommand{\riskProcessConfiabilidadDeSitioWeb}{
	Confiabilidad del servicio.
}\newcommand{\riskProcessIntegridad}{
	Integridad de datos.
}\newcommand{\riskProcessResguardoInformacionPersonal}{
	Resguardo de información personal.
}\newcommand{\riskProcessResguardoInformacionInstitucional}{
	Resguardo de información institucional.
}\newcommand{\riskProcessResguardoPlataforma}{
	Fiabilidad de plataforma
}
% 
\makeatletter
\define@key{riskItem}{riskTitle}{\def\risk@riskTitle{#1}}
\define@key{riskItem}{riskAuthor}{\def\risk@riskAuthor{#1}}
\define@key{riskItem}{riskDate}{\def\risk@riskDate{#1}}
\define@key{riskItem}{riskDescription}{\def\risk@riskDescription{#1}}
\define@key{riskItem}{riskResourceOwner}{\def\risk@riskResourceOwner{#1}}
\define@key{riskItem}{riskAssociatedProcess}{\def\risk@riskAssociatedProcess{#1}}
\define@key{riskItem}{riskSubArea}{\def\risk@riskSubArea{#1}}
\define@key{riskItem}{riskSubAreaDependencies}{\def\risk@riskSubAreaDependencies{#1}}
\define@key{riskItem}{riskVulnDetails}{\def\risk@riskVulnDetails{#1}}
\define@key{riskItem}{riskThreatDetails}{\def\risk@riskThreatDetails{#1}}
\define@key{riskItem}{riskResponse}{\def\risk@riskResponse{#1}}
\define@key{riskItem}{riskApproval}{\def\risk@riskApproval{#1}}

\setkeys{riskItem}{
riskTitle=riskTitle,
riskAuthor=riskAuthor,
riskDate=riskDate,
riskDescription=riskDescription,
riskResourceOwner=riskResourceOwner,
riskAssociatedProcess=riskAssociatedProcess,
riskSubArea=riskSubArea,
riskSubAreaDependencies=riskSubAreaDependencies,
riskVulnDetails=riskVulnDetails,
riskThreatDetails=riskThreatDetails,
riskResponse=riskResponse,
riskApproval=riskApproval
}

\newcommand{\riskItemTable}[2][]{%
    \setkeys{riskItem}{#1}% Set the keys

    \begin{tabularx}{\linewidth}{|r|X|}
        \hline
        \textbf{Título de Riesgo}      & \risk@riskTitle  \\ 
        \hline
        \textbf{Autor} & \risk@riskAuthor  \\ 
        \hline
        \textbf{Fecha de Levantamiento}   & \risk@riskDate  \\ 
        \hline
        \textbf{Descripción}   & \risk@riskDescription  \\ 
        \hline
        \textbf{Dueño del Activo} & \risk@riskResourceOwner  \\ 
        \hline
        \textbf{Proceso}      & \risk@riskAssociatedProcess  \\ 
        \hline
        \textbf{Sub Área} & \risk@riskSubArea  \\ 
        \hline
        \textbf{Dependencia}   & \risk@riskSubAreaDependencies  \\ 
        \hline
        \textbf{Detalle de la Vulnerabilidad}   & \risk@riskVulnDetails  \\ 
        \hline
        \textbf{Detalle de la amenaza} & \risk@riskThreatDetails  \\ 
        \hline
        \textbf{Respuesta}   & \risk@riskResponse  \\ 
        \hline
        \textbf{Aprobación} & \risk@riskApproval  \\ 
        \hline
        %\textbf{Valoración}  &  \makecell[l]{Confidencialidad: #6\\ Integridad: #7\\ Disponibilidad: #8}\\
        %\hline
        %\textbf{\makecell[r]{Vulnerabilidades y \\Amenazas}}  &  \makecell[l]{#9}\\ 
        %\hline
    \end{tabularx}
    #2
}

\makeatother
% %% SubAreas y dependencias
\newcommand{\riskSubAreaAsuntosMunicipales}{
	Oficina de Asuntos Municipales.
}
\newcommand{\riskSubAreaConsejoMunicipal}{
	Consejo municipal.
}
\newcommand{\riskSubAreaSecretariaMunicipal}{
	Secretaría Municipal
}
\newcommand{\riskSubAreaDireccionControl}{
	Dirección de Control.
}
\newcommand{\riskSubAreaDireccionComunicaciones}{
	Dirección de Comunicaciones.
}
\newcommand{\riskSubAreaDireccionAsistenciaJuridica}{
	Dirección de Asesoría Jurídica.
}
\newcommand{\riskSubAreaDepartamentoTecnologias}{
	Departamento de Tecnologías de la Información.
}
\newcommand{\riskSubAreaSecretariaComunalDePlanificacion}{
	Secretaría Comunal de Planificación.
}
\newcommand{\riskSubAreaDireccionTransitoTransportePublico}{
	Dirección de Tránsito y Transporte Público.|
}
\newcommand{\riskSubAreaDesarolloComunitario}{
	Dirección de Desarrollo Comunitario
}
\newcommand{\riskSubAreaDireccionObrasMunicipales}{
	Dirección de Obras Municipales
}
\newcommand{\riskSubAreaDireccionAdministracionFinanzas}{
	Dirección de Administración y Finanzas
}
\newcommand{\riskSubAreaDireccionFiscalizacion}{
	Dirección de Fiscalización
}
\newcommand{\riskSubAreaDireccionDeOperaciones}{
	Dirección de Operaciones y Servicios Urbanos
}
\newcommand{\riskSubAreaDireccionCulturaTurismo}{
	Dirección de Cultura y Turismo
}
\newcommand{\riskSubAreaDireccionSeguridadVecinal}{
	Dirección de Seguridad Vecinal y Resguardo
}
\newcommand{\riskSubAreaDireccionGestionPersonas}{
	Dirección de Gestión de Personas
}
\newcommand{\riskSubAreaDireccionSalud}{
	Dirección de Salud
}
\newcommand{\riskSubAreaDireccionEducacion}{
	Dirección de Educación
}






\newcommand{\riskSubAreaOIRS}{
	Informaciones, Reclamos y Sugerencias.
}
\newcommand{\riskSubAreaNone}{
	Ninguna.
}
\newcommand{\riskSubAreaAll}{
	Todas.
}

% %%Begin of content

% %% tengo que agregar las siguientes plataformas

Remuneraciones
Registro comunal de permisos de circulación
Licencias de conducir
Patentes municipalesControl de bienes
Adquisisión y control de bodegas
Banco de datos propiedas comunales
Oficina de partes
DepartamentoTecnologiasJuzgados de polícia local
mantencion y administración de cartografia digital


Administracion red departamente licencias de conducir
administracion red departamento de juzgados de policia local
partes empadronados departamento de seguridad ciudadana
administración red departamento patentes municpales y tesorería
administración red del departamento de seguridad ciudadana
administración de dirección de obras municipales
plataforma de egresos
administracion de sistemas de desarrollo comunitario
sistemas computaciones por internet y sistemas sociales


administración de sistemas de egresos, recursos humanos y remuneraciones **
Contabilidad gubernamental **

% %% Sistema contabilidad gubernamental
%     Sistema  contabilidad  gubernamental.
%     Ingreso de cuentas  contables,  programas  y centro  de costos.
%     Ingreso   de  tablas   para  el  funcionamiento    del  sistema   (meses,áreas,  tipo  de  comprobantes   contables,  tipo  de  documentos,   tablacentro  costos,  programas,  parámetros  y proveedores).
%     Ingreso de presupuesto  inicialymodificaciones   presupuestarias.
%     Ingreso d eobligaciones(contratos,orden de compra,adjudicaciones  y factibilidades).
%     Ingreso  de devengados  por proveedor  (facturas).
%     Confección  de órdenes  de pago.
%     Ingreso y contabilización   de documentos  contables,  rendiciones  decuentas.

% %% Sistema de tesoreria municipal
% Boletas  de garantía.
%     Mantención  de garantías.
%     Consulta  documento  en garantía.
%     Ingreso de contratos
% Egresos.
%     Emisión  de cheques  de distintas  cuentas  corrientes.
%     Emisión   de  listados   de  información,   como   cuenta   corriente   deproveedor.
%     Generación  de listado  de conciliaciones   bancarias  y retenciones  deimpuesto.
%     Contabilización  de movimientos  contables
% Ingresos.
%     Apertura  y cierre de cajas.
%     Anulación  de ingresos.
%     Cuadraturas  de cajas.
%     Contabilización  de ingresos.
%     Conciliación  de ingresos.
%     Pagos a través de Internet.
%     Emisión  informes  varios.
%     Consulta  de recaudación  por cajas.

% %% Sistema patentes comerciales
%     Consulta  de patentes.
%     Listar patentes  CIPA, según tipo.
%     Administrar  solicitud  de patente.
%     Mantención  del maestro  de patentes.
%     Cálculo  de patentes.
%     Anulación  de patentes  y/o giros.


% %% Sistema permisos de circulación
%     Generación  de giro para pago de permisos  de circulación.
%     Generación  de duplicado  de permisos  de circulación.
%     Emisión  de giros de fondos  a terceros
%     Bloqueo  por sistema  de placas  patentes.
%     Consultas   de  pagos  años  anteriores,   de  registro  de  multas,   deincorporaciones  y de traslados.
%     Generación  de giros de sellos.
%     Mantención  de traslado.
%     Asignación  de código de S.1.1.
%     Anulación  de giros mal emitidos

\section{Riesgos asociados a factores no tecnológicos}
%% desastres naturales
%% Multas por servicios operaciones (como agua, luz)
%% Pérdida de soporte de proyectos licitados
%% Fin de facturaciones
%% Desastres lógicos
%% Divulgación y copiado de informacion
%% falta de stock de recursos tecnológicos
%% candados de seguridad

\section{Riesgos asociados a procesos municpales}
%% Falta de documentación e implantación de políticas para envío de correos masivos
%% Falta de punto de contacto
%% Roles no definidos
%% Falta de monitoreo



\section{Riesgos asociados procesos de atención municipal}
%%Obtención y/o renovación de permisos de circulación sin ingreso o acreditación física y online de datos sobre propietario, placa patente única, seguro obligatorio, revisión técnica y/o multas impagas

%%Obtención y/o renovación de patentes municipales sin ingreso y/o acreditación de dataos del contribuyente, propiedad, sucursales y datos del servicio de impuestos internos

%% Emisión de decretos de pago sin registrar datos y/o sin comprobantes

%% Decretos de pago imputados a cuentas presupuestarias que no corresponden

%% Emisión de cheque individual sin consultar datos en sistema de contabilidad gubernamental

%% Recepción de pagos con cálculos de intereses y multas fuera de período

\section{Riesgos asociados a control del personal}
%% Quiebre autenticación de llave seguridad
%% quiebre autenticación de tarjeta magnética
%% Copia de llaves


\section{Riesgos asociados de índole técnica}
%% Interoperabilidad con estándar de norma técnica para los Órganos de la Administración del Estado
%% Integridad de información por estructura de datos
%% Carencia de licencias

\riskItemTable[
    riskTitle={Dependencia de licencias},
    riskAuthor={\riskAuthorErik},
    riskDate={12/06/2020},
    riskDescription={ Al materializarse este riesgo, la invalidación de una licencia podría provocar problemas de sguridad o bien interrupciones en la disponibilidad de un servicio. },
    riskResourceOwner={\riskPOIGeneral},
    riskAssociatedProcess={
        \riskProcessUsoDeSitioWeb
    },
    riskSubArea={ 
        \riskSubAreaAll
    },
    riskSubAreaDependencies={ \riskPOIDepartamentoTecnologias, \riskPOIEmpresasExternas  },
    riskVulnDetails={
        Para el caso de software que opera con licencias (Office 365 por ejemplo), la caducidad de las mismas puede generar interrupciones o bien dejar de dar soporte a nuevas amenazas
    },
    riskThreatDetails={
        Actualmente debido al convenio con Microsoft vigente por parte del gobierno actual, muchos softwares están a merced de que estas licencias no caduquen. Esto podría producirse por múltiples factores, no disponibilidad del retailer, cambio de versiones, no soporte de cambios, olvido de pagos, etc.
    },
    riskResponse={ \riskRepsonsePrevent },
    riskApproval={ \riskPOIDepartamentoTecnologias, \riskPOIPago, \riskPoiRevisionProcesos }
]{}

\riskItemTable[
    riskTitle={Transaccion rota},
    riskAuthor={\riskAuthorErik},
    riskDate={12/06/2020},
    riskDescription={ Al materializarse este riesgo, es posible leer la información directamente desdfe el medio en que se encuentra sin ninguna barrera de seguridad },
    riskResourceOwner={\riskPOIDepartamentoTecnologias},
    \riskProcessIntegridad
    riskAssociatedProcess={
    },
    riskSubArea={ 
        \riskSubAreaAll
    },
    riskSubAreaDependencies={ \riskPOIDepartamentoTecnologias, \riskPOIEmpresasExternas  },
    riskVulnDetails={
        Actualmente no hay ningun mecanismo de respaldo para operaciones de índole transaccional, lo cual puede provocar pérdidas de información.
    },
    riskThreatDetails={
        Al no haber un registro de comunicaciones llevadas a cabo de manera transaccional, en el momento de existir peticiones a los distintos serivcios que pudan provocar un conflicto, este puede resultar en inconsistencias, corrupción y pérdida de datos.

        Sin embargo, Dado que los riesgos son mínimos de por el momento y no ha ocurrido no se le da mayor importancia, a excepción del sistema de pago.
    },
    riskResponse={ \riskRepsonseCompensate },
    riskApproval={ \riskPOIDepartamentoTecnologias }
]{}

\riskItemTable[
    riskTitle={Falta de encriptado},
    riskAuthor={\riskAuthorErik},
    riskDate={12/06/2020},
    riskDescription={ Al materializarse este riesgo, es posible leer la información directamente desdfe el medio en que se encuentra sin ninguna barrera de seguridad },
    riskResourceOwner={\riskPOIDepartamentoTecnologias},
    riskAssociatedProcess={
        \riskProcessResguardoInformacionPersonal \\&
        \riskProcessResguardoInformacionInstitucional \\&
    },
    riskSubArea={ 
        \riskSubAreaAll
    },
    riskSubAreaDependencies={ \riskPOIDepartamentoTecnologias, \riskPOIEmpresasExternas  },
    riskVulnDetails={
        Una auditoria realizada por contraloría reveló que no existe encruiptación de los datos almacenados digitalmente salvo en la capa de transporte.
    },
    riskThreatDetails={
        La falta de encriptación puede producir fuga de información sensible.
    },
    riskResponse={ \riskRepsonseCorrect },
    riskApproval={ \riskPOIAlcalde }
]{}


\riskItemTable[
    riskTitle={Residuos de información},
    riskAuthor={\riskAuthorErik},
    riskDate={12/06/2020},
    riskDescription={ Al materializarse este riesgo, información que haya sido borrada sigue disponible dentro de una base de datos sin ser detectada },
    riskResourceOwner={\riskPOIDepartamentoTecnologias},
    riskAssociatedProcess={
        \riskProcessResguardoInformacionPersonal \\&
        \riskProcessResguardoInformacionInstitucional \\&
    },
    riskSubArea={ 
        \riskSubAreaAll
    },
    riskSubAreaDependencies={ \riskPOIDepartamentoTecnologias, \riskPOIEmpresasExternas  },
    riskVulnDetails={
        Una auditoria realizada por contraloría reveló que no existen protocolos de eliminación de la información de manera interna y esta tampoco forma parte de los servicios contratados.
    },
    riskThreatDetails={
        Al no existir un protocolo de eliminado de información claro, es altamente probable que la información no pueda ser eliminada de manera efectiva ya sea de plataformas, dispositivos, medios extraibles, etc. Este problema aplica también a los archivos físicos que no cuenten con respaldo y que dentro de las operaciones vigentes consideren su eliminación. 
    },
    riskResponse={ \riskRepsonseCorrect },
    riskApproval={ \riskPOIAlcalde }
]{}

\riskItemTable[
    riskTitle={Residuos de información},
    riskAuthor={\riskAuthorErik},
    riskDate={12/06/2020},
    riskDescription={ Al materializarse este riesgo, información que haya sido borrada sigue disponible dentro de una base de datos sin ser detectada },
    riskResourceOwner={\riskPOIDepartamentoTecnologias},
    riskAssociatedProcess={
        \riskProcessResguardoInformacionPersonal \\&
        \riskProcessResguardoInformacionInstitucional \\&
    },
    riskSubArea={ 
        \riskSubAreaAll
    },
    riskSubAreaDependencies={ \riskPOIDepartamentoTecnologias, \riskPOIEmpresasExternas  },
    riskVulnDetails={
        Una auditoria realizada por contraloría reveló que no existen protocolos de eliminación de la información de manera interna y esta tampoco forma parte de los servicios contratados.
    },
    riskThreatDetails={
        Al no existir un protocolo de eliminado de información claro, es altamente probable que la información no pueda ser eliminada de manera efectiva ya sea de plataformas, dispositivos, medios extraibles, etc. Este problema aplica también a los archivos físicos que no cuenten con respaldo y que dentro de las operaciones vigentes consideren su eliminación. 
    },
    riskResponse={ \riskRepsonseCorrect },
    riskApproval={ \riskPOIAlcalde }
]{}

\riskItemTable[
    riskTitle={Mantenimiento preventivo externalizado ejecutado deficientemente},
    riskAuthor={\riskAuthorErik},
    riskDate={12/06/2020},
    riskDescription={ Al materializarse este riesgo, las plataformas sujetas a mantenimiendo por parte de una empresa externa podrían quedar expuestas a vulnerabilidades },
    riskResourceOwner={\riskPOIEmpresasExternas},
    riskAssociatedProcess={
        \riskProcessResguardoInformacionPersonal \\&
        \riskProcessResguardoInformacionInstitucional \\&
        \riskProcessResguardoPlataforma
    },
    riskSubArea={ 
        \riskSubAreaAsuntosMunicipales
        \riskSubAreaPago
        \riskSubAreaAuditoria
    },
    riskSubAreaDependencies={ \riskPOIDepartamentoTecnologias, \riskPOIEmpresasExternas  },
    riskVulnDetails={
        Mantenciones negligentes, omitidas, incompletas.
    },
    riskThreatDetails={
        Una mantención negligente de las plataformas puede llevar a un uso malicioso de estas, las cuales pueden perjudicar enormemente el servicio entregado por la municipalidad como también poner en riesgo los datos disponibles en esta. Actualmente debido a la normativa actual, todas las aplicaciones están alojadas en servidores de la municipalidad, sin embargo, no implica que el código esté necesariamente abierto o que el personal propio del departamento de tecnologías pueda tener el conocimiento suficiente sobre este para tomar control completo.
    },
    riskResponse={ \riskRepsonsePrevent },
    riskApproval={ \riskPOIDepartamentoTecnologias }
]{}

\riskItemTable[
    riskTitle={Denegación de servicio},
    riskAuthor={\riskAuthorErik},
    riskDate={11/06/2020},
    riskDescription={
        Al materializarse este riesgo, el sitio web de la municipalidad deja de quedar disponible para todo público.},
    riskResourceOwner={\riskPOIDepartamentoTecnologias},
    riskAssociatedProcess={
        Nivel general, \riskProcessUsoDeSitioWeb
    },
    riskSubArea={ 
        \riskSubAreaDepartamentoTecnologias 
    },
    riskSubAreaDependencies={ \riskPOIDepartamentoTecnologias, \riskPOIEmpresasExternas  },
    riskVulnDetails={
        La denegación de servicio es un tipo de ataque cuyo fin es eliminar temporal o parcialmente la disponibilidad de un servicio, usualmente por medios como ICMP Flood.
    },
    riskThreatDetails={
        Si bien la aplicación está funcionando con las últimas versiones de PHP y de MYQSL disponibles, la infraestructura al ser local y no contar con un WAF, no hay filtro respecto a las peticiones que son resueltas en el servidor. Debido a esto, en caso de llegar un número importante de peticiones las cuales no pudiesen resolverse simultaneamente, podría ocurrir un problema de overflow de memoria colapsando el proceso.

        Cabe destacar que esto también puede ocurrir de manera orgánica en situaciones de alta demanda. Y debido a los acuerdos internos de desarrollo estandarizado, está presente en todas las plataformas desarolladas para uso interno.
    },
    riskResponse={ \riskRepsonseCorrect },
    riskApproval={ \riskPOIDepartamentoTecnologias }
]{}


\riskItemTable[
    riskTitle={Ejecución remota de código},
    riskAuthor={\riskAuthorErik},
    riskDate={11/06/2020},
    riskDescription={
        Al materializarse este riesgo, el atacante ejecuta código en el navegador del cliente sin previo consentimiento.},
    riskResourceOwner={\riskPOIDepartamentoTecnologias},
    riskAssociatedProcess={
        Nivel general, \riskProcessUsoDeSitioWeb
        % \threatInterest \\&
        % \threatHumanIntervention
    },
    riskSubArea={ 
        \riskSubAreaAll
    },
    riskSubAreaDependencies={ \riskPOIDepartamentoTecnologias  },
    riskVulnDetails={
        La ejecución remota de código permite que un usuario no autorizado ejecute instrucciones en otro equipo.
    },
    riskThreatDetails={
        Esta amenaza está atribuida a CVE-2019-9787, el cual especifica una vulnerabilidad sobre la ejecución remota de código por medio de CRSRF. Este tipo de ataque fuerza al usuario a ejecutar código utilizando sus credenciales ya cargadas en la aplicación.
    },
    riskResponse={ \riskRepsonsePrevent },
    riskApproval={ \riskPOIDepartamentoTecnologias }
]{}

\riskItemTable[
    riskTitle={Manipulación de redirecciones},
    riskAuthor={\riskAuthorErik},
    riskDate={11/06/2020},
    riskDescription={
        Al materializarse este riesgo, el atacante fuerza la redirección a un sitio externo.},
    riskResourceOwner={\riskPOIDepartamentoTecnologias},
    riskAssociatedProcess={
        Nivel general, \riskProcessUsoDeSitioWeb \\&
        Nivel general, \riskProcessConfiabilidadDeSitioWeb
    },
    riskSubArea={ 
        \riskSubAreaAll
    },
    riskSubAreaDependencies={ \riskPOIDepartamentoTecnologias  },
    riskVulnDetails={
        La ejecución remota de código permite que un usuario no autorizado ejecute instrucciones en otro equipo.
    },
    riskThreatDetails={
        Esta amenaza está atribuida a CVE-2019-16220, el cual especifica una vulnerabilidad sobre la ejecución remota de código por medio de CRSRF. Este tipo de ataque fuerza al usuario a ejecutar código utilizando sus credenciales ya cargadas en la aplicación.
    },
    riskResponse={ \riskRepsonseMitigate },
    riskApproval={ \riskPOIDepartamentoTecnologias }
]{}


\riskItemTable[
    riskTitle={Ejecución de malwarepor falta de software AV},
    riskAuthor={\riskAuthorErik},
    riskDate={11/06/2020},
    riskDescription={
        Al materializarse este riesgo, el servidor principal de la municipalidad queda comprometido.},
    riskResourceOwner={\riskPOIDepartamentoTecnologias},
    riskAssociatedProcess={
    },
    riskSubArea={ 
        \riskSubAreaAll
    },
    riskSubAreaDependencies={ \riskPOIDepartamentoTecnologias  },
    riskVulnDetails={
        Ataques por randomware, gusanos, trojanos, etc.
    },
    riskThreatDetails={
        Debido al alto número de vulnerabilidades presentes en el sistema operativo, es posible que la materalización de un riesgo en un equipo de una red adyacente pueda propagar procesos de terceros y estos comprometan el servidor principal.
    },
    riskResponse={ \riskRepsonsePrevent },
    riskApproval={ \riskPOIDepartamentoTecnologias }
]{}


\section{Riesgos generales asociados a ingenería social}

\riskItemTable[
    riskTitle={Phishing},
    riskAuthor={\riskAuthorErik},
    riskDate={11/06/2020},
    riskDescription={
        Al materializarse este riesgo, un usuario ingresa información institucional a un sitio falso.},
    riskResourceOwner={\riskPOIDepartamentoTecnologias},
    riskAssociatedProcess={
    },
    riskSubArea={ 
        \riskSubAreaAll
    },
    riskSubAreaDependencies={ \riskPOIDepartamentoTecnologias  },
    riskVulnDetails={
        Un usuario recibe un correo con un mensaje falso pero con apariencia visual creible, de esta manera para tentar al usuario a ejecutar alguna acción que pueda comprometer la seguridad, ya sea filtrando credenciales o información sensible.
    },
    riskThreatDetails={
        Un ataque de Phishing implica la personificación de otro individuo o entidad, la cual actua como emisor de un mensaje el cual puede ser de interés del usuario. En este caso la apuesta es que el lector del correo hará caso del call to action antes de verificar la veracidad del contenido, por lo que este tipo de ataques está dirigido a un público no tecnico.
    },
    riskResponse={ \riskRepsonsePrevent },
    riskApproval={ \riskPOIDepartamentoTecnologias }
]{}



\subsection{Objetivos de la política}
Esta política está diseñada para ayudar al proceso de estandarización y saneamiento de los problemas detectados durante la realización de este informe. Se espera que estas políticas sean leídas por el personal administrativo y general para informarse de los requisitos obligatorios para proteger la información del municipio.

Estas políticas están disponibles en nuestro sitio web oficial de la municipalidad como también son entregadas como documento adjunto al contrato de trabajo y debe ser entregada una copia firmada para notificar la lectura y aceptancia de estas políticas.

\subsection{Declaración de autoridad y alcance}
Esta politica es diseñada por parte del Departamento de Tecnologías de la Información del municipio para velar por la seguridad de TI desde un punto de vista general. Se espera que esta política establesca un lineamiento base respecto al uso de recursos y dispositivos a lo largo de la organización con tal de reducir el impacto de las vulnerabilidades de origen humano.

Dentro de esta política no se especificarán aspectos detallados sobre las contrataciones, mecanismos de licitación, mecanismos de auditoria ni ningún otro aspecto administrativo salvo el uso correcto de bienes y servicios por parte de los funcionarios, sin embargo, esta política base no excluye del efecto de otras políticas vigentes dentro de cada departamento, oficina y secretaria.

\subsection{Política de uso aceptable}
\begin{itemize}
    \item El usuario tiene prohibido extraer información de su dispositivo personal a otro medio físico.
    \item El usuario no puede utilizar aplicaciones externas o fuera del ecosistema utilizado por la organización.
    \item No esta permitido realizar cambios al sistema operativo, hardware o estructura interna del dispositivo institucional entregado .
\end{itemize}

\subsection{Política de identificación y autenticación}
\begin{itemize}
    \item Para poder ingresar a sus dispositivos será necesaria la utilización de una contraseña segura, alfanumérica con al menos un caracter especial de un largo no menor a 10 caracteres, como también debera utilizar en conjunto autenticación de doble factor biométrico.
    \item Todos los dispositivos no deben presentar medios para poder establecer cambios al sistema operativo ni a su entorno. 
    \item Todos los dispositivos deben contar con encendido automático cada cierto número de horas.
\end{itemize}

\subsection{Política de acceso a internet}
\begin{itemize}
    \item Todas las conexiones hacia los recursos de la empresa deben ser realizadas por medio de una VPN y por medio de conexión cableada mediante Ethernet.
    \item No está permitido el uso de redes inalámbricas, salvo la institucional, la cual para su uso se requiere un certificado el cual es provisto por el departamento de tecnologías de la información.
    \item No está permitido el acceso a sitios no autorizados u de ocio por medio de la red institucional.
\end{itemize}

\subsection{Política de acceso}
\begin{itemize}
    \item Para poder iniciar sesión en los recursos en linea de la empresa es necesario proveer de la contraseña de la cuenta corporativa y una verificación de dos pasos por medio de Office365 S2-E5.
    \item Las conexiones a recursos de la empresa solo pueden ser ejecutadas por medio de las aplicaciones stand-alone correspondientes.
    \item Todos los accesos a equipos corporativos deben hacerse por medio de Active Directory.
\end{itemize}


\subsection{Política de acceso remoto}
\begin{itemize}
    \item El acceso remoto solo está permitido por medio de el uso conjunto de una VPN y la red institucional.
    \item Las conexiones a recursos de la empresa solo pueden ser ejecutadas por medio de las aplicaciones stand-alone correspondientes.
    \item En caso de situaciones de fuerza mayor, las restricciones anteriores pueden ser levantadas y/o controladas.
\end{itemize}

\subsection{Políticas del manejo de incidentes}
En caso de robo o pérdida del equipo:
\begin{itemize}
    \item Se debe dar anuncio inmediato a el Departamento de Tecnologías de la Información para iniciar el procedimiento de traza y de ser necesario de borrado de información.
    \item El Departamento de Tecnologías de la Información debe dar anuncio inmediato del evento a las autoridades pertinentes e iniciar los trámites necesarios para la obtención del equipo. El individuo afectado también deberá concurrir junto con el Departamento de Tecnologías de la Información para tales efectos.
    \item La gerencia debera gestionar la entrega de un nuevo dispositivo en un plazo no mayor a 5 días hábiles e iniciará un procedimiento de eliminación e inhabilitación remota por medio del encendido automático.
\end{itemize}

En caso de mal uso:
\begin{itemize}
    \item En caso de detectarse mal uso el Departamento de Tecnologías de la Información deberá emitir una carta de amonestación a la persona, la cual deberá acusar su recibo con su jefe directo. Esta carta solo será entregada como máximo tres veces, luego de esto será considerado una violación al código de seguridad de la empresa.
    \item En caso de reiteradas violaciones al código de conducta, el caso será derivado a la gerencia de recursos humanos para sus respectivas medidas o sanciones que estimen pertinentes.
\end{itemize}

En caso de ser victima de un ataque informático u sospecha del mismo:
\begin{itemize}
    \item La persona afectada deberá dar cuenta al Departamento de Tecnologías de la Información de sus actividades y sospechas para que este pueda investigar en el problema.
    \item Se deberá hacer un security assesment para poder identificar los posibles afectados y comenzar con mitigaciones.
    \item De ser necesario afectado deberá hacer entrega de sus equipos digitales para ser examinados por el Departamento de Tecnologías de la Información y esclarecer el origen del problema y sus mitigaciones.
    \item El Departamento de Tecnologías de la Información deberá pasado el proceso de examen entregar los equipos limpios a su usuario para que este pueda retomar sus funciones.
    \item El incidente deberá ser archivado dentro de los registros del Departamento de Tecnologías de la Información. Dependiendo de la gravedad del incidente, este reporte puede ser elevado a otras unidades de la organización o bien a entes gubernamentales en acuerdo a la ley de protección de datos vigente actualmente en el país.
\end{itemize}