\subsection{Situación actual}

%TODO blabla

Durante noviembre del mes pasado, gracias a un informe de contraloría se han detectado las siguientes falencias en relación a los servicios contratados a empresas externas, ya que no se consideran cláusulas respecto a las siguientes operaciones:

\begin{itemize}
    \item Controles para asegurar la protección contra software malicioso
    \item PRocedimientos para determinar si ha ocurrido algún compromiso en los datos municipales
    \item Plan de contingencia para accesos indebidos, siniestros físicos y lógicos
    \item Restricciones de copiado y divulgacion de la información municipal
    \item Devolución o destrucción de la información y bienes, amparado por las regulaciones locales y término de la relación contractual.
    \item Posibilidad de auditar módulos del sistema administrativo municipal y los datos
\end{itemize}

Respecto a estos problemas, el alcalde ha mencionado que durante este año se deben de solucionar - parte del objetivo de este trabajo-

\subsection{Procesos}

Debido a que por ordenanzas del estado es necesario justificar el uso de recuros, la municipalidad hace uso de externalicaciones para la mayoría de sus recursos de software, siendo solo desarrollados o mantenidos de manera in-house las plataformas legadas o las que requieren atención crítica. Si bien el objetivo de la municipalidad es externalizar el desarrollo, gracias a los lineamientos descritos el año 2018\footnote{https://digital.gob.cl/doc/Guia_de_desarrollo_de_software_para_el_Estado.pdf}, los sistemas son alojados de manera interna y administrados internamente. Sin embargo, aún hay un par de sistemas legados, los cuales serán listados a continuación:


\subsubsection{Sistema contabilidad gubernamental}
Este sistema fue desarrllado por la Empresa Externa 1 durante el año 2010, por lo cual no está ligado directamente a la normativa de apertura de código digital. Las fuentes de esta plataforma están cerradas y la base de datos solo permite acceso al motor y su contenido pero no a la instancia de máquina virtual donde se aloja.

Funciones de este sistema:
\begin{itemize}
    \item Sistema  contabilidad  gubernamental.
    \item Ingreso de cuentas  contables,  programas  y centro  de costos.
    \item Ingreso   de  tablas   para  el  funcionamiento    del  sistema   (meses,áreas,  tipo  de  comprobantes contables,  tipo  de  documentos,   tablacentro  costos,  programas,  parámetros  y proveedores).
    \item Ingreso de presupuesto  inicialymodificaciones   presupuestarias.
    \item Ingreso de obligaciones(contratos,orden de compra,adjudicaciones  y factibilidades).
    \item Ingreso  de devengados  por proveedor  (facturas).
    \item Confección  de órdenes  de pago.
    \item Ingreso y contabilización   de documentos  contables,  rendiciones  decuentas.
\end{itemize}

\subsubsection{Sistema de tesoreria municipal}

Este es el sistema legado de mayor longevidad presente externalizado por la Empresa Externa 2, el cual data del año 1999. Sin embargo, debido a que el contrato con la empresa externa incluye actualizaciones continuas de la plataforma, esta se ha podido mantener vigente hasta el día de hoy sin mayores cambios visibles. De acuerdo a un informe de auditoria de contraloría realizado el año pasado, esta plataforma presenta problemas de interoperabilidad con los sistemas existentes, por lo cual un nuevo contrato es esperado de firmarse este año para iniciar un nuevo desarrollo de esta.


Funciones de este sistema:
\begin{itemize}
    \item{ Boletas  de garantía. 
        \begin{itemize}
        \item Mantención  de garantías.
        \item Consulta  documento  en garantía.
        \item Ingreso de contratos
        \end{itemize}
    }
\end{itemize}

\begin{itemize}
    \item{ Egresos. 
        \begin{itemize}
        \item Emisión  de cheques  de distintas  cuentas  corrientes.
        \item Emisión   de  listados   de  información,   como   cuenta   corriente   de proveedor.
        \item Generación  de listado  de conciliaciones   bancarias  y retenciones  de impuesto.
        \item Contabilización  de movimientos  contables
    \end{itemize}
}
\end{itemize}
\begin{itemize}
    \item{ Ingresos. 
        \begin{itemize}
        \item Apertura  y cierre de cajas.
        \item Anulación  de ingresos.
        \item Cuadraturas  de cajas.
        \item Contabilización  de ingresos.
        \item Conciliación  de ingresos.
        \item Pagos a través de Internet.
        \item Emisión  informes  varios.
        \item Consulta  de recaudación  por cajas.
    \end{itemize}
}
\end{itemize}
    

\subsection{Sistema patentes comerciales}
Este sistema acaba de ser contratado a la Empresa Externa 3 hace no mas de dos meses y su aprobación de uso fue entregada hace tres días atrás. Para mantener el uso con el archivo antiguo de la municipalidad, todos los registros físicos fueron migrados a sus versiones digitales para poder ser utilizados desde la nueva plataforma.

Funciones de este sistema:
\begin{itemize}
\item Consulta  de patentes.
\item Listar patentes  CIPA, según tipo.
\item Administrar  solicitud  de patente.
\item Mantención  del maestro  de patentes.
\item Cálculo  de patentes.
\item Anulación  de patentes  y/o giros.
\end{itemize}


\subsection{Sistema permisos de circulación}
Este es el segundo sistema externalizado a la Empresa Externa 1 y está en la misma situación que el sistema anteriormente mencionado para las patentes comerciales.

Funciones de este sistema:
\begin{itemize}
\item Generación  de giro para pago de permisos  de circulación.
\item Generación  de duplicado  de permisos  de circulación.
\item Emisión  de giros de fondos  a terceros
\item Bloqueo  por sistema  de placas  patentes.
\item Consultas   de  pagos  años  anteriores,   de  registro  de  multas,   deincorporaciones  y de traslados.
\item Generación  de giros de sellos.
\item Mantención  de traslado.
\item Asignación  de código de S.I.I.
\item Anulación  de giros mal emitidos
\end{itemize}