A continuación se listan los activos de caracter transversal, quiere decir, cuyo uso se extiende por más de una sola oficina.

\informationResource
{RTR\_PRINC\_001}
{Router principal Cisco 2901, gateway externo perteneciente a la municipalidad}
{Hardware TI}
{Sala de servidores - primer piso}
{Departamento de TI}
{1}{5}{5}
{
	\threatCVE{CVE-2013-1241}{Autenticación inválida en cabeceras del módulo ISM} \\ &
	\threatResourceLost \\ &
	\threatRemoteIntervention \\ &
	\threatNaturalDisaster \\ &
	\threatHumanDisaster \\ &
	\riskNameQuiebreAutenticacionDeTarjetaMagnetica \\ &
	\riskNameFaltaDeMonitoreo
}

\informationResource
{RTR\_SECUN\_001}
{Router secundario Cisco 2901, utilizado de punto intermedio hacia la red interna}
{Hardware TI}
{Sala de servidores - primer piso}
{Departamento de TI}
{1}{5}{5}
{
	\threatCVE{CVE-2017-3881}{Ejecución arbitraria de código (resuelto)} \\ &
	\threatResourceLost \\ &
	\threatRemoteIntervention \\ &
	\threatNaturalDisaster \\ &
	\threatHumanDisaster \\ &
	\riskNameQuiebreAutenticacionDeTarjetaMagnetica \\ &
	\riskNameFaltaDeMonitoreo
}

\informationResource
{SWI\_NODES\_001}
{Switch general Cisco Catalyst 2960, para nodo base del arbol de conectividad}
{Hardware TI}
{Sala de servidores - primer piso}
{Departamento de TI}
{1}{5}{5}
{
	\threatCVE{CVE-2017-3881}{Ejecución arbitraria de código (resuelto)} \\ &
	\threatResourceLost \\ &
	\threatRemoteIntervention \\ &
	\threatNaturalDisaster \\ &
	\threatHumanDisaster \\ &
	\riskNameQuiebreAutenticacionDeTarjetaMagnetica \\ &
	\riskNameFaltaDeMonitoreo
}

\informationResource
{SRV\_SHARE\_001}
{Dell PowerEdge R520 750W E5 2440}
{Hardware TI}
{Sala de servidores - primer piso}
{Departamento de TI}
{5}{5}{5}
{
	\threatRemoteIntervention \\ &
	\threatNaturalDisaster \\ &
	\threatHumanDisaster \\ &
	\riskNameQuiebreAutenticacionDeTarjetaMagnetica \\ &
	\riskNameFaltaDeMonitoreo
	
}


\informationResource
{OSS\_WINDO\_001}
{Windows Server 2019 Datacenter Edition}
{Sistemas Operativos}
{SRV_SHARE_001}
{Departamento de TI}
{2}{5}{5}
{
	Mas de 390 vulnerabilidades detectadas\footnote{https://www.cvedetails.com/product/50662/Microsoft-Windows-Server-2019.html?vendor_id=26} \\ &
	\threatRemoteIntervention \\ &
	\threatLicenceDependant \\ &
	\threatAntivirus
	\riskNameDesastresLogicos
	\riskNameFaltaDeEncriptado
	\riskNameCarenciaDeLicencias
	\riskNameFaltaDeProtocoloDeBorradoDeInformacion
}


\informationResource
{EXE\_EXCHA\_001}
{Módulo servidor para Microsoft Exchange 2016, para uso de correos corporativos de los funcionarios de la municipalidad.}
{Software}
{SRV_SHARE_001}
{Departamento de TI}
{5}{5}{5}
{
	\threatCVE{CVE-2018-8374}{Tampering Vulnerability existente al momento de un fallo en la información de los perfiles} \\ &
	\threatCVE{CVE-2018-8302}{Ejecución de código remota debido al fallo de manipulación de objetos en memoria, resultante en control total} \\ &
	\threatCVE{CVE-2018-8159}{ XSS resultante en elevación de privilegios por medio de requests web } \\ &
	\threatCVE{CVE-2018-8154}{Ejecución de código remota debido a la corrupción del manejo de objetos en memoria, resultante en control total} \\ &
	\threatCVE{CVE-2018-8153}{ Spoofing } \\ &
	\threatCVE{CVE-2018-8152}{ Elevación de privilegios } \\ &
	\threatCVE{CVE-2018-8151}{ Corrupción de memoria } \\ &
	\threatRemoteIntervention  \\ &
	\riskNameDesastresLogicos \\ &
	\riskNameFaltaDeEncriptado \\ &
	\riskNameCarenciaDeLicencias \\ &
	\riskNameFaltaDeProtocoloDeBorradoDeInformacion \\ &
	\riskNameRecuperacionDesastres \\ &
	\riskNameRespaldoInexistente \\ & 
	\riskNameFaltaDeDocumentacionEImplantacionDePoliticasParaEnvioDeCorreosMasivos
}

\informationResource
{ARC\_LOCAL\_001}
{Archivo general de la municipalidad - registro de documentos}
{Activos tangibles / Activos intangibles}
{Archivo - Primer piso}
{Departamento de Certificación y Archivos}
{5}{4}{2}
{
	\threatNoPhysicalBackup \\ &
	\threatNoDigitalBackup \\ &
	\threatHumanIntervention \\ &
	\threatNoTransactionRegistry
}

\phpApplicationBaseStructure[appdefinition={
	\informationResource
	{EXE\_WPRES\_001}
	{Servidor Wordpress 5.1 Beta3 para página institucional}
	{Software}
	{SRV\_SHARE\_001}
	{Departamento de TI}
	{1}{3}{3}
	{
		\threatCVE{CVE-2019-9787}{Ejecución remota de código por medio de CRSRF} \\ &
		\threatCVE{CVE-2019-16220}{Sanitización de wp\_validate manipula redirects} \\ &
		\threatHumanIntervention \\ &
		\threatRemoteIntervention
	}
},
apprefr=EXE\_WPRES\_001,
srvrefr=SRV\_SHARE\_001,
dbsrvrefr=EXE\_MYSQL\_001,
dbsrvrefconfi=1,
dbsrvrefinteg=3,
dbsrvrefavail=3,
dbinstrefr=EXE\_SQLTB\_001,
dbinstrefconfi=1,
dbinstrefinteg=3,
dbinstrefavail=3,
dbinstrefcve={
	\threatTransitive
},
phpinstrefr=EXE\_PHPSR\_001,
phpinstrefconfi=1,
phpinstrefinteg=3,
phpinstrefavail=3]{}



\phpApplicationBaseStructure[appdefinition={
	\informationResource
	{EXE\_ADMIN\_002}
	{Servidor con aplicativo de administración propia para municipio}
	{Software}
	{SRV\_SHARE\_002}
	{Departamento de TI}
	{5}{5}{5}
	{
		\threatUnkown \\ &
		\threatTransitive \\ &
		\riskNameRecuperacionDesastres \\ &
		\riskNameRespaldoInexistente \\ &
		\riskNameInteroperabilidadConEstandarInexistente \\ &
		\riskNameTransaccionRota \\ &
		\riskNameFaltaDeEncriptado \\ &
		\riskNameResiduosDeInformacion \\ &
		\riskNameEjecucionDeMalwareporFaltaDeSoftwareAv \\ &
		\riskNameDenegacionDeServicio \\ &
		\riskNameDesastresHumanos
	}
},
apprefr=EXE\_ADMIN\_002,
srvrefr=SRV\_SHARE\_002,
dbsrvrefr=EXE\_MYSQL\_002,
dbsrvrefconfi=4,
dbsrvrefinteg=5,
dbsrvrefavail=4,
dbinstrefr=EXE\_SQLTB\_002,
dbinstrefconfi=5,
dbinstrefinteg=5,
dbinstrefavail=5,
dbinstrefcve={
	\threatTransitive
	
},
phpinstrefr=EXE\_PHPSR\_002,
phpinstrefconfi=4,
phpinstrefinteg=5,
phpinstrefavail=4]{}


\phpApplicationBaseStructure[appdefinition={
	\informationResource
	{EXE\_ADMIN\_003}
	{Servidor con aplicativo de administración para archivo de municipio}
	{Software}
	{SRV\_SHARE\_003}
	{Departamento de TI}
	{5}{5}{5}
	{
		\threatUnkown \\ &
		\threatTransitive \\ &
		\riskNameQuiebreAutenticacionDeLlaveSeguridad \\ &
		\riskNameFaltaDeEncriptado \\ &
		\riskNameFaltaDeProtocoloDeBorradoDeInformacion
		
	}
},
apprefr=EXE\_ADMIN\_003,
srvrefr=SRV\_SHARE\_003,
dbsrvrefr=EXE\_MYSQL\_003,
dbsrvrefconfi=4,
dbsrvrefinteg=5,
dbsrvrefavail=4,
dbinstrefr=EXE\_SQLTB\_003,
dbinstrefconfi=5,
dbinstrefinteg=5,
dbinstrefavail=5,
dbinstrefcve={
	\threatTransitive
	
},
phpinstrefr=EXE\_PHPSR\_003,
phpinstrefconfi=4,
phpinstrefinteg=5,
phpinstrefavail=4]{}
