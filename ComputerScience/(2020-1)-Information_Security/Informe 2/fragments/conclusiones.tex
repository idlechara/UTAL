\section{Conclusiones}

Hemos presentado un analisis para la aplicación ThinkAgro desarrollada en 2018, desde el punto de vista de S-SDLC. Este análisis si bien es práctico, hemos demostrado que presenta graves falencias al momento de replicar conocimiento si es que no se tiene la aplicación como tal. Esto nos imposibilita de utilizar la metodología para poder ganar información de proyectos anteriores.

Por otro lado, es bastante práctica cuando los proyectos están en medio de su etapa de desarrollo o bien este ya ha terinado para poder generar recomendaciones. Bajo este contexto de desarrollo, podemos entregar las siguientes recomendaciones:

\begin{itemize}
    \item En caso de tener equipos altamente desbalanceados, es bueno mediar los asuntos de seguridad a nivel organizacional y separar las tareas técnicas a una persona especializada.
    \item Se prefiere utilizar un acercamiento de contenedores los cuales encapsulen de manera atómica los componentes de una aplicación a un despliegue donde todos los elementos esténm altamente cohesionados.
    \item Si es posible, reducir los nexos de comunicación a los necesarios para reducir el ruido en la transferencia de información en equipos grandes.
\end{itemize}

