\documentclass[11pt]{utalcaDoc}

\usepackage[utf8]{inputenc}
\usepackage[activeacute,spanish]{babel}
\usepackage{verbatim}

\usepackage{dirtree}

\title{{\bf Algoritmos y estructuras de datos}\\Tarea 4}
\author{Profesor: Rodrigo Paredes, Ayudante: Erik Regla}
\date{Plazo: Martes 9 de Diciembre de 2014. Estricto!!}

\begin{document}
\renewcommand{\figurename}{Figura~}
\renewcommand{\tablename}{Tabla~}

\maketitle

\section{Objetivos}


Es conocido que un 'arbol de b'usqueda binaria se puede desbalancear mucho.
En esta tarea vamos a considerar una variante del ABB que llamaremos
ABB con mediana de tres (ABB-m3), y se espera que el alumno realice la comparaci'on
entre el ABB-m3 con el ABB est'andar.


\section{Descripci'on de la tarea}

En esta tarea vamos a considerar una variante del ABB que llamaremos
ABB con mediana de tres (ABB-m3). La idea es bastante simple. Los nodos
internos tienen la misma estructura de un ABB, sin embargo en las hojas tendremos
nodos que pueden almacenar hasta tres llaves. La idea es que el ABB-m3 resultante
sea m'as balanceado que un ABB est'andar.

La {\bf b'usqueda} opera de la siguiente forma. En la bajada por el 'arbol, los nodos
internos permiten calcular la ruta hacia el nodo que debiese contener al elemento buscado,
y cuando se llega a una hoja, el elemento se compara con las llaves que est'en en el nodo.
Si durante el recorrido desde la ra'iz a la hoja no se encuentra al objeto se declara
que la b'usqueda fue infructuosa.
Considere la firma {\tt nodo* buscar(int x)} en C++ o {\tt nodo buscar(int x)} en Java,
considerando programaci'on orientada a objetos (OOP) de Java o C++.
Si prefiere usar programaci'on estructurada/procedural, use la firma
{\tt nodo* buscar(nodo *raiz, int x)} en C/C++ o {\tt nodo buscar(nodo raiz, int x)} en Java.

Las {\bf inserciones} se hacen en las hojas.
Si en la hoja hay espacio para una nueva llave, se inserta y las llaves se almacenan en orden.
Si la hoja est'a llena, primero se divide en tres, un nodo interno (que almacena la
llave del medio), una hoja izquierda (que almacena la llave menor) y
una hoja derecha (que almacena la llave mayor); y luego se inserta el nodo en
la hoja correspondiente.
Considere la firma {\tt nodo* insertar(int x)} en C++ o {\tt nodo insertar(int x)} en Java,
considerando OOP.
Si prefiere usar programaci'on estructurada/procedural, use la firma
{\tt nodo* insertar(nodo *raiz, int x)} en C/C++ o {\tt nodo insertar(nodo raiz, int x)} en Java.

Para la {\bf eliminaci'on}, es necesario localizar a la llave, una vez localizada se elimina y
si en el sub'arbol quedan tres o menos llaves se compacta a una hoja.
Como siempre, si la llave a eliminar es un nodo interno, tienen que hacer los ajustes
necesarios para poder hacer la eliminaci'on. Esto es, si s'olo hay llaves
en un subarbol, conectar al padre de la llave a eliminar con el hijo (saltando al
nodo), o si hay llaves en ambos hijos, buscar al mayor de los hijos menores
(o alternativamente, el menor de los mayores) para reemplazar al elemento borrado y
continuar con los ajustes.
Considere la firma {\tt nodo* borrar(int x)} en C++ o {\tt nodo borrar(int x)} en Java,
considerando OOP.
Si prefiere usar programaci'on estructurada/procedural, use la firma
{\tt nodo* borrar(nodo *raiz, int x)} en C/C++ o {\tt nodo borrar(nodo raiz, int x)} en Java.

Adicionalmente, implementen una funci'on para calcular la altura de un nodo,
{\tt int altura()} o {\tt int altura(Nodo raiz)}, dependiendo si es OOP o programaci'on
estructurada.

Noten que las tres operaciones, buscar, insertar y borrar, son muy parecidas a las de
un ABB est'andar, salvo que las hojas pueden tener hasta tres llaves.


\textbf{Experimentaci'on.} 

Para ejecutar el programa deben utilizar la siguiente l'inea:

\verb|tarea4 n|

El programa debe reportar para cada valor de $n$:
\begin{enumerate}
\item Tiempo de inserci'on de $n$ llaves.
\item Altura del 'arbol.
\item Tiempo de b'usqueda de 500000 llaves que pertenecen al 'arbol.
\item Tiempo de b'usqueda de 500000 llaves que no pertenecen al 'arbol.
\item Tiempo de borrado de 500000 llaves que pertenecen al 'arbol.
\end{enumerate}

Realicen pruebas para $n = 1, 2, \cdots, 10$,
es decir, 1MB, 2MB, $\ldots$, 1MB = $10^{6}$ enteros,
graficando el tiempo para cada una de las operaciones, y la altura de los 'arboles
obtenidos. Comparen los resultados con los que se obtienen de un ABB est'andar.

Para la generaci'on de los datos use el mismo generador de la tarea 3.


%\section{Pruebas}


\section{Restricciones}

\begin{itemize}
\item Los programas debe ser escritos en C, C++ o Java.
\item Plazo de entrega: Martes 9 de Diciembre de 2014. Si pueden venir antes
a mostrar la tarea mejor. As'i podr'an hacerle correcciones. Estricto!!
\item Queda prohibido utilizar bibliotecas o clases preconstruidas en el lenguaje
    que implementen 'arboles.
    USTED DEBE IMPLEMENTAR TODO.
\item La tarea es individual o en parejas.
\item Para la tarea tienen que entregar un informe de resultados junto al c'odigo
    fuente (el programa que implementa la tarea). No se aceptan c'odigos sin
    informe ni informes sin c'odigo.
\item Si la tarea no compila tiene nota 1.0 tanto en el c'odigo como en el informe.
\item Cualquier falta a las restricciones anteriores invalida irrevocablemente
la tarea.
\end{itemize}



\section{Ponderaci'on}

El informe equivale a un 30\% de la nota de tarea.
    
El c'odigo equivale a un 70\% de la nota de tarea.



\section{Entrega de la tarea}

La entrega de la tarea debe incluir (1) el código del programa que resuelve
la tarea y (2) un informe escrito donde se describa y documente el programa
entregado, y se incluyan ejemplos de entradas y salidas. La entrega de la tarea
se realizará a través de la plataforma {\tt lms.educandus.cl} hasta las 23:55 del día
del plazo final. Generen un único archivo que contenga tanto el informe como
los códigos para compilar. También se solicita que se adjunte un archivo
README que explique cómo compilar la tarea y un archivo EXECUTE el
cual contiene los comandos a usar para compilación y ejecución de la misma,
este debe tener permisos de ejecucion a modo de poder ser utilizado.
Adicionalmente, \emph{debe de entregarse una copia impresa del informe} a
subir en la plataforma {\tt lms.educandus.cl} a más tardar el día siguiente
del plazo final. La estructura del archivo a entregar debe tener el formato
ejemplificado en la Figura \ref{fig_estructura}.

\begin{figure}
\dirtree{%
.1 /.
.2 NombreApellido\textunderscore T4.zip.
.3 NombreApellido\textunderscore T4.
.4 fuente.
.5 EXECUTE.
.5 README.
.5 src.
.6 archivo1.cpp.
.6 archivo2.cpp.
.6 foo.bar.
.4 informe.
.5 NombreApellido\textunderscore T4.pdf.
}
\caption{Estructura de archivo de entrega.}
\label{fig_estructura}
\end{figure}
     

El contenido del informe debe seguir la siguiente pauta común para todas las tareas: 

\begin{itemize}
\item
Portada. El formato de la portada se muestra en la Figura \ref{fig1}.
%Portada. El formato de la portada se muestra a continuación:

\begin{figure}[tbp]
\epsffile{utalca.eps} %{\epsfxsize16mm \epsffile{utalca.eps}} %
\rlap{\kern 4mm \lower -9.5mm \hbox to0pt{{\Large \sc Universidad~~de~~Talca}\hfill}}%
\rlap{\kern 4mm \lower -5mm \hbox to0pt{\Large \sc Facultad~~de~~Ingeniería\hfill}}%
\rlap{\kern 4mm \lower -1mm \hbox to0pt{%
  {\large \sc Departamento~~de~~Ciencias~~de~~la~~Computación}\hfill}}%

\vspace{1cm}

\centerline{\huge \bf Informe Tarea X}
\vspace{0.8cm}
\centerline{\Large \bf Nombre de la Tarea X}

\vspace{2cm}

\hfill \begin{tabular}{ll}
Fecha:  & $<$fecha entrega$>$\\
Autor:  & $<$nombre alumno$>$\\
e-mail: & $<$correo del alumno$>$
\end{tabular}
\caption{Formato para la portada.}
\label{fig1}
\end{figure}

\item Introducción: Breve descripción del problema y de su solución.

\item Análisis del problema: Se expone en detalle el problema, los supuestos
que se van a ocupar, las situaciones de borde y la metodología
para abordar el problema.

\item Solución del problema:
  \begin{itemize}
  \item Algoritmo de solución: Se detallan los pasos a seguir para solucionar
  el problema. De ser necesario, se recomienda incluir figuras, tablas, etc.,
  que permitan mejorar la comprensión de la solución propuesta.
  
  \item Diagrama de Estados: Muestra en forma global el programa.
  
  \item Diseño: Explicitar las Pre y Post condiciones consideradas,
  mostrar los invariantes empleados.

  \item Implementación: Se debe mostrar el pseudo-código del programa que
  soluciona el problema, explicando lo que hace. Omita cualquier detalle de
  implementación que sea irrelevante para entender la solución del problema.
  Se recomienda usar nombres representativos para las variables.
  NOTA: El código generado para resolver la tarea debe corresponder al
  diseño descrito, preocúpese de comentar el código donde sea necesario
  para facilitar su lectura.
  
  \item Modo de uso: Se debe explicar el modo de uso del programa y el modo
  de compilación. Nota la tarea debe ser compilable en los computadores de la Universidad.
  \end{itemize}

\item Pruebas: Se debe mostrar las pruebas realizadas y sus resultados.
%Se debe señalar claramente cual es la llamada al programa y su salida.
El número de pruebas puede variar dependiendo del problema
En esta sección debe incluir las tablas y gráficos necesarios solicitados,
y el análisis de los resultados. En este capítulo adjunten las conclusiones
obtenidas de los resultados de la tarea.
%En general tres pruebas es suficiente.

\item Anexos: De ser necesario, cualquier información adicional se debe
agregar en los anexos y debe ser referenciada en alguna sección del
informe de la tarea. Dentro de los anexos se puede incluir un listado con
el programa completo que efectivamente fue compilado.

\end{itemize}

En este y los siguientes informes se va a valorar la ortografía y la
redacción de acuerdo a las siguientes reglas de penalización.

\begin{description}
\item[Ortografía]
Se permiten hasta 10 errores ortográficos en todo el informe.
Entre 11 y 20 errores descuenten 0.5 puntos al informe.
Entre 21 y 30 descuentan 1.0 puntos al informe.
Más de 30 descuenten 2.0 puntos al informe.
\item[Redacción]
Se permiten hasta 5 ideas o párrafos mal redactados (es decir que no se entiendan).
Entre 6 y 10 párrafos con problemas de redacción descuenten 0.5 puntos al informe.
Entre 11 y 14 párrafos descuentan 1.0 puntos al informe.
Más de 24 parrafos incomprensibles descuenten 2.0 puntos al informe.
\end{description}

\end{document}
